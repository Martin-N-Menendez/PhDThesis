\thispagestyle{plain}

\centerline{\begin{minipage}{10cm}
\
\vspace{70pt}
\begin{center}

{\huge\textit{Agradecimientos}}

\vspace{30pt}

A lo largo de mi vida he conocido muchas personas que me apoyaron, guiaron, influenciaron e inspiraron. No me alcanzan las palabras para agradecerles a todos ellos, pero trataré de, por medio de esta dedicatoria, darles su merecido reconocimiento.

\vspace{5pt}

A mis padres, Roberto y Silvina, por su infinito amor, su apoyo y su confianza en mi. A mis hermanos, Agustín y Sofía, por su cariño, por su amistad y por ser los mejores hermanos. A mis sobrinas, Valentina y Belén, por sus sonrisas, que deseo que siempre tengan. A mis abuelos, por motivarme a leer, aprender, construir, romper, inventar. A mi gata Mishu, por acompañarme desde el primer circuito.

\vspace{5pt}

A mi director y mentor, Ariel Lutenberg, el mejor docente que tuve la suerte de encontrar en la FIUBA, por todas las oportunidades brindadas, por ayudarme a cumplir este sueño. A mis pasados directores y profesores que me acompañaron a lo largo de este proyecto: Pablo Gómez, Facundo Larosa y Nicolás Álvarez. A mi compañero de laboratorio, Santiago Germino, que siempre me aconseja cómo mejorar en cada artículo. A cada miembro de GICSAFe por todo el trabajo realizado a lo largo de los años. A cada maestro que tuve en la escuela desde el jardín de infantes hasta el secundario, cada uno contribuyó con su granito de arena a mi amor por las matemáticas, la ciencia, la investigación y por incentivar mi curiosidad. Especialmente a mis maestras de primaria Leticia y Zulma, y a mi profesora de matemáticas Marisa Taller.

\vspace{5pt}

Es mi deseo que esta investigación contribuya a la ingeniería, a la ciencia, y a que nuestra sociedad tenga un sistema ferroviario mas robusto y seguro.

\end{center}
\end{minipage}}