\thispagestyle{plain}

\centerline{\begin{minipage}{10cm}

\vspace{70pt}
\begin{center}
    {\huge\textit{Resumen}}
    
    \vspace{30pt}
    
    Los sistemas de enclavamientos ferroviarios controlan el señalamiento de forma tal de garantizar que las formaciones se desplacen de forma segura, sin colisiones ni descarrilamientos. El señalamiento incluye los semáforos (o señales) que otorgan autoridad a los maquinistas para transitar por las vías, en función del estado de la infraestructura ferroviaria implicada, como pasos a nivel, desvíos, etc. El diseño del señalamiento es un proceso complejo que involucra el análisis de la red ferroviaria, la detección de zonas riesgosas y la correcta ubicación de las señales. La generación automática de la señalización es valiosa y útil cuando se desarrolla una nueva red ferroviaria o cuando se reactiva una red ferroviaria abandonada.

    \vspace{5pt}
    
    En este contexto, se diseñó un conjunto de herramientas que, a partir del trazado ferroviario, realizan de forma automática el diseño e implementación del señalamiento utilizando un lenguaje de descripción de hardware. La etapa encargada del diseño del señalamiento ferroviario es el Analizador de Redes Ferroviarias (RNA, por sus siglas en inglés). En tanto que de la implementación en VHDL se encarga el Generador Automático de Código (ACG, por sus siglas en inglés). Ambas herramientas intercambian información entre sí y con los usuarios, mediante el estándar abierto de intercambio de datos de infraestructura ferroviaria railML. Finalmente, el Generador Automático de Interfaz Gráfica (AGG, por sus siglas en inglés), construye una interfaz interactiva para el operador, que permite visualizar el estado del enclavamiento y comandarlo en tiempo real.

    \vspace{5pt}
    
    El señalamiento generado incluye el número de las señales necesarias, su posición y orientación, además de la tabla de enclavamiento. La tabla de enclavamientos, el cumplimiento de los principios de diseño ferroviario y la sintaxis del archivo railML generado son validados por el mismo RNA. El archivo railML generado junto con el modelo de comportamiento dinámico, definido en redes de Petri, es utilizado por el ACG para implementar el sistema de enclavamiento. La interfaz gráfica generada a medida de cada sistema por el AGG permite interactuar con el sistema implementado.
    
\end{center}
\end{minipage}}