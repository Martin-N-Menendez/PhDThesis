\subsection{Sistema generado por el ACG}

	En base a la red de grafos, ilustrada en la Figura \ref{fig:EJ7_8}, el ACG determinó la siguiente cantidad de elementos, tal puede visualizarse en el Código \ref{lst:EJ7_8}.
	
	\begin{lstlisting}[language = {}, caption = Cantidad de elementos a implementar por el ACG, label = {lst:EJ7_8}]
	n_netElements:7
	n_switch:3
	n_doubleSwitch:0
	n_borders:0
	n_buffers:5
	n_levelCrossings:0
	n_platforms:0
	n_scissorCrossings:0
	n_signals:13
	N : 47
	M : 40
	\end{lstlisting}
	
	El código VHDL generado por el ACG es importado en un proyecto de Vivado, donde es sintetizado e implementado para generar el bitstream que será utilizado para programar la FPGA. La cantidad de elementos de la FPGA utilizados por el sistema post-síntesis y post-implementación, así como el porcentaje de uso de la plataforma, son detallados en la Tabla \ref{Tab:tabla_ACG_7}.
	
	\begin{table}[H]
		{
			\caption{Síntesis e implementación del ejemplo 7 generado por el ACG.}
			\label{Tab:tabla_ACG_7}
			\centering
			%\small
			%\centering
			\begin{center}
				\resizebox{0.7\textwidth}{!}{
					\begin{tabular}{ c c c c }
						\hline	
						Recursos & Síntesis & Implementación & Uso \\	
						\hline
						LUT & 2102 & 2102 & 3.95\%\\
						FF & 2983 & 2983 & 2.80\%\\
						IO & 16 & 16 & 12.80\%\\
						BUFG & 1 & 1 & 3.13\%\\
						\hline
					\end{tabular}
				}
			\end{center}
		}    
	\end{table}
	
	En este ejemplo, la cantidad de recursos utilizados es baja y el tiempo de sintetización e implementación es de 35 segundos y 36 segundos respectivamente.