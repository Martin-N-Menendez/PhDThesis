\subsection{Señalamiento generado por el RNA}

    El RNA también exporta la misma información mostrada en el Código \ref{lst:EJ7_8} en una hoja de cálculo, similar a la que se visualiza en la Tabla \ref{Tab:tabla_generated_7}.
    
    \begin{table}[H]
        {
        \caption{Tabla de enclavamiento del ejemplo 7 generada por el RNA.}
        \label{Tab:tabla_generated_7}
        \centering
        \resizebox{1\textwidth}{!}{
            \begin{tabular}{ c c c c c c c }
                \hline	
                    Ruta & Inicio & Final & Cambio & Plataforma & Cruce & netElement \\	
                \hline
                    R$_{01}$  & T$_{02}$ & S$_{14}$ & - & - & - & ne$_{01}$\\
                    R$_{02}$  & T$_{04}$ & T$_{01}$ & Sw$_{14}^{R}$+Sw$_{18}^{R}$ & - & - & ne$_{31}$-ne$_{01}$\\
                    R$_{03}$  & T$_{06}$ & T$_{01}$ & Sw$_{14}^{N}$+Sw$_{18}^{R}$ & - & - & ne$_{32}$-ne$_{01}$\\
                    R$_{04}$  & T$_{08}$ & H$_{20}$ & - & - & - & ne$_{42}$\\
                    R$_{05}$  & T$_{10}$ & T$_{07}$ & Sw$_{19}^{N}$ & - & - & ne$_{43}$-ne$_{42}$\\
                    R$_{06}$  & S$_{14}$ & B$_{18}$ & Sw$_{18}^{N}$ & - & - & ne$_{01}$-ne$_{41}$\\
                    R$_{07}$  & S$_{14}$ & T$_{05}$ & Sw$_{18}^{R}$+Sw$_{14}^{N}$ & - & - & ne$_{01}$-ne$_{32}$\\
                    R$_{08}$  & S$_{14}$ & T$_{03}$ & Sw$_{18}^{R}$+Sw$_{14}^{R}$ & - & - & ne$_{01}$-ne$_{31}$\\
                    R$_{09}$  & B$_{18}$ & T$_{07}$ & Sw$_{19}^{R}$ & - & - & ne$_{41}$-ne$_{42}$\\
                    R$_{10}$  & H$_{20}$ & T$_{09}$ & Sw$_{19}^{N}$ & - & - & ne$_{42}$-ne$_{43}$\\
                    R$_{11}$  & H$_{20}$ & T$_{01}$ & Sw$_{19}^{R}$+Sw$_{18}^{N}$ & - & - & ne$_{42}$-ne$_{01}$\\
                \hline
            \end{tabular}
        }
     }
    \end{table}
    
    En una primera inspección podemos ver que el nuevo señalamiento tiene 11 rutas, versus las 8 rutas del señalamiento original. Esto se debe a que el RNA añade protecciones extras para los finales de vía absolutos, ausentes en el señalamiento original.