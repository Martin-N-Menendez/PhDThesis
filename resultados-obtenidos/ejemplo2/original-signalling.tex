\subsection{Señalamiento original}

    \lipsum[1]
    
    \begin{table}[H]
        {
        \caption{Tabla de enclavamiento original del ejemplo 2.}
        \label{Tab:tabla_original_2}
        \centering
        \resizebox{1\textwidth}{!}{
            \begin{tabular}{ c c c c c c c }
                \hline	
                    Ruta & Inicio & Final & Cambio & Plataforma & Cruce & netElement \\	
                \hline
                    R$_{01}$  & S$_{07}$ & S$_{11}$ & Sw$_{01}^{N}$ & - & - & ne$_{14}$-ne$_{16}$\\
                    R$_{02}$  & S$_{08}$ & S$_{11}$ & Sw$_{01}^{R}$ & - & - & ne$_{15}$-ne$_{16}$\\
                    R$_{03}$  & S$_{09}$ & S$_{12}$ & Sw$_{02}^{N}$ & - & - & ne$_{18}$-ne$_{16}$\\
                    R$_{04}$  & S$_{10}$ & S$_{13}$ & Sw$_{03}^{N}$ & - & - & ne$_{20}$-ne$_{19}$\\
                    R$_{05}$  & S$_{10}$ & S$_{12}$ & Sw$_{03}^{R}$+Sw$_{02}^{R}$ & - & - & ne$_{20}$-ne$_{17}$-ne$_{16}$\\  
                \hline
            \end{tabular}
        }
     }
    \end{table}