\section{Señalamiento original}

    El señalamiento original, ilustrado en la Figura \ref{fig:EJ5_2}, incluye señales de parada próximas a los finales de vías absolutos (S13, S14, S15, S16), señales de maniobras antes de converger en una vía principal (S03, S04, S05, S06, S08, S09, S10, S11) y señales múltiples para cambios de vías divergentes (S01, S02, S07, S12).
    
    \begin{figure}[H]
    	\centering
    	\includegraphics[width=1\textwidth]{resultados-obtenidos/ejemplo5/images/5_original.png}
    	\centering\caption{Señalamiento original del ejemplo 5.}
    	\label{fig:EJ5_2}
    \end{figure}
    
    Estas señales permiten definir hasta un máximo de 16 rutas, todas ellas detalladas en la Tabla \ref{Tab:tabla_original_5}. En una primera inspección, se puede comprobar que todos los elementos ferroviarios son alcanzados por al menos una de las rutas, en al menos una dirección. Además, todos los cambios de vías son utilizados, de forma simple o compuesta. 
    
    \begin{table}[H]
        {
        \caption{Tabla de enclavamiento original del ejemplo 5.}
        \label{Tab:tabla_original_5}
        %\centering
    	\begin{center}
        \resizebox{0.8\textwidth}{!}{
            \begin{tabular}{ c c c c c c c }
                \hline	
                    Ruta & Inicio & Final & Cambio & Plataforma & Cruce & netElement \\	
                \hline
                    R$_{01}$  & S$_{01}$ & S$_{06}$ & Sw$_{01}^{N}$ & - & - & ne$_{01}$-ne$_{02}$\\
                    R$_{02}$  & S$_{05}$ & S$_{13}$ & Sw$_{01}^{N}$ & - & - & ne$_{02}$-ne$_{01}$\\
                    R$_{03}$  & S$_{01}$ & S$_{03}$ & Sw$_{01}^{R}$ & - & - & ne$_{01}$-ne$_{03}$\\
                    R$_{04}$  & S$_{04}$ & S$_{13}$ & Sw$_{01}^{R}$ & - & - & ne$_{03}$-ne$_{01}$\\
                    R$_{05}$  & S$_{02}$ & S$_{04}$ & Sw$_{02}^{R}$ & - & - & ne$_{04}$-ne$_{02}$\\
                    R$_{06}$  & S$_{06}$ & S$_{15}$ & Sw$_{02}^{N}$ & - & - & ne$_{02}$-ne$_{04}$\\
                    R$_{07}$  & S$_{02}$ & S$_{05}$ & Sw$_{02}^{N}$ & - & - & ne$_{04}$-ne$_{03}$\\
                    R$_{08}$  & S$_{03}$ & S$_{15}$ & Sw$_{02}^{R}$ & - & - & ne$_{03}$-ne$_{04}$\\
                    R$_{09}$  & S$_{07}$ & S$_{11}$ & Sw$_{03}^{N}$ & - & - & ne$_{05}$-ne$_{06}$\\
                    R$_{10}$  & S$_{10}$ & S$_{14}$ & Sw$_{03}^{N}$ & - & - & ne$_{06}$-ne$_{05}$\\
                    R$_{11}$  & S$_{07}$ & S$_{09}$ & Sw$_{03}^{R}$ & - & - & ne$_{05}$-ne$_{07}$\\
                    R$_{12}$  & S$_{08}$ & S$_{14}$ & Sw$_{03}^{R}$ & - & - & ne$_{07}$-ne$_{05}$\\
                    R$_{13}$  & S$_{12}$ & S$_{10}$ & Sw$_{04}^{N}$ & - & - & ne$_{08}$-ne$_{06}$\\
                    R$_{14}$  & S$_{11}$ & S$_{16}$ & Sw$_{04}^{N}$ & - & - & ne$_{06}$-ne$_{08}$\\
                    R$_{15}$  & S$_{12}$ & S$_{08}$ & Sw$_{04}^{R}$ & - & - & ne$_{08}$-ne$_{07}$\\
                    R$_{16}$  & S$_{09}$ & S$_{16}$ & Sw$_{04}^{R}$ & - & - & ne$_{07}$-ne$_{08}$\\
                \hline
            \end{tabular}
        }
    	\end{center}
     }
    \end{table}
    
    Todas las rutas abarcan mas de un \textit{netElement}, como por ejemplo la ruta R16 que comienza en la señal S09 y finaliza en la señal S16, atravesando los \textit{netElements} ne07 y ne08, utilizando el cambio de vías Sw04 en posición reversa.
    