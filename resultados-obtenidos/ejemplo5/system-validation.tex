\section{Validación del sistema de enclavamientos}

    La validación de las rutas de la tabla de enclavamientos es realizada por el RNA aplicando el Algoritmo \ref{alg:interlocking_tables}, explicado en la Sección \ref{sec:validar_tabla}. Las 16 rutas del señalamiento original (Tabla \ref{Tab:tabla_original_5}) tienen 16 rutas equivalentes en el señalamiento generado por el RNA (Tabla \ref{Tab:tabla_generated_5}), tal como se puede visualizar en la Tabla \ref{Tab:tabla_validation_5}, generada automáticamente por el RNA.

    \begin{table}[H]
        {
        \caption{Equivalencias entre las rutas originales y las generadas por el RNA.}
        \label{Tab:tabla_validation_5}
        %\centering
        \begin{center}
        \resizebox{0.5\textwidth}{!}{
            \begin{tabular}{ c c c c }
                \hline	
                    Original & Señales & RNA & Señales \\	
                \hline
                    R$_{01}$ & S$_{01}$-S$_{06}$ & R$_{01}$ & T$_{02}$-C$_{25}$ \\
                    R$_{02}$ & S$_{05}$-S$_{13}$ & R$_{09}$ & C$_{21}$-T$_{01}$ \\
                    R$_{03}$ & S$_{01}$-S$_{03}$ & R$_{02}$ & T$_{02}$-B$_{26}$ \\
                    R$_{04}$ & S$_{04}$-S$_{13}$ & R$_{10}$ & B$_{22}$-T$_{01}$ \\
                    R$_{05}$ & S$_{02}$-S$_{04}$ & R$_{03}$ & T$_{04}$-C$_{21}$ \\
                    R$_{06}$ & S$_{06}$-S$_{15}$ & R$_{11}$ & C$_{25}$-T$_{03}$ \\
                    R$_{07}$ & S$_{02}$-S$_{05}$ & R$_{04}$ & T$_{04}$-B$_{26}$ \\
                    R$_{08}$ & S$_{03}$-S$_{15}$ & R$_{12}$ & B$_{26}$-T$_{03}$ \\
                    R$_{09}$ & S$_{07}$-S$_{11}$ & R$_{05}$ & T$_{06}$-C$_{33}$ \\
                    R$_{10}$ & S$_{10}$-S$_{14}$ & R$_{13}$ & C$_{29}$-T$_{05}$ \\
                    R$_{11}$ & S$_{07}$-S$_{09}$ & R$_{06}$ & T$_{06}$-B$_{34}$ \\
                    R$_{12}$ & S$_{08}$-S$_{14}$ & R$_{14}$ & B$_{30}$-T$_{05}$ \\
                    R$_{13}$ & S$_{12}$-S$_{10}$ & R$_{07}$ & T$_{08}$-C$_{29}$ \\
                    R$_{14}$ & S$_{11}$-S$_{16}$ & R$_{15}$ & C$_{33}$-T$_{07}$ \\
                    R$_{15}$ & S$_{12}$-S$_{08}$ & R$_{08}$ & T$_{08}$-B$_{30}$ \\
                    R$_{16}$ & S$_{09}$-S$_{16}$ & R$_{16}$ & B$_{34}$-T$_{07}$ \\
                \hline
            \end{tabular}
     		}
     	\end{center}
        }    
    \end{table}
    
    Todas las rutas generadas por el RNA (Tabla \ref{Tab:tabla_generated_5}) tienen una ruta equivalente en el señalamiento original (Tabla \ref{Tab:tabla_original_5}). El RNA no ha generado nuevas señales ni rutas ya que el señalamiento original ya satisfacía todos los requerimientos de seguridad necesarios. Tampoco se eliminaron señales ni rutas, lo que demuestra que el RNA puede mejorar los sistemas de enclavamientos cuando es posible, o igualarlos cuando se ha alcanzado la cota máxima de seguridad, sin recurrir a redundancias improductivas.
    
    Para finalizar, el RNA comprueba los principios de señalamiento ferroviario explicados en la Sección \ref{sec:validar_principios}, aplicando los algoritmos indicados, de los cuáles se obtuvieron los siguientes resultados:
    
    \begin{itemize}
    	\item Principio de autoridad (Algoritmo \ref{alg:ppio_autoridad}): cobertura del 100\% de los \textit{netElements}.
    	\item Principio de claridad (Algoritmo \ref{alg:ppio_claridad}): rutas 100\% independientes.
    	\item Principio de anticipación (Algoritmo \ref{alg:ppio_anticipacion}): cobertura del 100\% de los puntos críticos.
    	\item Principio de granularidad (Algoritmo \ref{alg:ppio_granularidad}): 100\% de rutas divididas a su mínima expresión.
    	\item Principio de terminalidad (Algoritmo \ref{alg:ppio_terminalidad}): 100\% de finales de vías protegidos.
    	\item Principio de infraestructura (Algoritmo \ref{alg:ppio_infraestructura}): 100\% de infraestructura protegida.
    	\item Principio de no bloqueo (Algoritmo \ref{alg:ppio_nobloqueo}): 100\% de cambios de vías protegidos.
    \end{itemize}	