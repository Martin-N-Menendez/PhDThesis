\subsection{Señalamiento generado por el RNA}

    El RNA también exporta la misma información mostrada en el Código \ref{lst:EJ5_8} en una hoja de cálculo, similar a la que se visualiza en la Tabla \ref{Tab:tabla_generated_5}.
    
    \begin{table}[!h]
        {
        \caption{Tabla de enclavamiento del ejemplo 5 generada por el RNA.}
        \label{Tab:tabla_generated_5}
        \centering
        \resizebox{1\textwidth}{!}{
            \begin{tabular}{ c c c c c c c }
                \hline	
                    Ruta & Inicio & Final & Cambio & Plataforma & Cruce & netElement \\	
                \hline
                    R$_{01}$  & T$_{02}$ & C$_{25}$ & Sw$_{01}^{N}$ & - & - & ne$_{01}$-ne$_{03}$\\
                    R$_{02}$  & T$_{02}$ & B$_{26}$ & Sw$_{01}^{R}$ & - & - & ne$_{01}$-ne$_{04}$\\
                    R$_{03}$  & T$_{04}$ & C$_{21}$ & Sw$_{02}^{N}$ & - & - & ne$_{04}$-ne$_{03}$\\
                    R$_{04}$  & T$_{04}$ & B$_{26}$ & Sw$_{02}^{R}$ & - & - & ne$_{04}$-ne$_{04}$\\
                    R$_{05}$  & T$_{06}$ & C$_{33}$ & Sw$_{03}^{N}$ & - & - & ne$_{05}$-ne$_{06}$\\
                    R$_{06}$  & T$_{06}$ & B$_{34}$ & Sw$_{03}^{R}$ & - & - & ne$_{05}$-ne$_{07}$\\
                    R$_{07}$  & T$_{08}$ & C$_{29}$ & Sw$_{04}^{N}$ & - & - & ne$_{08}$-ne$_{06}$\\
                    R$_{08}$  & T$_{08}$ & B$_{30}$ & Sw$_{04}^{R}$ & - & - & ne$_{08}$-ne$_{07}$\\
                    R$_{09}$  & C$_{21}$ & T$_{01}$ & Sw$_{01}^{N}$ & - & - & ne$_{02}$-ne$_{01}$\\
                    R$_{10}$  & B$_{22}$ & T$_{01}$ & Sw$_{01}^{R}$ & - & - & ne$_{03}$-ne$_{01}$\\
                    R$_{11}$  & C$_{25}$ & T$_{03}$ & Sw$_{02}^{N}$ & - & - & ne$_{02}$-ne$_{04}$\\
                    R$_{12}$  & B$_{26}$ & T$_{03}$ & Sw$_{02}^{R}$ & - & - & ne$_{03}$-ne$_{04}$\\
                    R$_{13}$  & C$_{29}$ & T$_{05}$ & Sw$_{03}^{N}$ & - & - & ne$_{06}$-ne$_{05}$\\
                    R$_{14}$  & B$_{30}$ & T$_{05}$ & Sw$_{03}^{R}$ & - & - & ne$_{07}$-ne$_{05}$\\
                    R$_{15}$  & C$_{33}$ & T$_{07}$ & Sw$_{04}^{N}$ & - & - & ne$_{06}$-ne$_{08}$\\
                    R$_{16}$  & B$_{34}$ & T$_{07}$ & Sw$_{04}^{R}$ & - & - & ne$_{07}$-ne$_{08}$\\
                \hline
            \end{tabular}
        }
     }
    \end{table}
    
    En una primera inspección podemos ver que el nuevo señalamiento tiene 16 rutas, al igual que el señalamiento original que también posee 16 rutas. Esto se debe a que el señalamiento original ya contemplaba todas las rutas posibles y el RNA generó un nuevo señalamiento equivalente, sin eliminar rutas ni tampoco agregando rutas que no sean útiles.