\subsection{Generación de señalamiento paso a paso}

	Al ejecutar el RNA, primero detectará todos los \textit{netElements}, sus coordenadas iniciales y finales en la topología, y el sentido en el que fueron definidas. El resultado obtenido se muestra en el Cóodigo \ref{lst:EJ5_1}.
	
	\begin{lstlisting}[language = {}, caption = Detección de \textit{netElements} por parte del RNA , label = {lst:EJ5_1}]
		###### Starting Railway Network Analyzer #####
		Reading .railML file
		Creating railML object
		Analyzing railML object
		Analyzing graph
		ne01 [-1451, -150] [-763, -150] >>
		ne02 [-763, -150] [736, -150] >>
		ne03 [-763, -150] [736, -150] >>
		ne04 [736, -150] [1451, -150] >>
		ne05 [-1451, -450] [-763, -450] >>
		ne06 [-763, -450] [736, -450] >>
		ne07 [-763, -450] [736, -450] >>
		ne08 [736, -450] [1451, -450] >>
		The network is not connected
	\end{lstlisting}
	
	Por ejemplo, el \textit{netElement} ne08 inicia en la coordenada (736;-450) y finaliza en la coordenada (1451;-450). El símbolo $>>$ indica que ne08 se encuentra definido de izquierda a derecha, ya que la componente x de la coordenada final es mayor a la de la coordenada inicial, teniendo la misma componente y. Además, se puede comprobar que la lista obtenida en consistente con la Figura \ref{fig:EJ5_2}. Por ejemplo, ne01, ne02 y ne03 comparten la coordenada (736;-450), que coincide con la coordenada del cambio de vías Sw01.
	
	A continuación, el RNA detectará la infraestructura ferroviaria, las curvas peligrosas y los puntos medios de los netElements que el RNA considera demasiado largos. El resultado de este proceso se puede visualizar en el Código \ref{lst:EJ5_2} y puede leerse también en el archivo Infrastructure.RNA.
	
	\begin{lstlisting}[language = {}, caption = Detección de puntos críticos por parte del RNA , label = {lst:EJ5_2}]
	Analyzing infrastructure --> Infrastructure.RNA
	Detecting Danger --> Safe_points.RNA
	ne01 has a RailJoint[J01] @ [-1101, -150]     
	ne02 has a Middle point @ [-548.9, -150]      
	ne02 has a Middle point @ [-334.7, -150]      
	ne02 has a Middle point @ [-120.6, -150]
	ne02 has a Middle point @ [93.6, -150]
	ne02 has a Middle point @ [307.7, -150]
	ne02 has a Middle point @ [521.9, -150]
	ne03 has a RailJoint[J05] @ [-11, 150]
	ne03 has a Curve(3 lines) @ [[-463, 150], [436, 150]]
	ne04 has a RailJoint[J08] @ [1100, -150]
	ne05 has a RailJoint[J09] @ [-1118, -450]
	ne06 has a Middle point @ [-548.9, -450]
	ne06 has a Middle point @ [-334.7, -450]
	ne06 has a Middle point @ [-120.6, -450]
	ne06 has a Middle point @ [93.6, -450]
	ne06 has a Middle point @ [307.7, -450]
	ne06 has a Middle point @ [521.9, -450]
	ne07 has a RailJoint[J12] @ [-5, -750]
	ne07 has a Curve(3 lines) @ [[-463, -750], [436, -750]]
	ne08 has a RailJoint[J16] @ [1100, -450]
	\end{lstlisting}
	
	Una vez que el RNA detectó cada punto crítico de la red ferroviaria, procede a generar el señalamiento. El orden de generación no es importante, pero para poder describirlo de forma consistente se iniciará generando el señalamiento para proteger los finales de vías, las junturas entre rieles, las plataformas, los cruces de vía y los cambios de vías. Luego se procederá a mostrar el señalamiento pre y post simplificación. Las señales generadas para proteger los finales de vías relativos y absolutos son ilustradas en la Figura \ref{fig:EJ5_3}.
	
	\begin{figure}[H]
		\centering
		\includegraphics[width=1\textwidth]{resultados-obtenidos/ejemplo5/images/5_step1.png}
		\centering\caption{Señalamiento generado por el RNA para proteger el fin de vía.}
		\label{fig:EJ5_3}
	\end{figure}
	
	Los finales de vías absolutos son protegidos por las señales de parada T01, T03, T05 y T07, y las señales de partida son T02, T04, T06 y T08. No existen finales de vías relativos que proteger.
	
	La Figura \ref{fig:EJ5_4} ilustra la generación de señales destinadas a proteger las junturas entre los rieles. Las señales generadas son todas las señales entre J09 y J20, indicadas en color rojo.
	
	\begin{figure}[H]
		\centering
		\includegraphics[width=1\textwidth]{resultados-obtenidos/ejemplo5/images/5_step2.png}
		\centering\caption{Señalamiento generado por el RNA para proteger las junturas.}
		\label{fig:EJ5_4}
	\end{figure}
	
	Al generar el señalamiento para proteger la infraestructura, tal como se explicó en la Sección \ref{sec:horizontal}, el Algoritmo \ref{alg:horizontal} simplificará las señales entre dos elementos ferroviarios si no existe espacio suficiente entre ellos. El señalamiento generado para proteger las plataformas y los cruces de vía se ilustra en rojo en la Figura \ref{fig:EJ5_5}. Al no existir plataformas o cruces de vías que proteger, ninguna señal fue generada por el RNA.
	
	\begin{figure}[H]
		\centering
		\includegraphics[width=1\textwidth]{resultados-obtenidos/ejemplo5/images/5_step3.png}
		\centering\caption{Señalamiento generado por el RNA para proteger plataformas y cruces de vía.}
		\label{fig:EJ5_5}
	\end{figure}
	
	El RNA generó las señales C21, S23, B26 y H24 para proteger el cambio de vías Sw02; las señales C25, S27, B22 y H28 para proteger el cambio de vías Sw02; las señales C29, S31. B34 y H32 para proteger el cambio de vías Sw03 y las señales C33, S35, B30 y H36 para proteger el cambio de vías Sw04. Las señales mencionadas se encuentran resaltadas en rojo en la Figura \ref{fig:EJ5_6}.
	
	\begin{figure}[H]
		\centering
		\includegraphics[width=1\textwidth]{resultados-obtenidos/ejemplo5/images/5_step4.png}
		\centering\caption{Señalamiento generado por el RNA para proteger los cambios de vías.}
		\label{fig:EJ5_6}
	\end{figure}
	
	Una vez obtenido todo el señalamiento, el RNA procede a simplificar las señales redundantes, repetidas o cuyas funciones o ubicaciones se superponen entre sí. El proceso de simplificación de señales fue explicado en la Sección \label{sec:simplificacion}. En este ejemplo, el Algoritmo \ref{alg:vertical} de herencia vertical no fue aplicado, al no cumplirse las condiciones de aplicación.
	
	Las señales simplificadas al aplicar el Algoritmo \ref{alg:horizontal} de herencia horizontal son: J09, J10, J11, J12, J13, J14, J15, J16, J17, J18, S23, H24, S27, H28, S31, H32, S35 y H23. Las señales J09, S23 y H24 fueron eliminadas por su cercanía con la señal T02, con la cual comparten dirección y sentido. Lo mismo ocurre entre las señales J14, S27 y H28, borradas por la señal T04; entre las señales J15, S31 y H32, borradas por la señal T06; entre las señales J20, S35 y H36, borradas por la señal T08;entre la señal J10 y la señal T01; la señal J13 y la señal T03; la señal J16 y la señal T05; y entre la señal J19 y la señal T07. En todos los casos, se aplicó el Algoritmo \ref{alg:horizontal}, diseñado para agrupar objetos cercanos como un único objeto, generando el señalamiento acorde a los elementos contenidos en cada extremo del nuevo elemento contenedor.
	
	Finalmente, las señales son simplificadas aplicando el Algoritmo \ref{alg:reduction} de eliminación por prioridad de señales. El resultado de este proceso es detallado en el Código \ref{lst:EJ5_3}.
	
	\begin{lstlisting}[language = {}, caption = Reducción de señalamiento por prioridad de señales, label = {lst:EJ5_3}]
	Reducing redundant signals
	removing sig10 for sig01
	removing sig09 for sig02
	removing sig23 for sig02
	removing sig24 for sig02
	removing sig13 for sig03
	removing sig14 for sig04
	removing sig27 for sig04
	removing sig28 for sig04
	removing sig16 for sig05
	removing sig15 for sig06
	removing sig31 for sig06
	removing sig32 for sig06
	removing sig19 for sig07
	removing sig20 for sig08
	removing sig35 for sig08
	removing sig36 for sig08
	removing sig11 for sig26
	removing sig12 for sig22
	removing sig17 for sig34
	removing sig18 for sig30
	\end{lstlisting}
	
	El resultado de la simplificación del señalamiento se ilustra en la Figura \ref{fig:EJ5_7}.
	
	\begin{figure}[H]
		\centering
		\includegraphics[width=1\textwidth]{resultados-obtenidos/ejemplo5/images/5_RNA.png}
		\centering\caption{Señalamiento generado y simplificado por el RNA.}
		\label{fig:EJ5_7}
	\end{figure}
	
	Al finalizar la generación del señalamiento, el RNA debe detectar todas las posibles rutas admitidas por la red para crear la tabla de enclavamientos. El RNA exporta los resultados del análisis en los siguientes cuatro documentos:
	
	Infrastructure.RNA (Código \ref{lst:EJ5_4}): resumen de cada elemento ferroviario en cada \textit{netElement}.
	
	\begin{lstlisting}[language = {}, caption = Infrastructure.RNA, label = {lst:EJ5_4}]
	Nodes: 8|Switches: 4|Signals: 0|Detectors: 6|Ends: 4|Barriers: 0
	Node ne01:
		Track = track1
		TrainDetectionElements -> tde89
		Type -> insulatedRailJoint
		Type = BufferStop -> ['bus174']
		Neighbours = 2 -> ['ne02', 'ne03']
		Switches -> Sw01
			ContinueCourse -> right -> ne02
			BranchCourse -> left -> ne03
	Node ne02:
		Track = track5
		Neighbours = 3 -> ['ne01', 'ne03', 'ne04']
	Node ne03:
		Track = track6
		TrainDetectionElements -> tde93
		Type -> insulatedRailJoint
		Neighbours = 3 -> ['ne01', 'ne02', 'ne04']
	Node ne04:
		Track = track3
		TrainDetectionElements -> tde96
		Type -> insulatedRailJoint
		Type = BufferStop -> ['bus176']
		Neighbours = 2 -> ['ne02', 'ne03']
		Switches -> Sw02
			ContinueCourse -> left -> ne02
			BranchCourse -> right -> ne03
	Node ne05:
		Track = track2
		TrainDetectionElements -> tde105
		Type -> insulatedRailJoint
		Type = BufferStop -> ['bus175']
		Neighbours = 2 -> ['ne06', 'ne07']
		Switches -> Sw03
			ContinueCourse -> left -> ne06
			BranchCourse -> right -> ne07
	Node ne06:
		Track = track7
		Neighbours = 3 -> ['ne05', 'ne07', 'ne08']
	Node ne07:
		Track = track8
		TrainDetectionElements -> tde108
		Type -> insulatedRailJoint
		Neighbours = 3 -> ['ne05', 'ne06', 'ne08']
	Node ne08:
		Track = track4
		TrainDetectionElements -> tde112
		Type -> insulatedRailJoint
		Type = BufferStop -> ['bus177']
		Neighbours = 2 -> ['ne06', 'ne07']
		Switches -> Sw04
			ContinueCourse -> right -> ne06
			BranchCourse -> left -> ne07
	\end{lstlisting}
	
	SafePoints.RNA (Código \ref{lst:EJ5_5}): coordenadas absolutas de cada punto donde puede colocarse una señal, en cada \textit{netElement}.
	
	\begin{lstlisting}[language = {}, caption = SafePoints.RNA, label = {lst:EJ5_5}]
	ne01:
		Next: [[-1201.0, -150]]
		Prev: [[-1001.0, -150]]
	ne02:
		Next: [[-548.9, -150], [-334.7, -150], [-120.6, -150], [93.6, -150], [307.7, -150], [521.9, -150]]
		Prev: [[-548.9, -150], [-334.7, -150], [-120.6, -150], [93.6, -150], [307.7, -150], [521.9, -150]]
	ne03:
		Next: [[-111.0, 150], [336.0, 150]]
		Prev: [[89.0, 150], [-363.0, 150]]
	ne04:
		Next: [[1000.0, -150]]
		Prev: [[1200.0, -150]]
	ne05:
		Next: [[-1218.0, -450]]
		Prev: [[-1018.0, -450]]
	ne06:
		Next: [[-548.9, -450], [-334.7, -450], [-120.6, -450], [93.6, -450], [307.7, -450], [521.9, -450]]
		Prev: [[-548.9, -450], [-334.7, -450], [-120.6, -450], [93.6, -450], [307.7, -450], [521.9, -450]]
	ne07:
		Next: [[-105.0, -750], [336.0, -750]]
		Prev: [[95.0, -750], [-363.0, -750]]
	ne08:
		Next: [[1000.0, -450]]
		Prev: [[1200.0, -450]]
	\end{lstlisting}
	
	Signalling.RNA (Código \ref{lst:EJ5_6}): información detallada de todas las señales generadas.
	
	\begin{lstlisting}[language = {}, caption = Signalling.RNA, label = {lst:EJ5_6}]
	sig01 [T01] <<:
		From: ne01 | To: bus174_left
		Type: Stop | Direction: reverse | AtTrack: right 
		Position: [-1351, 150] | Coordinate: 0.1453
	sig02 [T02] >>:
		From: ne01 | To: ne01_right
		Type: Stop | Direction: normal | AtTrack: left 
		Position: [-1351, 150] | Coordinate: 0.1453
	sig03 [T03] >>:
		From: ne04 | To: bus176_right
		Type: Stop | Direction: normal | AtTrack: left 
		Position: [1351, 150] | Coordinate: 0.8601
	sig04 [T04] <<:
		From: ne04 | To: ne04_left
		Type: Stop | Direction: reverse | AtTrack: right 
		Position: [1351, 150] | Coordinate: 0.8601
	sig05 [T05] <<:
		From: ne05 | To: bus175_left
		Type: Stop | Direction: reverse | AtTrack: right 
		Position: [-1351, 450] | Coordinate: 0.1453
	sig06 [T06] >>:
		From: ne05 | To: ne05_right
		Type: Stop | Direction: normal | AtTrack: left 
		Position: [-1351, 450] | Coordinate: 0.1453
	sig07 [T07] >>:
		From: ne08 | To: bus177_right
		Type: Stop | Direction: normal | AtTrack: left 
		Position: [1351, 450] | Coordinate: 0.8601
	sig08 [T08] <<:
		From: ne08 | To: ne08_left
		Type: Stop | Direction: reverse | AtTrack: right 
		Position: [1351, 450] | Coordinate: 0.8601
	sig21 [C21] <<:
		From: ne02 | To: ne02_left
		Type: Circulation | Direction: reverse | AtTrack: right 
		Position: [-548.9, 150] | Coordinate: 0.1428
	sig22 [B22] <<:
		From: ne03 | To: ne03_left
		Type: Manouver | Direction: reverse | AtTrack: right 
		Position: [89.0, -150] | Coordinate: 0.8014
	sig25 [C25] >>:
		From: ne02 | To: ne02_right
		Type: Circulation | Direction: normal | AtTrack: left 
		Position: [521.9, 150] | Coordinate: 0.8571
	sig26 [B26] >>:
		From: ne03 | To: ne03_right
		Type: Manouver | Direction: normal | AtTrack: left 
		Position: [-111.0, -150] | Coordinate: 0.6869
	sig29 [C29] <<:
		From: ne06 | To: ne06_left
		Type: Circulation | Direction: reverse | AtTrack: right 
		Position: [-548.9, 450] | Coordinate: 0.1428
	sig30 [B30] <<:
		From: ne07 | To: ne07_left
		Type: Manouver | Direction: reverse | AtTrack: right 
		Position: [95.0, 750] | Coordinate: 0.8048
	sig33 [C33] >>:
		From: ne06 | To: ne06_right
		Type: Circulation | Direction: normal | AtTrack: left 
		Position: [521.9, 450] | Coordinate: 0.8571
	sig34 [B34] >>:
		From: ne07 | To: ne07_right
		Type: Manouver | Direction: normal | AtTrack: left 
		Position: [-105.0, 750] | Coordinate: 0.6904
	\end{lstlisting}
	
	Routes.RNA (Código \ref{lst:EJ5_7}): tabla de enclavamientos.
	
	\begin{lstlisting}[language = {}, caption = Routes.RNA, label = {lst:EJ5_7}]
	route_1 [sig02 >> sig25]:
		Path: ['ne01', 'ne02']
		Switches: ['Sw01']
	route_2 [sig02 >> sig26]:
		Path: ['ne01', 'ne03']
		Switches: ['Sw01']
	route_3 [sig04 << sig21]:
		Path: ['ne04', 'ne02']
		Switches: ['Sw02']
	route_4 [sig04 << sig22]:
		Path: ['ne04', 'ne03']
		Switches: ['Sw02']
	route_5 [sig06 >> sig33]:
		Path: ['ne05', 'ne06']
		Switches: ['Sw03']
	route_6 [sig06 >> sig34]:
		Path: ['ne05', 'ne07']
		Switches: ['Sw03']
	route_7 [sig08 << sig29]:
		Path: ['ne08', 'ne06']
		Switches: ['Sw04']
	route_8 [sig08 << sig30]:
		Path: ['ne08', 'ne07']
		Switches: ['Sw04']
	route_9 [sig21 << sig01]:
		Path: ['ne02', 'ne01']
		Switches: ['Sw01']
	route_10 [sig22 << sig01]:
		Path: ['ne03', 'ne01']
		Switches: ['Sw01']
	route_11 [sig25 >> sig03]:
		Path: ['ne02', 'ne04']
		Switches: ['Sw02']
	route_12 [sig26 >> sig03]:
		Path: ['ne03', 'ne04']
		Switches: ['Sw02']
	route_13 [sig29 << sig05]:
		Path: ['ne06', 'ne05']
		Switches: ['Sw03']
	route_14 [sig30 << sig05]:
		Path: ['ne07', 'ne05']
		Switches: ['Sw03']
	route_15 [sig33 >> sig07]:
		Path: ['ne06', 'ne08']
		Switches: ['Sw04']
	route_16 [sig34 >> sig07]:
		Path: ['ne07', 'ne08']
		Switches: ['Sw04']
	\end{lstlisting}