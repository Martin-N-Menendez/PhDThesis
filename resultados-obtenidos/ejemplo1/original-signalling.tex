\subsection{Señalamiento original}

    El señalamiento diseñado en forma manual por el autor de esta tesis para la topología de la Figura \ref{fig:EJ1_1} se ilustra en la Figura \ref{fig:EJ1_2}. Se observa que incluye señales de parada próximas a los finales de vías absolutos (S18, S19, S20), señales de partida en las plataformas (S04, S06, S08, S09), señales de protección antes de cada paso a nivel (S04, S05), señales de maniobras antes de converger en una vía principal (S03, S08) y señales múltiples para cambios de vías divergentes (S01, S17, S10, S07), entre varias otras señales. Estas señales permiten definir hasta un máximo de 14 rutas, todas ellas detalladas en la Tabla \ref{Tab:tabla_original_1}. En una primera inspección, se puede comprobar que todos los elementos ferroviarios son alcanzados por al menos una de las rutas, en al menos una dirección. Además, todos los cambios de vías son utilizados, de forma simple o compuesta. 
    
    \begin{figure}[H]
    	\centering
    	\includegraphics[width=1\textwidth]{resultados-obtenidos/ejemplo1/images/1_original.png}
    	\centering\caption{Señalamiento original del ejemplo 1.}
    	\label{fig:EJ1_2}
    \end{figure}
    
    \begin{table}[H]
        {
        \caption{Tabla de enclavamiento original del ejemplo 1.}
        \label{Tab:tabla_original_1}
        \centering
        \resizebox{1\textwidth}{!}{
            \begin{tabular}{ c c c c c c c }
                \hline	
                    Ruta & Inicio & Final & Cambio & Plataforma & Cruce & netElement \\	
                \hline
                    R$_{01}$  & S$_{05}$ & S$_{06}$ & - & Plat$_{11}$ & Lc$_{06}$ & ne$_{14}$\\
                    R$_{02}$  & S$_{06}$ & S$_{20}$ & - & - & - & ne$_{14}$\\
                    R$_{03}$  & S$_{09}$ & S$_{18}$ & - & - & - & ne$_{13}$\\
                    R$_{04}$  & S$_{13}$ & S$_{12}$ & Sw$_{13}^{N}$ & - & - & ne$_{23}$-ne$_{12}$\\
                    R$_{05}$  & S$_{16}$ & S$_{02}$ & Sw$_{12}^{N}$ & - & - & ne$_{22}$-ne$_{08}$\\
                    R$_{06}$  & S$_{07}$ & S$_{10}$ & Sw$_{06}^{N}$ & - & - & ne$_{02}$-ne$_{12}$\\
                    R$_{07}$  & S$_{07}$ & S$_{09}$ & Sw$_{06}^{R}$ & Plat$_{13}$ & Lc$_{08}$ & ne$_{02}$-ne$_{13}$\\
                    R$_{08}$  & S$_{10}$ & S$_{14}$ & Sw$_{13}^{N}$ & - & - & ne$_{12}$-ne$_{23}$\\
                    R$_{09}$  & S$_{10}$ & S$_{02}$ & Sw$_{12}^{R}$+Sw$_{13}^{R}$ & - & - & ne$_{12}$-ne$_{24}$-ne$_{08}$\\
                    R$_{10}$  & S$_{01}$ & S$_{17}$ & Sw$_{04}^{N}$ & - & - & ne$_{01}$-ne$_{08}$\\
                    R$_{11}$  & S$_{01}$ & S$_{19}$ & Sw$_{04}^{R}$+Sw$_{07}^{N}$ & - & - & ne$_{01}$-ne$_{15}$\\
                    R$_{12}$  & S$_{01}$ & S$_{05}$ & Sw$_{04}^{R}$+Sw$_{07}^{R}$ & - & - & ne$_{01}$-ne$_{14}$\\
                    R$_{13}$  & S$_{17}$ & S$_{15}$ & Sw$_{12}^{N}$ & - & - & ne$_{08}$-ne$_{22}$\\
                    R$_{14}$  & S$_{17}$ & S$_{12}$ & Sw$_{12}^{R}$+Sw$_{13}^{R}$ & - & - & ne$_{08}$-ne$_{24}$-ne$_{12}$\\    
                \hline
            \end{tabular}
        }
     }
    \end{table}
    
    Algunas rutas abarcan mas de un \textit{netElement}, como por ejemplo la ruta R14 que comienza en la señal S17 y finaliza en la señal S12, atraviesa los \textit{netElements} ne8, ne24 y ne12, y utiliza los cambios de vías Sw12 y Sw13, ambos en posición reversa.
