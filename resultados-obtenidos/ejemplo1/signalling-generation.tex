\subsection{Generación de señalamiento paso a paso}

	Al ejecutar el RNA, primero detectará todos los \textit{netElements}, sus coordenadas iniciales y finales en la topología, y el sentido en el que fueron definidas. El resultado obtenido se muestra en el Cóodigo \ref{lst:EJ1_1}.
	
	\begin{lstlisting}[language = {}, caption = Detección de \textit{netElements} por parte del RNA , label = {lst:EJ1_1}]
		###### Starting Railway Network Analyzer #####
		Reading .railML file
		Creating railML object
		Analysing railML object
		Analysing graph
		ne1 [810, 150] [1320, 150] >>
		ne2 [2970, 0] [2460, 0] <<
		ne8 [1320, 150] [1800, 150] >>
		ne9 [1320, 150] [1530, 360] >>
		ne12 [2460, 0] [1950, 0] <<
		ne13 [2460, 0] [810, -180] <<
		ne14 [1530, 360] [2970, 360] >>
		ne15 [1530, 360] [2970, 570] >>
		ne22 [1800, 150] [2970, 150] >>
		ne23 [1950, 0] [810, 0] <<
		ne24 [1800, 150] [1950, 0] >>
		The network is connected
	\end{lstlisting}
	
	Por ejemplo, el \textit{netElement} ne1 inicia en la coordenada (810;150) y finaliza en la coordenada (1320;150). El símbolo $>>$ indica que ne1 se encuentra definido de izquierda a derecha, ya que la componente x de la coordenada final es mayor a la de la coordenada inicial, teniendo la misma componente y. Además, se puede comprobar que la lista obtenida en consistente con la Figura \ref{fig:EJ1_2}. Por ejemplo, ne1, ne8 y ne9 comparten la coordenada (1320;150), que coincide con la coordenada del cambio de vías Sw04.
	
	A continuación, el RNA detectará la infraestructura ferroviaria, las curvas peligrosas y los puntos medios de los netElements que el RNA considera demasiado largos. El resultado de este proceso se puede visualizar en el Código \ref{lst:EJ1_2} y puede leerse también en el archivo Infrastructure.RNA.
	
	\begin{lstlisting}[language = {}, caption = Detección de puntos críticos por parte del RNA , label = {lst:EJ1_2}]
		Analysing infrastructure --> Infrastructure.RNA
		Detecting Danger --> Safe_points.RNA
		ne1 has a Middle point @ [1065.0, 150]
		ne2 has a Middle point @ [2715.0, 0]
		ne8 has a Middle point @ [1560.0, 150]
		ne12 has a Middle point @ [2205.0, 0]
		ne13 has a Platform[plf75] @ [1564, 180]
		ne13 has a LevelCrossing[lcr74] @ [1362, 180]
		ne13 has a Curve(2 lines) @ [[2280, -180]]
		ne14 has a Platform[plf68] @ [2490, -360]
		ne14 has a LevelCrossing[lcr69] @ [1945, -360]
		ne15 has a Curve(2 lines) @ [[1740, 570]]
		ne22 has a RailJoint[J15] @ [2452, 150]
		ne23 has a RailJoint[J14] @ [1284, 0]
	\end{lstlisting}
	
	Una vez que el RNA detectó cada punto crítico de la red ferroviaria, procede a generar el señalamiento. El orden de generación no es importante, pero para poder describirlo de forma consistente se iniciará generando el señalamiento para proteger los finales de vías, las junturas entre rieles, las plataformas, los cruces de vía y los cambios de vías. Luego se procederá a mostrar el señalamiento pre y post simplificación. Las señales generadas para proteger los finales de vías relativos y absolutos son ilustradas en la Figura \ref{fig:EJ1_3}.
	
	\begin{figure}[H]
		\centering
		\includegraphics[width=1\textwidth]{resultados-obtenidos/ejemplo1/images/1_step1.png}
		\centering\caption{Señalamiento generado por el RNA para proteger el fin de vía.}
		\label{fig:EJ1_3}
	\end{figure}
	
	Los finales de vías absolutos son protegidos por las señales de parada T01, T03, T05 y las señales de partida son T02, T04 y T06. A su vez, los finales de vías relativos poseen las señales de parada L07, L08, L09 y L10, cercanos al límite del externo del \textit{netElement} al que pertenecen.
	
	La Figura \ref{fig:EJ1_3} ilustra la generación de señales destinadas a proteger las junturas entre los rieles. Las señales generadas son J11, J12, J13 y J14, indicadas en color rojo. De no existir junturas que proteger, el RNA salteará este paso.
	
	\begin{figure}[H]
		\centering
		\includegraphics[width=1\textwidth]{resultados-obtenidos/ejemplo1/images/1_step2.png}
		\centering\caption{Señalamiento generado por el RNA para proteger las junturas.}
		\label{fig:EJ1_4}
	\end{figure}
	
	Al generar el señalamiento para proteger la infraestructura, tal como se explicó en la Sección \ref{sec:horizontal}, el Algoritmo \ref{alg:horizontal} simplificará las señales entre dos elementos ferroviarios si no existe espacio suficiente entre ellos, tal como sucede con los elementos levelCrossing8 y Platform13. El señalamiento generado para proteger las plataformas y los cruces de vía se ilustra en rojo en la Figura \ref{fig:EJ1_5}. Las señales generadas para proteger las plataformas son las señales de partida P18, P19 y P20, mientras que las señales que protegen los cruces de vía son las señales X15, X16 y X17.
		
	\begin{figure}[H]
		\centering
		\includegraphics[width=1\textwidth]{resultados-obtenidos/ejemplo1/images/1_step3.png}
		\centering\caption{Señalamiento generado por el RNA para proteger plataformas y cruces de vía.}
		\label{fig:EJ1_5}
	\end{figure}
	
	El RNA generó las señales S22, C21, H23 y H24 para proteger el cambio de vías Sw04; las señales S27, C25, B26 y H28 para proteger el cambio de vías Sw06; las señales C29 y B30 para proteger el cambio de vías Sw07; las señales S32, C31 y H33 para proteger el cambio de vías Sw12 y las señales S35, C34 y H36 para proteger el cambio de vías Sw13. Las señales mencionadas se encuentran resaltadas en rojo en la Figura \ref{fig:EJ1_6}.
	
	\begin{figure}[H]
		\centering
		\includegraphics[width=1\textwidth]{resultados-obtenidos/ejemplo1/images/1_step4.png}
		\centering\caption{Señalamiento generado por el RNA para proteger los cambios de vías.}
		\label{fig:EJ1_6}
	\end{figure}
	
	Una vez obtenido todo el señalamiento, el RNA procede a simplificar las señales redundantes, repetidas o cuyas funciones o ubicaciones se superponen entre sí. El proceso de simplificación de señales fue explicado en la Sección \label{sec:simplificacion}. El Algoritmo \ref{alg:vertical} de herencia vertical fue aplicado en las señales B entre los cambios de vías Sw12 y Sw13, desplazando las señales hasta convertirlas en las señales H33 y H36 respectivamente. Análogamente, las señales C y S del \textit{netElement} se convirtieron en las señales H23 y H24 respectivamente.
	
	Las señales simplificadas al aplicar el Algoritmo \ref{alg:horizontal} de herencia horizontal son: X17, P18, P19, B26, B30, C31 y C34. Las señales X17 y B26 fueron eliminadas por su cercanía con la señal T02, con la cual comparten dirección y sentido. Lo mismo ocurre entre el par de señales P18/B30 y la señal T04; entre las señales P19 y T03; entre las señales C31 y J12; y entre las señales C34 y J13. En todos los casos, se aplicó el Algoritmo \ref{alg:horizontal}, diseñado para agrupar objetos cercanos como un único objeto, generando el señalamiento acorde a los elementos contenidos en cada extremo del nuevo elemento contenedor.
	
	Finalmente, las señales son simplificadas aplicando el Algoritmo \ref{alg:reduction} de eliminación por prioridad de señales. El resultado de este proceso es detallado en el Código \ref{lst:EJ1_3}.
	
	\begin{lstlisting}[language = {}, caption = Reducción de señalamiento por prioridad de señales, label = {lst:EJ1_3}]
		Reducing redundant signals
		removing sig17 for sig02
		removing sig26 for sig02
		removing sig19 for sig03
		removing sig30 for sig04
		removing sig31 for sig12
		removing sig34 for sig13
		removing sig18 for sig30
	\end{lstlisting}
	
	El resultado de la simplificación del señalamiento se ilustra en la Figura \ref{fig:EJ1_7}.
	
	\begin{figure}[H]
		\centering
		\includegraphics[width=1\textwidth]{resultados-obtenidos/ejemplo1/images/1_RNA.png}
		\centering\caption{Señalamiento generado y simplificado por el RNA.}
		\label{fig:EJ1_7}
	\end{figure}
	
	Al finalizar la generación del señalamiento, el RNA debe detectar todas las posibles rutas admitidas por la red para crear la tabla de enclavamientos. El RNA exporta los resultados del análisis en los siguientes cuatro documentos:
	
	Infrastructure.RNA (Código \ref{lst:EJ1_4}): resumen de cada elemento ferroviario en cada \textit{netElement}.
	
	\begin{lstlisting}[language = {}, caption = Infrastructure.RNA, label = {lst:EJ1_4}]
	Nodes: 11|Switches: 5|Signals: 0|Detectors: 2|Ends: 7|Barriers: 2
	Node ne1:
	Track = track1
	Neighbours = 2 -> ['ne8', 'ne9']
	Switches -> Sw04
	ContinueCourse -> right -> ne8
	BranchCourse -> left -> ne9
	Node ne2:
	Track = track7
	Neighbours = 2 -> ['ne12', 'ne13']
	Switches -> Sw06
	ContinueCourse -> right -> ne12
	BranchCourse -> left -> ne13
	Node ne8:
	Track = track2
	Neighbours = 4 -> ['ne1', 'ne9', 'ne22', 'ne24']
	Switches -> Sw12
	ContinueCourse -> left -> ne22
	BranchCourse -> right -> ne24
	Node ne9:
	Track = track3
	Neighbours = 4 -> ['ne1', 'ne8', 'ne14', 'ne15']
	Switches -> Sw07
	ContinueCourse -> left -> ne15
	BranchCourse -> right -> ne14
	Node ne12:
	Track = track8
	Neighbours = 4 -> ['ne2', 'ne13', 'ne23', 'ne24']
	Switches -> Sw13
	ContinueCourse -> left -> ne23
	BranchCourse -> right -> ne24
	Node ne13:
	Track = track9
	Type = BufferStop -> ['bus56']
	Neighbours = 2 -> ['ne2', 'ne12']
	Level crossing -> lcr74
	Protection -> true | Barriers -> none | Lights -> none Acoustic -> none
	Position -> [1317, 180] | Coordinate: 0.7059
	Node ne14:
	Track = track4
	Type = BufferStop -> ['bus10']
	Neighbours = 2 -> ['ne9', 'ne15']
	Level crossing -> lcr69
	Protection -> true | Barriers -> none | Lights -> none Acoustic -> none
	Position -> [1990, -360] | Coordinate: 0.3197
	Node ne15:
	Track = track10
	Type = BufferStop -> ['bus59']
	Neighbours = 2 -> ['ne9', 'ne14']
	Node ne22:
	Track = track6
	TrainDetectionElements -> tde78
	Type -> insulatedRailJoint
	Neighbours = 2 -> ['ne8', 'ne24']
	Node ne23:
	Track = track5
	TrainDetectionElements -> tde77
	Type -> insulatedRailJoint
	Neighbours = 2 -> ['ne12', 'ne24']
	Node ne24:
	Track = track11
	Neighbours = 4 -> ['ne8', 'ne12', 'ne22', 'ne23']
	\end{lstlisting}
	
	SafePoints.RNA (Código \ref{lst:EJ1_5}): coordenadas absolutas de cada punto donde puede colocarse una señal, en cada \textit{netElement}.
	
	\begin{lstlisting}[language = {}, caption = SafePoints.RNA, label = {lst:EJ1_5}]
	ne1:
	Next: [[1065.0, 150]]
	Prev: [[1065.0, 150]]
	ne2:
	Next: [[2715.0, 0]]
	Prev: [[2715.0, 0]]
	ne8:
	Next: [[1560.0, 150]]
	Prev: [[1560.0, 150]]
	ne12:
	Next: [[2205.0, 0]]
	Prev: [[2205.0, 0]]
	ne13:
	Next: [[1162, 180], [2180.0, -180]]
	Prev: [[1764, 180]]
	ne14:
	Next: [[2290, -360], [1745, -360]]
	Prev: [[2690, -360], [2145, -360]]
	ne15:
	Prev: [[1840.0, 570]]
	ne22:
	Next: [[2352.0, 150]]
	Prev: [[2552.0, 150]]
	ne23:
	Next: [[1184.0, 0]]
	Prev: [[1384.0, 0]]
	\end{lstlisting}
	
	Signalling.RNA (Código \ref{lst:EJ1_6}): información detallada de todas las señales generadas.
	
	\begin{lstlisting}[language = {}, caption = Signalling.RNA, label = {lst:EJ1_6}]
	sig01 [T01] <<:
	From: ne13 | To: bus56_left
	Type: Stop | Direction: normal | AtTrack: left 
	Position: [910, 180] | Coordinate: 0.2055
	sig02 [T02] >>:
	From: ne13 | To: ne13_right
	Type: Stop | Direction: reverse | AtTrack: right 
	Position: [910, 180] | Coordinate: 0.2055
	sig03 [T03] >>:
	From: ne14 | To: bus10_right
	Type: Stop | Direction: normal | AtTrack: left 
	Position: [2870, -360] | Coordinate: 0.9305
	sig04 [T04] <<:
	From: ne14 | To: ne14_left
	Type: Stop | Direction: reverse | AtTrack: right 
	Position: [2870, -360] | Coordinate: 0.9305
	sig05 [T05] >>:
	From: ne15 | To: bus59_right
	Type: Stop | Direction: normal | AtTrack: left 
	Position: [2870, -570] | Coordinate: 0.9345
	sig06 [T06] <<:
	From: ne15 | To: ne15_left
	Type: Stop | Direction: reverse | AtTrack: right 
	Position: [2870, -570] | Coordinate: 0.9345
	sig07 [L07] <<:
	From: ne1 | To: oe40_left
	Type: Circulation | Direction: reverse | AtTrack: right 
	Position: [910, -150] | Coordinate: 0.1960
	sig08 [L08] >>:
	From: ne2 | To: oe42_right
	Type: Circulation | Direction: reverse | AtTrack: right 
	Position: [2870, 0] | Coordinate: 0.8039
	sig09 [L09] >>:
	From: ne22 | To: oe41_right
	Type: Circulation | Direction: normal | AtTrack: left 
	Position: [2870, -150] | Coordinate: 0.9145
	sig10 [L10] <<:
	From: ne23 | To: oe39_left
	Type: Circulation | Direction: normal | AtTrack: left 
	Position: [910, 0] | Coordinate: 0.0877
	sig11 [J11] >>:
	From: ne22 | To: ne22_right
	Type: Circulation | Direction: normal | AtTrack: left 
	Position: [2352.0, -150] | Coordinate: 0.4717
	sig12 [J12] <<:
	From: ne22 | To: ne22_left
	Type: Circulation | Direction: reverse | AtTrack: right 
	Position: [2552.0, -150] | Coordinate: 0.6427
	sig13 [J13] >>:
	From: ne23 | To: ne23_right
	Type: Circulation | Direction: reverse | AtTrack: right 
	Position: [1184.0, 0] | Coordinate: 0.3280
	sig14 [J14] <<:
	From: ne23 | To: ne23_left
	Type: Circulation | Direction: normal | AtTrack: left 
	Position: [1384.0, 0] | Coordinate: 0.5035
	sig15 [X15] >>:
	From: ne14 | To: ne14_right
	Type: Circulation | Direction: normal | AtTrack: left 
	Position: [1745, 360] | Coordinate: 0.5218
	sig16 [X16] <<:
	From: ne14 | To: ne14_left
	Type: Circulation | Direction: reverse | AtTrack: right 
	Position: [2145, 360] | Coordinate: 0.6575
	sig20 [P20] >>:
	From: ne13 | To: ne13_right
	Type: Circulation | Direction: reverse | AtTrack: right 
	Position: [1844, -180] | Coordinate: 0.7824
	sig21 [C21] <<:
	From: ne8 | To: ne8_left
	Type: Circulation | Direction: reverse | AtTrack: right 
	Position: [1560.0, -150] | Coordinate: 0.5
	sig22 [S22] >>:
	From: ne1 | To: ne1_right
	Type: Circulation | Direction: normal | AtTrack: left 
	Position: [1065.0, -150] | Coordinate: 0.5
	sig25 [C25] >>:
	From: ne12 | To: ne12_right
	Type: Circulation | Direction: reverse | AtTrack: right 
	Position: [2205.0, 0] | Coordinate: 0.5
	sig27 [S27] <<:
	From: ne2 | To: ne2_left
	Type: Circulation | Direction: normal | AtTrack: left 
	Position: [2715.0, 0] | Coordinate: 0.5
	sig29 [C29] <<:
	From: ne15 | To: ne15_left
	Type: Manouver | Direction: reverse | AtTrack: right 
	Position: [1840.0, -570] | Coordinate: 0.2599
	sig32 [S32] >>:
	From: ne8 | To: ne8_right
	Type: Circulation | Direction: normal | AtTrack: left 
	Position: [1560.0, -150] | Coordinate: 0.5
	sig35 [S35] <<:
	From: ne12 | To: ne12_left
	Type: Circulation | Direction: normal | AtTrack: left 
	Position: [2205.0, 0] | Coordinate: 0.5
	\end{lstlisting}
	
	Routes.RNA (Código \ref{lst:EJ1_7}): tabla de enclavamientos.
	
	\begin{lstlisting}[language = {}, caption = Routes.RNA, label = {lst:EJ1_7}]
	route_1 [sig02 >> sig20]:
	Path: ['ne13']
	Platforms: ['plf75']
	route_2 [sig04 << sig16]:
	Path: ['ne14']
	Platforms: ['plf68']
	route_3 [sig06 << sig29]:
	Path: ['ne15']
	route_4 [sig11 >> sig09]:
	Path: ['ne22']
	route_5 [sig12 << sig21]:
	Path: ['ne22', 'ne8']
	Switches: ['Sw12']
	route_6 [sig13 >> sig25]:
	Path: ['ne23', 'ne12']
	Switches: ['Sw13']
	route_7 [sig14 << sig10]:
	Path: ['ne23']
	route_8 [sig15 >> sig03]:
	Path: ['ne14']
	Platforms: ['plf68']
	route_9 [sig16 << sig07]:
	Path: ['ne14', 'ne9', 'ne1']
	Switches: ['Sw04', 'Sw07']
	Platforms: ['plf68']
	route_10 [sig20 >> sig08]:
	Path: ['ne13', 'ne2']
	Switches: ['Sw06']
	Platforms: ['plf75']
	route_11 [sig21 << sig07]:
	Path: ['ne8', 'ne1']
	Switches: ['Sw04', 'Sw12']
	route_12 [sig22 >> sig32]:
	Path: ['ne1', 'ne8']
	Switches: ['Sw04', 'Sw12']
	route_13 [sig22 >> sig15]:
	Path: ['ne1', 'ne9', 'ne14']
	Switches: ['Sw04', 'Sw07']
	Platforms: ['plf68']
	route_14 [sig22 >> sig05]:
	Path: ['ne1', 'ne9', 'ne15']
	Switches: ['Sw04', 'Sw07']
	route_15 [sig25 >> sig08]:
	Path: ['ne12', 'ne2']
	Switches: ['Sw06', 'Sw13']
	route_16 [sig27 << sig35]:
	Path: ['ne2', 'ne12']
	Switches: ['Sw06', 'Sw13']
	route_17 [sig27 << sig01]:
	Path: ['ne2', 'ne13']
	Switches: ['Sw06']
	Platforms: ['plf75']
	route_18 [sig29 << sig07]:
	Path: ['ne15', 'ne9', 'ne1']
	Switches: ['Sw04', 'Sw07']
	route_19 [sig32 >> sig11]:
	Path: ['ne8', 'ne22']
	Switches: ['Sw12']
	route_20 [sig32 >> sig25]:
	Path: ['ne8', 'ne24', 'ne12']
	Switches: ['Sw12', 'Sw13']
	route_21 [sig35 << sig14]:
	Path: ['ne12', 'ne23']
	Switches: ['Sw13']
	route_22 [sig35 << sig21]:
	Path: ['ne12', 'ne24', 'ne8']
	Switches: ['Sw12', 'Sw13']
	\end{lstlisting}
	
	
	
	
	
	
	
	