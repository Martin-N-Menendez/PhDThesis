\subsection{Validación del sistema de enclavamientos}
	
    La validación de las rutas de la tabla de enclavamientos obtenida (Tabla \ref{lst:EJ1_8}) es realizada por el RNA aplicando el Algoritmo \ref{alg:interlocking_tables}, explicado en la Sección \ref{sec:validar_tabla}. Las 14 rutas del señalamiento original (Tabla \ref{Tab:tabla_original_1}) tienen 14 rutas equivalentes en el señalamiento generado por el RNA (Tabla \ref{Tab:tabla_generated_1}), tal como se puede visualizar en la Tabla \ref{Tab:tabla_validation_1}, generada automáticamente por el RNA.

    \begin{table}[H]
        {
        \caption{Equivalencias entre las rutas originales y las generadas por el RNA.}
        \label{Tab:tabla_validation_1}
        \centering
        %\small
            %\centering
            \begin{center}
            \resizebox{0.5\textwidth}{!}{
            \begin{tabular}{ c c c c }
                \hline	
                    Original & Señales & RNA & Señales \\	
                \hline
                    R$_{01}$ & S$_{05}$-S$_{06}$ & R$_{07}$ & X$_{15}$-T$_{03}$ \\
                    R$_{02}$ & S$_{06}$-S$_{20}$ & R$_{07}$ & X$_{15}$-T$_{03}$ \\
                    R$_{03}$ & S$_{09}$-S$_{18}$ & R$_{16}$ & S$_{27}$-T$_{01}$ \\
                    R$_{04}$ & S$_{13}$-S$_{12}$ & R$_{06}$ & J$_{13}$-C$_{25}$ \\
                    R$_{05}$ & S$_{16}$-S$_{02}$ & R$_{05}$ & J$_{12}$-C$_{21}$ \\
                    R$_{06}$ & S$_{07}$-S$_{10}$ & R$_{15}$ & S$_{27}$-S$_{35}$ \\
                    R$_{07}$ & S$_{07}$-S$_{09}$ & R$_{16}$ & S$_{27}$-T$_{01}$ \\
                    R$_{08}$ & S$_{10}$-S$_{14}$ & R$_{20}$ & S$_{35}$-J$_{14}$ \\
                    R$_{09}$ & S$_{10}$-S$_{02}$ & R$_{21}$ & S$_{35}$-C$_{21}$ \\
                    R$_{10}$ & S$_{01}$-S$_{17}$ & R$_{11}$ & S$_{22}$-S$_{32}$ \\
                    R$_{11}$ & S$_{01}$-S$_{19}$ & R$_{13}$ & S$_{22}$-X$_{15}$ \\
                    R$_{12}$ & S$_{01}$-S$_{05}$ & R$_{12}$ & S$_{22}$-T$_{05}$ \\
                    R$_{13}$ & S$_{17}$-S$_{15}$ & R$_{18}$ & S$_{32}$-J$_{11}$ \\
                    R$_{14}$ & S$_{17}$-S$_{12}$ & R$_{19}$ & S$_{32}$-C$_{25}$ \\   
                \hline
            \end{tabular}
            }
            \end{center}
        }    
    \end{table}
    
    Las rutas R1, R2, R3, R4, R7, R8, R9, R10, R13, R14 y R16 (Tabla \ref{Tab:tabla_generated_1}) generadas por el RNA que no tienen equivalencias en el señalamiento original (Tabla \ref{Tab:tabla_original_1}) se deben a que el RNA creó señales extras. Las señales T01, T02, T03, T04, T05 y T06 fueron creadas por el RNA para proteger los finales de vía absolutos, mientras que las señales L07, L08, L09 y L10 fueron creadas para proteger los finales de vías relativos. Estas nuevas señales constituyen nuevas rutas que permiten a las formaciones detenerse previo al final (relativo o absoluto) de la red ferroviaria, lo cual incrementa la seguridad y añade nuevas rutas en sentido contrario, mejorando la logística. Estos elementos ferroviarios no se encontraban protegidos en el señalamiento diseñado por el autor de esta tesis. No obstante, esto no se debe a un error en el diseño del señalamiento del autor de esta tesis, sino a que el diseño de señalamientos en la vida real se encuentra restringido por las necesidades particulares de cada locación, operador o leyes locales. Algunos operadores priorizarán tener rutas largas, dejando largas distancias sin señalamiento electrónico, algunas normativas pueden indicar que la protección de ciertos elementos puede ser optativa, en el caso de los finales de vía pueden utilizarse señales lumínicas intermitentes que no son controladas por el señalamiento, etc.
    
    En otros ejemplos el señalamiento original puede estar '\textit{incompleto}', es decir, solo se consideraron las rutas en un sentido determinado, en base al uso que se le quería dar a ese señalamiento. El RNA, en cambio, siempre generará el señalamiento completo, que abarque la totalidad de la red ferroviaria ingresada, a menos que se seleccione la opción de señalamiento parcial, que respetará el sentido único de circulación que se defina en cada vía.    
    
    Para finalizar, el RNA comprueba los principios de señalamiento ferroviario explicados en la Sección \ref{sec:validar_principios}, aplicando los algoritmos correspondientes, de los cuáles se obtuvieron los siguientes resultados:
    
	\begin{itemize}
		\item Principio de autoridad (Algoritmo \ref{alg:ppio_autoridad}): cobertura del 100\% de los \textit{netElements}.
		\item Principio de claridad (Algoritmo \ref{alg:ppio_claridad}): rutas 100\% independientes.
		\item Principio de anticipación (Algoritmo \ref{alg:ppio_anticipacion}): cobertura del 100\% de los puntos críticos.
		\item Principio de granularidad (Algoritmo \ref{alg:ppio_granularidad}): 100\% de rutas divididas a su mínima expresión.
		\item Principio de terminalidad (Algoritmo \ref{alg:ppio_terminalidad}): 100\% de finales de vías protegidos.
		\item Principio de infraestructura (Algoritmo \ref{alg:ppio_infraestructura}): 100\% de infraestructura protegida.
		\item Principio de no bloqueo (Algoritmo \ref{alg:ppio_nobloqueo}): 100\% de cambios de vías protegidos.
	\end{itemize}	