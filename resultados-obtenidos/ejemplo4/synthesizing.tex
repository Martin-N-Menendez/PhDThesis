\section{Sistema generado por el ACG}
	
	En base a la red de grafos, ilustrada en la Figura \ref{fig:EJ4_8}, el ACG determinó la siguiente cantidad de elementos, tal puede visualizarse en el Código \ref{lst:EJ4_8}.

	\begin{lstlisting}[language = {}, caption = Cantidad de elementos a implementar por el ACG, label = {lst:EJ4_8}]
	n_netElements:47
	n_switch:21
	n_doubleSwitch:1
	n_borders:5
	n_buffers:14
	n_levelCrossings:2
	n_platforms:0
	n_scissorCrossings:0
	n_signals:77
	N : 238
	\end{lstlisting}
	
	Se repetirán los pasos detallados en el ejemplo 1, Sección \ref{sec:EJEMPLO1_ACG}, por lo que solamente se destacarán aspectos particulares de este ejemplo. El ACG genera 256 archivos en formato VHDL. El archivo \textit{Arty\_Z7-10.XDC} es el mismo para todos los ejemplos, al ser invariante respecto a la cantidad de pines a utilizar. Se deberá modificar el script mostrado en el Código \ref{lst:EJ1_script} y cambiar el parámetro \textit{chosen} a 4 para automatizar la importación de los archivos del ejemplo 4 y desvincular cualquier otro conjunto de archivos de ejemplos anteriores.
	
	Una vez ejecutado el script, Vivado ordenará los archivos de forma jerárquica, donde el módulo \textit{global} incluye todos los módulos que fueron detallados en la Sección \ref{sec:interlockingArch}. Cada una de las instancias del módulo \textit{network} contienen sus propias 238 instancias de los mismos módulos de cada elemento ferroviario ya que N, cantidad de elementos ferroviarios, es 238 en el Código \ref{lst:EJ4_8}. El ejemplo 4 utiliza mas de 82000 sub módulos conectados automáticamente mediante mas de 162000 señales, lo cual se aleja bastante de un desarrollo que pueda realizarse manualmente de forma trivial.
	
	Cuando Vivado genera el diagrama de bloques ya tenemos la certeza de que el código VHDL ha pasado la prueba de sintaxis del entorno de desarrollo. A continuación, se deberá sintetizar e implementar el sistema para generar el bitstream que será utilizado para programar la FPGA. Los procesos de síntesis e implementación fueron detallados en el ejemplo 1, Sección \ref{sec:EJEMPLO1_ACG}.
	
	Los resultados de ambos procesos son detallados en la Tabla \ref{Tab:tabla_ACG_4}. Los porcentajes de uso son calculados por Vivado automáticamente, teniendo en cuenta que la plataforma Arty Z7 20 posee 53200 Look-Up-Tables (LUTs), 106400 Flip-Flops (FFs), 125 Pines de entrada y salida (IOs) y 32 Buffers (BUFGs), tal cómo se explicó en la Sección \ref{sec:AGG}. En este ejemplo, la cantidad de recursos utilizados es mediana y el tiempo de síntesis e implementación es de 2 minutos con 2 segundos y 3 minutos con 13 segundos, respectivamente.
	
	\begin{table}[H]
		{
			\caption{Síntesis e implementación del ejemplo 4 generado por el ACG.}
			\label{Tab:tabla_ACG_4}
			\centering
			%\small
			%\centering
			\begin{center}
				\resizebox{0.65\textwidth}{!}{
					\begin{tabular}{ c c c c }
						\hline	
						Recursos & Síntesis & Implementación & Uso \\	
						\hline
						LUT & 13801 & 13666 & 25.94-25.69\%\\
						FF & 15069 & 15072 & 14.16-14.17\%\\
						IO & 15 & 15 & 12.00\%\\
						BUFG & 3 & 3 & 9.38\%\\
						\hline
					\end{tabular}
				}
			\end{center}
		}    
	\end{table}