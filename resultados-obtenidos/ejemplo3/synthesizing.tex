\subsection{Sistema generado por el ACG}

En base a la red de grafos, ilustrada en la Figura \ref{fig:EJ3_8}, el ACG determinó la siguiente cantidad de elementos, tal puede visualizarse en el Código \ref{lst:EJ3_8}.

\begin{lstlisting}[language = {}, caption = Cantidad de elementos a implementar por el ACG, label = {lst:EJ3_8}]
n_netElements:53
n_switch:15
n_doubleSwitch:2
n_borders:18
n_buffers:12
n_levelCrossings:1
n_platforms:13
n_scissorCrossings:1
n_signals:82
N : 329
M : 276
\end{lstlisting}

El código VHDL generado por el ACG es importado en un proyecto de Vivado, donde es sintetizado e implementado para generar el bitstream que será utilizado para programar la FPGA. La cantidad de elementos de la FPGA utilizados por el sistema post-síntesis y post-implementación, así como el porcentaje de uso de la plataforma, son detallados en la Tabla \ref{Tab:tabla_ACG_3}.

\begin{table}[H]
	{
		\caption{Síntesis e implementación del ejemplo 3 generado por el ACG.}
		\label{Tab:tabla_ACG_3}
		\centering
		%\small
		%\centering
		\begin{center}
			\resizebox{0.7\textwidth}{!}{
				\begin{tabular}{ c c c c }
					\hline	
					Recursos & Síntesis & Implementación & Uso \\	
					\hline
					LUT & 13802 & 137873 & 25.94-25.91\%\\
					FF & 17321 & 17321 & 16.28\%\\
					IO & 16 & 16 & 12.80\%\\
					BUFG & 1 & 1 & 3.13\%\\
					\hline
				\end{tabular}
			}
		\end{center}
	}    
\end{table}

En este ejemplo, la cantidad de recursos utilizados es baja y el tiempo de sintetización e implementación es de 1 minuto con 43 segundos y 1 minuto con 20 segundos respectivamente.