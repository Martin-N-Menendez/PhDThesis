\subsection{Generación de señalamiento paso a paso}
	
	Al ejecutar el RNA, primero detectará todos los \textit{netElements}, sus coordenadas iniciales y finales en la topología, y el sentido en el que fueron definidas. El resultado obtenido se muestra en el Cóodigo \ref{lst:EJ3_1}.
	
	\begin{lstlisting}[language = {}, tabsize=4, basicstyle=\footnotesize\ttfamily, showspaces=false, showstringspaces=false, caption = Detección de \textit{netElements} por parte del RNA , label = {lst:EJ3_1}]
	###### Starting Railway Network Analyzer #####
	Reading .railML file
	Creating railML object
	Analysing railML object
	Analysing graph
	ne1 [-2010, 300] [-300, 300] >>   
	ne4 [7083, 150] [6686, 150] <<    
	ne7 [-2010, 150] [-300, 300] >>   
	ne9 [-300, 300] [188, 300] >>     
	ne11 [2580, 300] [2730, 150] >>   
	ne14 [2730, 150] [2190, 150] <<   
	ne17 [4380, 300] [2580, 300] <<   
	ne23 [3810, -900] [2490, -1050] <<
	ne24 [3810, -900] [2490, -1050] <<
	ne26 [2490, -1050] [2138, -1050] <<
	ne29 [-996, -1050] [239, -1050] >>
	ne30 [-996, -1050] [-1373, -1050] <<
	ne32 [239, -1050] [-150, -1050] <<
	ne41 [-966, -1200] [-300, -1200] >>
	ne43 [1560, -1050] [1710, -1200] >>
	ne44 [1560, -1050] [1165, -1050] <<
	ne47 [1710, -1200] [1290, -1200] <<
	ne48 [1710, -1200] [1920, -1410] >>
	ne59 [2009, 300] [2580, 300] >>
	ne64 [4108, -782] [3810, -900] <<
	ne65 [2138, -1050] [1866, -1050] <<
	ne67 [2984, -480] [3882, -480] >>
	ne70 [2984, -480] [2910, 0] <<
	ne78 [4380, 300] [4825, 300] >>
	ne79 [4230, 150] [4380, 300] >>
	ne82 [-150, -1050] [-300, -1200] <<
	ne83 [-150, -1050] [-996, -1050] <<
	ne84 [-300, -1200] [-966, -1350] <<
	ne86 [-2013, -750] [-1673, -750] >>
	ne87 [-1673, -750] [-1523, -900] >>
	ne88 [-1673, -750] [-1323, -750] >>
	ne89 [-1373, -1050] [-2013, -1050] <<
	ne90 [-2013, -900] [-1523, -900] >>
	ne91 [-1523, -900] [-1373, -1050] >>
	ne93 [4825, 300] [5149, 300] >>
	ne94 [4826, 150] [4230, 150] <<
	ne95 [188, 300] [2009, 300] >>
	ne96 [5149, 300] [5842, 300] >>
	ne97 [5149, 150] [4826, 150] <<
	ne98 [5842, 300] [6686, 300] >>
	ne99 [5841, 150] [5149, 150] <<
	ne100 [880, -1050] [578, -1050] <<
	ne101 [1165, -1050] [880, -1050] <<
	ne102 [1866, -1050] [1560, -1050] <<
	ne103 [3882, -480] [4108, -782] >>
	ne105 [6686, 300] [7083, 300] >>
	ne106 [6686, 150] [5841, 150] <<
	ne85 [-300, -1200] [578, -1050] >>
	ne77 [4230, 150] [3045, 75] <<
	ne104 [2910, 0] [3045, 75] >>
	ne52 [2730, 150] [3045, 75] >>
	ne21 [4230, 150] [3045, 75] <<
	ne110 [578, -1050] [239, -1050] <<
	The network is connected
	\end{lstlisting}
	
	Por ejemplo, el \textit{netElement} ne110 inicia en la coordenada (578;-1050) y finaliza en la coordenada (239;-1050). El símbolo $<<$ indica que ne110 se encuentra definido de derecha a izquierda, ya que la componente x de la coordenada final es menor a la de la coordenada inicial, teniendo la misma componente y. Además, se puede comprobar que la lista obtenida en consistente con la Figura \ref{fig:EJ3_2}. Por ejemplo, ne30, ne29 y ne83 comparten la coordenada (-996;-1050), que coincide con la coordenada del cambio de vías Sw08.
	
	A continuación, el RNA detectará la infraestructura ferroviaria, las curvas peligrosas y los puntos medios de los netElements que el RNA considera demasiado largos. El resultado de este proceso se puede visualizar en el Código \ref{lst:EJ3_2} y puede leerse también en el archivo Infrastructure.RNA.
	
	\begin{lstlisting}[language = {}, tabsize=4, basicstyle=\footnotesize\ttfamily, showspaces=false, showstringspaces=false, caption = Detección de puntos críticos por parte del RNA , label = {lst:EJ3_2}]
	Analysing infrastructure --> Infrastructure.RNA
	Detecting Danger --> Safe_points.RNA
	ne1 has a Platform[plf177] @ [-1539, -300]
	ne4 has a Middle point @ [6884.5, 150]
	ne7 has a Platform[plf178] @ [-1538, -150]
	ne7 has a Curve(2 lines) @ [[-450, 150]]
	ne9 has a Middle point @ [-56.0, 300]
	ne14 has a Middle point @ [2460.0, 150]
	ne17 has a Platform[plf185] @ [3623, -300]
	ne23 has a Platform[plf181] @ [3151, 1050]
	ne23 has a Curve(2 lines) @ [[3660, -1050]]
	ne24 has a Platform[plf182] @ [3151, 900]
	ne24 has a Curve(2 lines) @ [[2640, -900]]
	ne26 has a Middle point @ [2314.0, -1050]
	ne29 has a Curve(3 lines) @ [[-846, -900], [89, -900]]
	ne30 has a Middle point @ [-1184.5, -1050]
	ne32 has a Middle point @ [44.5, -1050]
	ne41 has a Platform[plf180] @ [-653, 1200]
	ne44 has a Middle point @ [1362.5, -1050]
	ne47 has a Middle point @ [1500.0, -1200]
	ne59 has a Middle point @ [2294.5, 300]
	ne64 has a Curve(2 lines) @ [[3990, -900]]
	ne65 has a Middle point @ [2002.0, -1050]
	ne67 has a Platform[plf183] @ [3430, 480]
	ne70 has a Curve(5 lines) @ [[2670, -150], [2820, -480], [2820, 0], [2910, 0]]
	ne78 has a Middle point @ [4602.5, 300]
	ne83 has a Platform[plf179] @ [-653, 1050]
	ne84 has a Curve(2 lines) @ [[-450, -1350]]
	ne86 has a Middle point @ [-1843.0, -750]
	ne88 has a Middle point @ [-1498.0, -750]
	ne89 has a Middle point @ [-1799.7, -1050]
	ne89 has a Middle point @ [-1586.3, -1050]
	ne90 has a Middle point @ [-1768.0, -900]
	ne93 has a Middle point @ [4987.0, 300]
	ne94 has a Middle point @ [4528.0, 150]
	ne95 has a LevelCrossing[lcr176] @ [1100, -300]
	ne96 has a Platform[plf187] @ [5459, -300]
	ne97 has a Middle point @ [4987.5, 150]
	ne98 has a Middle point @ [6053.0, 300]
	ne98 has a Middle point @ [6264.0, 300]
	ne98 has a Middle point @ [6475.0, 300]
	ne99 has a Platform[plf186] @ [5459, -150]
	ne100 has a Middle point @ [729.0, -1050]
	ne101 has a Middle point @ [1022.5, -1050]
	ne102 has a Middle point @ [1713.0, -1050]
	ne103 has a Curve(4 lines) @ [[3990, -480], [4108, -782], [4140, -630]]
	ne105 has a Middle point @ [6884.5, 300]
	ne106 has a Middle point @ [6052.2, 150]
	ne106 has a Middle point @ [6263.5, 150]
	ne106 has a Middle point @ [6474.8, 150]
	ne85 has a Curve(2 lines) @ [[428, -1200]]
	ne77 has a Platform[plf322] @ [3619, 0]
	ne77 has a Curve(3 lines) @ [[3120, 0], [4080, 0]]
	ne104 has a Curve(2 lines) @ [[2970, 0]]
	ne52 has a Curve(2 lines) @ [[2970, 150]]
	ne21 has a Platform[plf321] @ [3621, -150]
	ne21 has a Curve(2 lines) @ [[3120, 150]]
	ne110 has a Middle point @ [408.5, -1050]
	\end{lstlisting}
	
	Una vez que el RNA detectó cada punto crítico de la red ferroviaria, procede a generar el señalamiento. El orden de generación no es importante, pero para poder describirlo de forma consistente se iniciará generando el señalamiento para proteger los finales de vías, las junturas entre rieles, las plataformas, los cruces de vía y los cambios de vías. Luego se procederá a mostrar el señalamiento pre y post simplificación. Las señales generadas para proteger los finales de vías relativos y absolutos son ilustradas en la Figura \ref{fig:EJ3_3}.
	
	\begin{figure}[H]
		\centering
		\includegraphics[width=1\textwidth]{resultados-obtenidos/ejemplo3/images/3_step1.png}
		\centering\caption{Señalamiento generado por el RNA para proteger el fin de vía.}
		\label{fig:EJ3_3}
	\end{figure}
	
	Los finales de vías absolutos son protegidos por las señales de parada T01, T03, T05, T07, T09, T11, T13, T15, T17, T19, T21, T23 y las señales de partida son T02, T04, T06, T08, T10, T12, T14, T16, T18, T20, T22 y T24. A su vez, al no existir finales de vías relativos, el RNA no asignó señales para su protección.
	
	La Figura \ref{fig:EJ3_4} ilustra la generación de señales destinadas a proteger las junturas entre los rieles. Las señales generadas son todas las señales comprendidas entre J43 y J49, indicadas en color rojo. De no existir junturas que proteger, el RNA salteará este paso.
	
	\begin{figure}[H]
		\centering
		\includegraphics[width=1\textwidth]{resultados-obtenidos/ejemplo3/images/3_step2.png}
		\centering\caption{Señalamiento generado por el RNA para proteger las junturas.}
		\label{fig:EJ3_4}
	\end{figure}
	
	Al generar el señalamiento para proteger la infraestructura, tal como se explicó en la Sección \ref{sec:horizontal}, el Algoritmo \ref{alg:horizontal} simplificará las señales entre dos elementos ferroviarios si no existe espacio suficiente entre ellos. El señalamiento generado para proteger las plataformas y los cruces de vía se ilustra en rojo en la Figura \ref{fig:EJ3_5}. Las señales generadas para proteger las plataformas son las señales de partida P52 a P77, mientras que las señales que protegen los cruces de vía son las señales X50 y X51, ya que los cruces de vía Ucr01 y Ocr01 son cruces bajo nivel y sobre nivel respectivamente, por lo que no interrumpen la circulación de formaciones ferroviaria.
	
	\begin{figure}[H]
		\centering
		\includegraphics[width=1\textwidth]{resultados-obtenidos/ejemplo3/images/3_step3.png}
		\centering\caption{Señalamiento generado por el RNA para proteger plataformas y cruces de vía.}
		\label{fig:EJ3_5}
	\end{figure}
	
	Al tener dos cambios de vías dobles, un cambio de vías en tijeras y quince cambios de vías simples, resulta poco práctico enumerar cada una de las setenta señales generadas para proteger estos elementos ferroviarios. Todas las señales se encuentran resaltadas en rojo en la Figura \ref{fig:EJ3_6}.
	
	\begin{figure}[H]
		\centering
		\includegraphics[width=1\textwidth]{resultados-obtenidos/ejemplo3/images/3_step4.png}
		\centering\caption{Señalamiento generado por el RNA para proteger los cambios de vías.}
		\label{fig:EJ3_6}
	\end{figure}
	
	Una vez obtenido todo el señalamiento, el RNA procede a simplificar las señales redundantes, repetidas o cuyas funciones o ubicaciones se superponen entre sí. El proceso de simplificación de señales fue explicado en la Sección \ref{sec:simplificacion}. El Algoritmo \ref{alg:vertical} de herencia vertical fue aplicado en las señales B entre los cambios de vías que compartan al menos una rama secundaria, desplazando las señales hasta convertirlas en las señales H en los nodos divergentes de cada cambio de vías. 
	
	Las señales simplificadas al aplicar el Algoritmo \ref{alg:horizontal} de herencia horizontal son demasiadas como para ser listadas manualmente. En todos los casos, se aplicó el Algoritmo \ref{alg:horizontal}, diseñado para agrupar objetos cercanos como un único objeto, generando el señalamiento acorde a los elementos contenidos en cada extremo del nuevo elemento contenedor.
	
	Finalmente, las señales son simplificadas aplicando el Algoritmo \ref{alg:reduction} de eliminación por prioridad de señales. El resultado de este proceso es detallado en el Código \ref{lst:EJ3_3}.
	
	\begin{lstlisting}[language = {}, tabsize=4, basicstyle=\footnotesize\ttfamily, showspaces=false, showstringspaces=false, caption = Reducción de señalamiento por prioridad de señales, label = {lst:EJ3_3}]
	Reducing redundant signals
	removing sig52 for sig01
	removing sig53 for sig02
	removing sig25 for sig04
	removing sig54 for sig05
	removing sig55 for sig06
	removing sig79 for sig06
	removing sig85 for sig08
	removing sig58 for sig09
	removing sig59 for sig10
	removing sig108 for sig11
	removing sig116 for sig16
	removing sig122 for sig16
	removing sig115 for sig18
	removing sig118 for sig20
	removing sig123 for sig22
	removing sig26 for sig44
	removing sig44 for sig26
	removing sig46 for sig26
	removing sig71 for sig36
	removing sig36 for sig71
	removing sig39 for sig68
	removing sig68 for sig39
	removing sig47 for sig45
	removing sig45 for sig80
	removing sig50 for sig48
	removing sig51 for sig49
	removing sig78 for sig53
	removing sig95 for sig56
	removing sig112 for sig57
	removing sig82 for sig66
	removing sig109 for sig67
	removing sig76 for sig74
	removing sig77 for sig75
	removing sig88 for sig93
	removing sig107 for sig105
	\end{lstlisting}
	
	El resultado de la simplificación del señalamiento se ilustra en la Figura \ref{fig:EJ3_7}.
	
	\begin{figure}[H]
		\centering
		\includegraphics[width=1\textwidth]{resultados-obtenidos/ejemplo3/images/3_RNA.png}
		\centering\caption{Señalamiento generado y simplificado por el RNA.}
		\label{fig:EJ3_7}
	\end{figure}
	
	Al finalizar la generación del señalamiento, el RNA debe detectar todas las posibles rutas admitidas por la red para crear la tabla de enclavamientos. El RNA exporta los resultados del análisis en los siguientes cuatro documentos: Infrastructure.RNA (Apéndice \ref{sec:infrastructureRNA}), SafePoint.RNA (Apéndice \ref{sec:safePointsRNA}), Signalling.RNA (Apéndice \ref{sec:signallingRNA}) y Routes.RNA (Apéndice \ref{sec:routesRNA}).