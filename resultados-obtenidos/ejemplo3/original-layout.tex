\section{Topología ferroviaria original}

	El tercer ejemplo, ilustrado en la Figura \ref{fig:EJ3_1}, es una de los ejemplos provistos por railML.org llamado 'AdvancedExample'. Esta topología es de las mas complejas disponibles para analizar. Consiste en siete estaciones interconectadas por medio de variadas topologías ferroviarias que incluyen el uso de cambios de vías simples, dobles y en tijeras. También presenta varias playas de maniobras, curvas, pasos bajo nivel y sobre nivel. El objetivo de este ejemplo fue comprobar el funcionamiento del RNA con la topología mas compleja disponible, exigiendo al máximo al ACG y verificando la escalabilidad del sistema generado.
	
	\begin{figure}[H]
		\centering
		\includegraphics[width=1\textwidth]{resultados-obtenidos/ejemplo3/images/3_empty.png}
		\centering\caption{Topología ferroviaria del ejemplo 3 sin señalamiento.}
		\label{fig:EJ3_1}
	\end{figure}
	
	Debido a la alta complejidad de este ejemplo, no será posible referirse a detalles muy puntuales de la red, al ser demasiados y muy extensos. No obstante, se listarán cada una de las señales creadas en cada paso y se ilustrarán los resultados a la par del proceso. Se extraerán conclusiones generales del análisis y se dará un detalle pormenorizado de las diferencias entre el señalamiento original y el generado.
