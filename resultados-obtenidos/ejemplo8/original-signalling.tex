\section{Señalamiento original}

    El señalamiento original, ilustrado en la Figura \ref{fig:EJ8_2}, incluye señales de parada próximas a los finales de vías relativos (S01, S02, S03, S04), señales de partida en las plataformas (S10, S16), señales de protección antes de cada paso a nivel (S06, S07, S08, S10, S11, S12, S13, S14, S15, S16), entre varias otras señales.
    
    \begin{figure}[H]
    	\centering
    	\includegraphics[width=1\textwidth]{resultados-obtenidos/ejemplo8/images/8_original.png}
    	\centering\caption{Señalamiento original del ejemplo 8.}
    	\label{fig:EJ8_2}
    \end{figure}
    
    Estas señales permiten definir hasta un máximo de 12 rutas, todas ellas detalladas en la Tabla \ref{Tab:tabla_original_8}. En una primera inspección, se puede comprobar que todos los elementos ferroviarios son alcanzados por al menos una de las rutas, en al menos una dirección. Además, todos los cambios de vías son utilizados, de forma simple o compuesta. 
    
    \begin{table}[H]
        {
        \caption{Tabla de enclavamiento original del ejemplo 8.}
        \label{Tab:tabla_original_8}
        \centering
        \resizebox{1\textwidth}{!}{
            \begin{tabular}{ c c c c c c c }
                \hline	
                    Ruta & Inicio & Final & Cambio & Plataforma & Cruce & netElement \\	
                \hline
                    R$_{01}$  & S$_{06}$ & S$_{16}$ & - & Plat$_{01}$ & Lc$_{03}$ & ne$_{26}$\\
                    R$_{02}$  & S$_{07}$ & S$_{05}$ & - & Plat$_{01}$ & Lc$_{04}$ & ne$_{26}$\\
                    R$_{03}$  & S$_{08}$ & S$_{10}$ & - & Plat$_{02}$ & Lc$_{07}$ & ne$_{23}$\\
                    R$_{04}$  & S$_{10}$ & S$_{12}$ & Sw$_{11}^{N}$ & - & Lc$_{09}$ & ne$_{23}$-ne$_{25}$\\
                    R$_{05}$  & S$_{10}$ & S$_{14}$ & Sw$_{11}^{R}$+Sw$_{12}^{R}$ & - & Lc$_{09}$ & ne$_{23}$-ne$_{03}$\\
                    R$_{06}$  & S$_{11}$ & S$_{09}$ & - & Plat$_{02}$ & Lc$_{09}$ & ne$_{23}$\\
                    R$_{07}$  & S$_{12}$ & S$_{03}$ & - & - & Lc$_{05}$ & ne$_{25}$\\
                    R$_{08}$  & S$_{13}$ & S$_{11}$ & - & - & Lc$_{05}$ & ne$_{25}$-ne$_{23}$\\
                    R$_{09}$  & S$_{14}$ & S$_{01}$ & - & - & Lc$_{06}$ & ne$_{03}$\\
                    R$_{10}$  & S$_{15}$ & S$_{07}$ & Sw$_{12}^{N}$ & - & Lc$_{06}$ & ne$_{03}$-ne$_{26}$\\
                    R$_{11}$  & S$_{15}$ & S$_{11}$ & Sw$_{11}^{R}$+Sw$_{12}^{N}$ & - & Lc$_{06}$ & ne$_{03}$-ne$_{23}$\\
                    R$_{12}$  & S$_{16}$ & S$_{14}$ & - & - & Lc$_{04}$ & ne$_{26}$-ne$_{03}$\\
                \hline
            \end{tabular}
        }
     }
    \end{table}
    
    Algunas rutas abarcan mas de un \textit{netElement}, como por ejemplo la ruta R11 que comienza en la señal S15 y finaliza en la señal S11, atravesando los \textit{netElements} ne03 y ne23, utilizando los cambios de vías Sw11 y Sw12, en posición reversa y normal respectivamente.