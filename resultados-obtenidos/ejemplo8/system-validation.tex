\section{Validación del sistema de enclavamientos}

    La validación de las rutas de la tabla de enclavamientos es realizada por el RNA aplicando el Algoritmo \ref{alg:interlocking_tables}, explicado en la Sección \ref{sec:validar_tabla}. Las 12 rutas del señalamiento original (Tabla \ref{Tab:tabla_original_8}) tienen 12 rutas equivalentes en el señalamiento generado por el RNA (Tabla \ref{Tab:tabla_generated_8}), tal como se puede visualizar en la Tabla \ref{Tab:tabla_validation_8}, generada automáticamente por el RNA.

    \begin{table}[H]
        {
        \caption{Equivalencias entre las rutas originales y las generadas por el RNA.}
        \label{Tab:tabla_validation_8}
        \centering
        %\small
            %\centering
            \begin{center}
            \resizebox{0.5\textwidth}{!}{
            \begin{tabular}{ c c c c }
                \hline	
                    Original & Señales & RNA & Señales \\	
                \hline
                    R$_{01}$ & S$_{06}$-S$_{16}$ & R$_{02}$ & X$_{06}$-C$_{24}$ \\
                    R$_{02}$ & S$_{07}$-S$_{05}$ & R$_{03}$ & X$_{07}$-X$_{05}$ \\
                    R$_{03}$ & S$_{08}$-S$_{10}$ & R$_{04}$ & X$_{09}$-S$_{22}$ \\
                    R$_{04}$ & S$_{10}$-S$_{12}$ & R$_{10}$ & S$_{22}$-X$_{13}$ \\
                    R$_{05}$ & S$_{10}$-S$_{14}$ & R$_{11}$ & S$_{22}$-X$_{16}$ \\
                    R$_{06}$ & S$_{11}$-S$_{09}$ & R$_{09}$ & P$_{19}$-L$_{02}$ \\
                    R$_{07}$ & S$_{12}$-S$_{03}$ & R$_{06}$ & X$_{13}$-L$_{03}$ \\
                    R$_{08}$ & S$_{13}$-S$_{11}$ & R$_{07}$ & X$_{14}$-X$_{12}$ \\
                    R$_{09}$ & S$_{14}$-S$_{01}$ & R$_{08}$ & X$_{16}$-L$_{01}$ \\
                    R$_{10}$ & S$_{15}$-S$_{07}$ & R$_{13}$ & S$_{25}$-X$_{07}$ \\
                    R$_{11}$ & S$_{15}$-S$_{11}$ & R$_{14}$ & S$_{25}$-X$_{12}$ \\
                    R$_{12}$ & S$_{16}$-S$_{14}$ & R$_{12}$ & C$_{24}$-X$_{16}$ \\
                \hline
            \end{tabular}
            }
            \end{center}
        }    
    \end{table}
    
    Las rutas R1 y R5 (Tabla \ref{Tab:tabla_generated_1}) generadas por el RNA que no tienen equivalencias en el señalamiento original (Tabla \ref{Tab:tabla_original_1}) se deben a que el RNA creó señales extras. La ruta R1 es definida por el RNA al crear la señal X05 entre la plata forma Plat01 y el cruce de vías Lc03, lo cual genera una parada intermedia que antes no existía, hasta culminar la ruta en la señal L04. De la misma manera, el RNA define la ruta R5 al crear la señal P19, lo cual genera una parada intermedia entre la plataforma Plat02 y el cruce de vías Lc07. En definitiva, el RNA dividió las rutas R2 y R6 originales en las nuevas rutas R3+R1 y R5+R9. Lo cual mejora la logística de la red ferroviaria.
    
    Para finalizar, el RNA comprueba los principios de señalamiento ferroviario explicados en la Sección \ref{sec:validar_principios}, aplicando los algoritmos indicados, de los cuáles se obtuvieron los siguientes resultados:
    
    \begin{itemize}
    	\item Principio de autoridad (Algoritmo \ref{alg:ppio_autoridad}): cobertura del 100\% de los \textit{netElements}.
    	\item Principio de claridad (Algoritmo \ref{alg:ppio_claridad}): rutas 100\% independientes.
    	\item Principio de anticipación (Algoritmo \ref{alg:ppio_anticipacion}): cobertura del 100\% de los puntos críticos.
    	\item Principio de granularidad (Algoritmo \ref{alg:ppio_granularidad}): 100\% de rutas divididas a su mínima expresión.
    	\item Principio de terminalidad (Algoritmo \ref{alg:ppio_terminalidad}): 100\% de finales de vías protegidos.
    	\item Principio de infraestructura (Algoritmo \ref{alg:ppio_infraestructura}): 100\% de infraestructura protegida.
    	\item Principio de no bloqueo (Algoritmo \ref{alg:ppio_nobloqueo}): 100\% de cambios de vías protegidos.
    \end{itemize}	