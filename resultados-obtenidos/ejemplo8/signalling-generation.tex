\section{Generación de señalamiento paso a paso}

	Inicialmente, el RNA ejecuta el Algoritmo \ref{alg:graph_network} (ver Sección \ref{sec:grafos}) para detectar todos los \textit{netElements}, sus coordenadas iniciales y finales en la topología, y el sentido en el que fueron definidas. Al concluir el Algoritmo \ref{alg:graph_network}, el RNA ejecuta el Algoritmo \ref{alg:connectedness} (ver Sección \ref{sec:grafos}) para analizar la conexidad de la red. El resultado obtenido se muestra en el Código \ref{lst:EJ8_1}, donde se describen las coordenadas de cada \textit{netElement} y se confirma que la red es conexa.
	
	\begin{lstlisting}[language = {}, caption = Detección de \textit{netElements} por parte del RNA , label = {lst:EJ8_1}]
	###### Starting Railway Network Analyzer #####
	Reading .railML file
	Creating railML object
	Analyzing railML object
	Analyzing graph
	ne3 [2040, 120] [1020, 120] <<
	ne23 [-960, -300] [600, -300] >>
	ne25 [600, -300] [2040, -300] >>
	ne26 [1020, 120] [-960, 120] <<
	ne27 [600, -300] [1020, 120] >>
	The network is connected
	\end{lstlisting}
	
	Por ejemplo, el \textit{netElement} ne26 inicia en la coordenada (1020;120) y finaliza en la coordenada (-960;120). El símbolo $<<$ indica que ne1 se encuentra definido de derecha a izquierda, ya que la componente x de la coordenada final es menor a la de la coordenada inicial, teniendo la misma componente y. Además, se puede comprobar que la lista obtenida en consistente con la Figura \ref{fig:EJ8_2}. Por ejemplo, ne03, ne26 y ne27 comparten la coordenada (1020;120), que coincide con la coordenada del cambio de vías Sw12.
	
	A continuación, el RNA detectará la infraestructura ferroviaria, las curvas peligrosas y los puntos medios de los netElements que el RNA considera demasiado largos. El análisis de la infraestructura se detalla en la Sección \ref{sec:bufferstop}, Sección \ref{sec:detectors}, Sección \ref{sec:platform} y Sección \ref{sec:crossing}, mientras que la detección de curvas y puntos medios se detalla en la Sección \ref{sec:tracks}. El RNA ejecuta el Algoritmo \ref{alg:switches_1}, Algoritmo \ref{alg:switches_2} y Algoritmo \ref{alg:switches_3} para confirmar la detección de cambios de vías simples, dobles y en tijeras. El resultado de este proceso se puede visualizar en el Código \ref{lst:EJ8_2}.
	
	\begin{lstlisting}[language = {}, caption = Detección de puntos críticos por parte del RNA , label = {lst:EJ8_2}]
	Analyzing infrastructure --> Infrastructure.RNA
	Detecting Danger --> Safe_points.RNA
	ne3 has a LevelCrossing[Lc06] @ [1536, -120]
	ne23 has a Platform[Plat02] @ [-160, 300]
	ne23 has a LevelCrossing[Lc07] @ [-494, 300]
	ne23 has a LevelCrossing[Lc09] @ [266, 300]
	ne25 has a LevelCrossing[Lc05] @ [1166, 300]
	ne26 has a Platform[Plat01] @ [-41, -120]
	ne26 has a LevelCrossing[Lc03] @ [-404, -120]
	ne26 has a LevelCrossing[Lc04] @ [572, -120]
	\end{lstlisting}
	
	Esta información es exportada por el RNA, con mayor detalle, en el archivo Infrastructure.RNA (Código \ref{lst:EJ8_4}) que resume cada elemento ferroviario asociado a su respectivo \textit{netElement}.
	
	\begin{lstlisting}[language = {}, caption = Infrastructure.RNA, label = {lst:EJ8_4}]
Nodes: 5|Switches: 2|Signals: 0|Detectors: 0|Ends: 4|Barriers: 6
Node ne3:
	Track = track2
	Neighbours = 2 -> ['ne26', 'ne27']
	Level crossing -> Lc06
		Protection -> true | Barriers -> none | Lights -> none Acoustic -> none
		Position -> [1491, -120] | Coordinate: 0.5373
	Switches -> Sw12
		ContinueCourse -> right -> ne26
		BranchCourse -> left -> ne27
Node ne23:
	Track = track3
	Neighbours = 2 -> ['ne25', 'ne27']
	Level crossing -> Lc07
		Protection -> true | Barriers -> none | Lights -> none Acoustic -> none
		Position -> [-449, 300] | Coordinate: 0.3273
	Level crossing -> Lc09
		Protection -> true | Barriers -> none | Lights -> none Acoustic -> none
		Position -> [311, 300] | Coordinate: 0.8148
	Switches -> Sw11
		ContinueCourse -> right -> ne25
		BranchCourse -> left -> ne27
Node ne25:
	Track = track4
	Neighbours = 2 -> ['ne23', 'ne27']
	Level crossing -> Lc05
		Protection -> true | Barriers -> none | Lights -> none Acoustic -> none
		Position -> [1211, 300] | Coordinate: 0.4248
Node ne26:
	Track = track1
	Neighbours = 2 -> ['ne3', 'ne27']
	Level crossing -> Lc03
		Protection -> true | Barriers -> none | Lights -> none Acoustic -> none
		Position -> [-449, -120] | Coordinate: 0.7421
	Level crossing -> Lc04
		Protection -> true | Barriers -> none | Lights -> none Acoustic -> none
		Position -> [527, -120] | Coordinate: 0.2487
Node ne27:
	Track = track5
	Neighbours = 4 -> ['ne3', 'ne23', 'ne25', 'ne26']
	\end{lstlisting}
	
	La información de la infraestructura es utilizada por el RNA para detectar los puntos críticos de la red, es decir, las zonas donde es recomendable colocar una señal, según el sentido de circulación que se desee. El resultado es exportado al archivo SafePoints.RNA (Código \ref{lst:EJ8_5}). En el caso de que un mismo \textit{netElement} tenga más de un punto crítico para el mismo sentido, el RNA tomará el más cercano al elemento a proteger. El criterio de selección de puntos críticos se aplica para cada elemento ferroviario detectado, cada curva y cada cambio de vías y fue explicado en las secciones correspondientes ya mencionadas.
	
	\begin{lstlisting}[language = {}, caption = SafePoints.RNA, label = {lst:EJ8_5}]
ne3:
	Next: [[1336, -120]]
	Prev: [[1736, -120]]
ne23:
	Next: [[-360, 300], [-694, 300], [66, 300]]
	Prev: [[40, 300], [-294, 300], [466, 300]]
ne25:
	Next: [[966, 300]]
	Prev: [[1366, 300]]
ne26:
	Next: [[-241, -120], [-604, -120], [372, -120]]
	Prev: [[159, -120], [-204, -120], [772, -120]]
	\end{lstlisting}	
	
	Una vez que el RNA detectó cada punto crítico de la red ferroviaria, procede a generar el señalamiento. El orden en que se procesan los elementos ferroviarios no impacta en el resultado final, pero para poder describirlo de forma ordenada se iniciará generando el señalamiento para proteger los finales de vías, las junturas entre rieles, las plataformas (explicado en la Sección \ref{sec:sig_platform}), los cruces de vía (explicado en la Sección \ref{sec:sig_levelcrossing}) y los cambios de vías (explicado en la Sección \ref{sec:signal_switches}). Luego se procederá a mostrar el señalamiento completo antes y después de la simplificación de señales (explicado en la Sección \label{sec:simplificacion}). 
	
	Tal cómo se explicó en la Sección \ref{sec:sig_border}, el RNA aplica el Algoritmo \ref{alg:lineBorder} y el Algoritmo \ref{alg:bufferStop} para generar las señales para proteger los finales de vías relativos y absolutos. Estas señales son ilustradas en la Figura \ref{fig:EJ8_3}.
	
	\begin{figure}[H]
		\centering
		\includegraphics[width=1\textwidth]{resultados-obtenidos/ejemplo8/images/8_step1.png}
		\centering\caption{Señalamiento generado por el RNA para proteger el fin de vía.}
		\label{fig:EJ8_3}
	\end{figure}
	
	Al no existir finales de vías absolutos, el RNA no les asignó señalamiento. En cambio, los finales de vías relativos poseen las señales de parada L01, L02, L03 y L04, cercanos al límite del externo del \textit{netElement} al que pertenecen.  La Figura \ref{fig:EJ8_4} no presenta cambios en el señalamiento, al no existir junturas entre los rieles que proteger.
	
	\begin{figure}[H]
		\centering
		\includegraphics[width=1\textwidth]{resultados-obtenidos/ejemplo8/images/8_step2.png}
		\centering\caption{Señalamiento generado por el RNA para proteger las junturas.}
		\label{fig:EJ8_4}
	\end{figure}
	
	Al generar el señalamiento para proteger la infraestructura, tal como se explicó en la Sección \ref{sec:horizontal}, el Algoritmo \ref{alg:horizontal} simplificará las señales entre dos elementos ferroviarios si no existe espacio suficiente entre ellos, tal como sucede con los elementos LevelCrossing03 y Platform01. El señalamiento generado para proteger las plataformas y los cruces de vía, producto de aplicar el Algoritmo \ref{alg:PTF} y el Algoritmo \ref{alg:LC}, respectivamente, se ilustra en rojo en la Figura \ref{fig:EJ8_5}. Las señales generadas para proteger las plataformas son las señales de partida P17, P18, P19 y P20, mientras que las señales que protegen los cruces de vía son todas las señales entre X05 y X16.
	
	\begin{figure}[H]
		\centering
		\includegraphics[width=1\textwidth]{resultados-obtenidos/ejemplo8/images/8_step3.png}
		\centering\caption{Señalamiento generado por el RNA para proteger plataformas y cruces de vía.}
		\label{fig:EJ8_5}
	\end{figure}
	
	Al aplicar el Algoritmo \ref{alg:SW} de generación de señalamiento para cambios de vías, tal como fue explicado en la Sección \label{sec:signal_switches}, el RNA genera las señales S22, C21 y H23 para proteger el cambio de vías Sw11 y las señales S25, C24 y H26 para proteger el cambio de vías Sw12. Las señales mencionadas se encuentran resaltadas en rojo en la Figura \ref{fig:EJ8_6}.
	
	\begin{figure}[H]
		\centering
		\includegraphics[width=1\textwidth]{resultados-obtenidos/ejemplo8/images/8_step4.png}
		\centering\caption{Señalamiento generado por el RNA para proteger los cambios de vías.}
		\label{fig:EJ8_6}
	\end{figure}
	
	Una vez obtenido todo el señalamiento, el RNA procede a simplificar las señales redundantes, repetidas o cuyas funciones o ubicaciones se superponen entre sí. El proceso de simplificación de señales fue explicado en la Sección \ref{sec:simplificacion}. El Algoritmo \ref{alg:vertical} de herencia vertical fue aplicado en las señales B entre los cambios de vías Sw11 y Sw12, desplazando las señales hasta convertirlas en las señales H33 y H36 respectivamente.
	
	Las señales simplificadas al aplicar el Algoritmo \ref{alg:horizontal} de herencia horizontal son: X11, H23, P18, X08, H26, X15, S19, P17 y X10. Las señales P20, X11 y H23 fueron eliminadas por su cercanía con la señal S22, con la cual comparten dirección y sentido. Lo mismo ocurre entre las señales H26 y X15. En todos los casos, se aplicó el Algoritmo \ref{alg:horizontal}, diseñado para agrupar objetos cercanos como un único objeto, generando el señalamiento acorde a los elementos contenidos en cada extremo del nuevo elemento contenedor.
	
	Finalmente, las señales son simplificadas aplicando el Algoritmo \ref{alg:reduction} de eliminación por prioridad de señales. El resultado de este proceso es detallado en el Código \ref{lst:EJ8_3}.
	
	\begin{lstlisting}[language = {}, caption = Reducción de señalamiento por prioridad de señales, label = {lst:EJ8_3}]
	Reducing redundant signals
	removing sig17 for sig05
	removing sig18 for sig08
	removing sig08 for sig24
	removing sig10 for sig19
	removing sig11 for sig22
	removing sig21 for sig14
	removing sig15 for sig25
	removing sig20 for sig22
	removing sig23 for sig22
	removing sig26 for sig25
	\end{lstlisting}
	
	El resultado de la simplificación del señalamiento se ilustra en la Figura \ref{fig:EJ8_7}.
	
	\begin{figure}[H]
		\centering
		\includegraphics[width=1\textwidth]{resultados-obtenidos/ejemplo8/images/8_RNA.png}
		\centering\caption{Señalamiento generado y simplificado por el RNA.}
		\label{fig:EJ8_7}
	\end{figure}
	
	Además, toda la información del señalamiento generado es exportada por el RNA en el archivo Signalling.RNA (Código \ref{lst:EJ1_6}), que incluye información detallada de la posición, orientación, sentido, coordenada, nombre y tipo de señal.
	
	\begin{lstlisting}[language = {}, caption = Signalling.RNA, label = {lst:EJ8_6}]
sig01 [L01] >>:
	From: ne3 | To: lb70_right
	Type: Circulation | Direction: reverse | AtTrack: right 
	Position: [1940, -120] | Coordinate: 0.9019
sig02 [L02] <<:
	From: ne23 | To: oe178_left
	Type: Circulation | Direction: reverse | AtTrack: right 
	Position: [-860, 300] | Coordinate: 0.0641
sig03 [L03] >>:
	From: ne25 | To: oe179_right
	Type: Circulation | Direction: normal | AtTrack: left 
	Position: [1940, 300] | Coordinate: 0.9305
sig04 [L04] <<:
	From: ne26 | To: lb69_left
	Type: Circulation | Direction: normal | AtTrack: left 
	Position: [-860, -120] | Coordinate: 0.0505
sig05 [X05] <<:
	From: ne26 | To: ne26_right
	Type: Circulation | Direction: normal | AtTrack: left 
	Position: [-204, 120] | Coordinate: 0.4005
sig06 [X06] >>:
	From: ne26 | To: ne26_left
	Type: Circulation | Direction: reverse | AtTrack: right 
	Position: [-604, 120] | Coordinate: 0.2168
sig07 [X07] <<:
	From: ne26 | To: ne26_right
	Type: Circulation | Direction: normal | AtTrack: left 
	Position: [772, 120] | Coordinate: 0.8831
sig09 [X09] >>:
	From: ne23 | To: ne23_right
	Type: Circulation | Direction: normal | AtTrack: left 
	Position: [-694, -300] | Coordinate: 0.4207
sig12 [X12] <<:
	From: ne23 | To: ne23_left
	Type: Circulation | Direction: reverse | AtTrack: right 
	Position: [466, -300] | Coordinate: 0.9917
sig13 [X13] >>:
	From: ne25 | To: ne25_right
	Type: Circulation | Direction: normal | AtTrack: left 
	Position: [966, -300] | Coordinate: 0.4880
sig14 [X14] <<:
	From: ne25 | To: ne25_left
	Type: Circulation | Direction: reverse | AtTrack: right 
	Position: [1366, -300] | Coordinate: 0.6757
sig16 [X16] >>:
	From: ne3 | To: ne3_left
	Type: Circulation | Direction: reverse | AtTrack: right 
	Position: [1336, 120] | Coordinate: 0.3890
sig19 [P19] <<:
	From: ne23 | To: ne23_left
	Type: Circulation | Direction: reverse | AtTrack: right 
	Position: [-280, -300] | Coordinate: 0.5813
sig22 [S22] >>:
	From: ne23 | To: ne23_right
	Type: Circulation | Direction: normal | AtTrack: left 
	Position: [40, -300] | Coordinate: 0.7475
sig24 [C24] >>:
	From: ne26 | To: ne26_right
	Type: Circulation | Direction: reverse | AtTrack: right 
	Position: [372, 120] | Coordinate: 0.6835
sig25 [S25] <<:
	From: ne3 | To: ne3_left
	Type: Circulation | Direction: normal | AtTrack: left 
	Position: [1736, 120] | Coordinate: 0.7403
	\end{lstlisting}
	
	Al finalizar la generación del señalamiento, el RNA ejecuta el Algoritmo \ref{alg:routes}, explicado en la Sección \ref{sec:rutas}, para detectar todas las posibles rutas admitidas por la red para crear la tabla de enclavamientos. La cuál puede ser visualizada en el archivo Routes.RNA (Código \ref{lst:EJ8_7}). La misma detalla las señales de inicio y final, los \textit{netElements} abarcados por la ruta y cualquier infraestructura involucrada, incluyendo el estado que deben tener para que la ruta sea activada.
	
	\begin{lstlisting}[language = {}, caption = Routes.RNA, label = {lst:EJ8_7}]
route_1 [X05 << L04]:
	Path: ['ne26']
	LevelCrossings: ['Lc03']
route_2 [X06 >> C24]:
	Path: ['ne26']
	Platforms: ['Plat01']
	LevelCrossings: ['Lc03']
route_3 [X07 << X05]:
	Path: ['ne26']
	Platforms: ['Plat01']
	LevelCrossings: ['Lc04']
route_4 [X09 >> S22]:
	Path: ['ne23']
	Platforms: ['Plat02']
	LevelCrossings: ['Lc07']
route_5 [X12 << P19]:
	Path: ['ne23']
	Platforms: ['Plat02']
	LevelCrossings: ['Lc09']
route_6 [X13 >> L03]:
	Path: ['ne25']
	LevelCrossings: ['Lc05']
route_7 [X14 << X12]:
	Path: ['ne25', 'ne23']
	Switches: ['Sw11_N']
	LevelCrossings: ['Lc05']
route_8 [X16 >> L01]:
	Path: ['ne3']
	LevelCrossings: ['Lc06']
route_9 [P19 << L02]:
	Path: ['ne23']
	LevelCrossings: ['Lc07']
route_10 [S22 >> X13]:
	Path: ['ne23', 'ne25']
	Switches: ['Sw11_N']
	LevelCrossings: ['Lc09']
route_11 [S22 >> X16]:
	Path: ['ne23', 'ne27', 'ne3']
	Switches: ['Sw11_R', 'Sw12_R']
	LevelCrossings: ['Lc09']
route_12 [C24 >> X16]:
	Path: ['ne26', 'ne3']
	Switches: ['Sw12_N']
	LevelCrossings: ['Lc04']
route_13 [S25 << X07]:
	Path: ['ne3', 'ne26']
	Switches: ['Sw12_N']
	LevelCrossings: ['Lc06']
route_14 [S25 << X12]:
	Path: ['ne3', 'ne27', 'ne23']
	Switches: ['Sw11_R', 'Sw12_R']
	LevelCrossings: ['Lc06']
	\end{lstlisting}	