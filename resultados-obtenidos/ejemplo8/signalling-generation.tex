\subsection{Generación de señalamiento paso a paso}

	Al ejecutar el RNA, primero detectará todos los \textit{netElements}, sus coordenadas iniciales y finales en la topología, y el sentido en el que fueron definidas. El resultado obtenido se muestra en el Cóodigo \ref{lst:EJ8_1}.
	
	\begin{lstlisting}[language = {}, caption = Detección de \textit{netElements} por parte del RNA , label = {lst:EJ8_1}]
	###### Starting Railway Network Analyzer #####
	Reading .railML file
	Creating railML object
	Analyzing railML object
	Analyzing graph
	ne3 [2040, 120] [1020, 120] <<
	ne23 [-960, -300] [600, -300] >>
	ne25 [600, -300] [2040, -300] >>
	ne26 [1020, 120] [-960, 120] <<
	ne27 [600, -300] [1020, 120] >>
	The network is connected
	\end{lstlisting}
	
	Por ejemplo, el \textit{netElement} ne26 inicia en la coordenada (1020;120) y finaliza en la coordenada (-960;120). El símbolo $<<$ indica que ne1 se encuentra definido de derecha a izquierda, ya que la componente x de la coordenada final es menor a la de la coordenada inicial, teniendo la misma componente y. Además, se puede comprobar que la lista obtenida en consistente con la Figura \ref{fig:EJ8_2}. Por ejemplo, ne03, ne26 y ne27 comparten la coordenada (1020;120), que coincide con la coordenada del cambio de vías Sw12.
	
	A continuación, el RNA detectará la infraestructura ferroviaria, las curvas peligrosas y los puntos medios de los netElements que el RNA considera demasiado largos. El resultado de este proceso se puede visualizar en el Código \ref{lst:EJ8_2} y puede leerse también en el archivo Infrastructure.RNA.
	
	\begin{lstlisting}[language = {}, caption = Detección de puntos críticos por parte del RNA , label = {lst:EJ8_2}]
	Analyzing infrastructure --> Infrastructure.RNA
	Detecting Danger --> Safe_points.RNA
	ne3 has a LevelCrossing[Lc06] @ [1536, -120]
	ne23 has a Platform[Plat02] @ [-160, 300]
	ne23 has a LevelCrossing[Lc07] @ [-494, 300]
	ne23 has a LevelCrossing[Lc09] @ [266, 300]
	ne25 has a LevelCrossing[Lc05] @ [1166, 300]
	ne26 has a Platform[Plat01] @ [-41, -120]
	ne26 has a LevelCrossing[Lc03] @ [-404, -120]
	ne26 has a LevelCrossing[Lc04] @ [572, -120]
	\end{lstlisting}
	
	Una vez que el RNA detectó cada punto crítico de la red ferroviaria, procede a generar el señalamiento. El orden de generación no es importante, pero para poder describirlo de forma consistente se iniciará generando el señalamiento para proteger los finales de vías, las junturas entre rieles, las plataformas, los cruces de vía y los cambios de vías. Luego se procederá a mostrar el señalamiento pre y post simplificación. Las señales generadas para proteger los finales de vías relativos y absolutos son ilustradas en la Figura \ref{fig:EJ8_3}.
	
	\begin{figure}[H]
		\centering
		\includegraphics[width=1\textwidth]{resultados-obtenidos/ejemplo8/images/8_step1.png}
		\centering\caption{Señalamiento generado por el RNA para proteger el fin de vía.}
		\label{fig:EJ8_3}
	\end{figure}
	
	Al no existir finales de vías absolutos, el RNA no les asignó señalamiento. En cambio, los finales de vías relativos poseen las señales de parada L01, L02, L03 y L04, cercanos al límite del externo del \textit{netElement} al que pertenecen.
	
	La Figura \ref{fig:EJ8_4} no presenta cambios en el señalamiento, al no existir junturas entre los rieles que proteger.
	
	\begin{figure}[H]
		\centering
		\includegraphics[width=1\textwidth]{resultados-obtenidos/ejemplo8/images/8_step2.png}
		\centering\caption{Señalamiento generado por el RNA para proteger las junturas.}
		\label{fig:EJ8_4}
	\end{figure}
	
	Al generar el señalamiento para proteger la infraestructura, tal como se explicó en la Sección \ref{sec:horizontal}, el Algoritmo \ref{alg:horizontal} simplificará las señales entre dos elementos ferroviarios si no existe espacio suficiente entre ellos, tal como sucede con los elementos Lc03 y Plat01. El señalamiento generado para proteger las plataformas y los cruces de vía se ilustra en rojo en la Figura \ref{fig:EJ8_5}. Las señales generadas para proteger las plataformas son las señales de partida P17, P18, P19 y P20, mientras que las señales que protegen los cruces de vía son todas las señales entre X05 y X16.
	
	\begin{figure}[H]
		\centering
		\includegraphics[width=1\textwidth]{resultados-obtenidos/ejemplo8/images/8_step3.png}
		\centering\caption{Señalamiento generado por el RNA para proteger plataformas y cruces de vía.}
		\label{fig:EJ8_5}
	\end{figure}
	
	El RNA generó las señales S22, C21 y H23 para proteger el cambio de vías Sw11 y las señales S25, C24 y H26 para proteger el cambio de vías Sw12. Las señales mencionadas se encuentran resaltadas en rojo en la Figura \ref{fig:EJ8_6}.
	
	\begin{figure}[H]
		\centering
		\includegraphics[width=1\textwidth]{resultados-obtenidos/ejemplo8/images/8_step4.png}
		\centering\caption{Señalamiento generado por el RNA para proteger los cambios de vías.}
		\label{fig:EJ8_6}
	\end{figure}
	
	Una vez obtenido todo el señalamiento, el RNA procede a simplificar las señales redundantes, repetidas o cuyas funciones o ubicaciones se superponen entre sí. El proceso de simplificación de señales fue explicado en la Sección \label{sec:simplificacion}. El Algoritmo \ref{alg:vertical} de herencia vertical fue aplicado en las señales B entre los cambios de vías Sw11 y Sw12, desplazando las señales hasta convertirlas en las señales H33 y H36 respectivamente.
	
	Las señales simplificadas al aplicar el Algoritmo \ref{alg:horizontal} de herencia horizontal son: X11, H23, P18, X08, H26, X15, S19, P17 y X10. Las señales P20, X11 y H23 fueron eliminadas por su cercanía con la señal S22, con la cual comparten dirección y sentido. Lo mismo ocurre entre las señales H26 y X15. En todos los casos, se aplicó el Algoritmo \ref{alg:horizontal}, diseñado para agrupar objetos cercanos como un único objeto, generando el señalamiento acorde a los elementos contenidos en cada extremo del nuevo elemento contenedor.
	
	Finalmente, las señales son simplificadas aplicando el Algoritmo \ref{alg:reduction} de eliminación por prioridad de señales. El resultado de este proceso es detallado en el Código \ref{lst:EJ8_3}.
	
	\begin{lstlisting}[language = {}, caption = Reducción de señalamiento por prioridad de señales, label = {lst:EJ8_3}]
	Reducing redundant signals
	removing sig17 for sig05
	removing sig18 for sig08
	removing sig08 for sig24
	removing sig10 for sig19
	removing sig11 for sig22
	removing sig21 for sig14
	removing sig15 for sig25
	removing sig20 for sig22
	removing sig23 for sig22
	removing sig26 for sig25
	\end{lstlisting}
	
	El resultado de la simplificación del señalamiento se ilustra en la Figura \ref{fig:EJ8_7}.
	
	\begin{figure}[H]
		\centering
		\includegraphics[width=1\textwidth]{resultados-obtenidos/ejemplo8/images/8_RNA.png}
		\centering\caption{Señalamiento generado y simplificado por el RNA.}
		\label{fig:EJ8_7}
	\end{figure}
	
	Al finalizar la generación del señalamiento, el RNA debe detectar todas las posibles rutas admitidas por la red para crear la tabla de enclavamientos. El RNA exporta los resultados del análisis en los siguientes cuatro documentos: Infrastructure.RNA (Apéndice \ref{sec:infrastructureRNA}), SafePoint.RNA (Apéndice \ref{sec:safePointsRNA}), Signalling.RNA (Apéndice \ref{sec:signallingRNA}) y Routes.RNA (Apéndice \ref{sec:routesRNA}).	