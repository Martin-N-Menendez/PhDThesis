\section{Sistema generado por el ACG}
	
	En base a la red de grafos, ilustrada en la Figura \ref{fig:EJ6_8}, el ACG determinó la siguiente cantidad de elementos, tal puede visualizarse en el Código \ref{lst:EJ6_8}.
	
	\begin{lstlisting}[language = {}, caption = Cantidad de elementos a implementar por el ACG, label = {lst:EJ6_8}]
	n_netElements:11
	n_switch:5
	n_doubleSwitch:0
	n_borders:2
	n_buffers:5
	n_levelCrossings:0
	n_platforms:0
	n_scissorCrossings:0
	n_signals:24
	N : 86
	M : 75
	\end{lstlisting}
	
	El código VHDL generado por el ACG es importado en un proyecto de Vivado, donde es sintetizado e implementado para generar el bitstream que será utilizado para programar la FPGA. La cantidad de elementos de la FPGA utilizados por el sistema post-síntesis y post-implementación, así como el porcentaje de uso de la plataforma, son detallados en la Tabla \ref{Tab:tabla_ACG_6}.
	
	\begin{table}[H]
		{
			\caption{Síntesis e implementación del ejemplo 6 generado por el ACG.}
			\label{Tab:tabla_ACG_6}
			\centering
			%\small
			%\centering
			\begin{center}
				\resizebox{0.7\textwidth}{!}{
					\begin{tabular}{ c c c c }
						\hline	
						Recursos & Síntesis & Implementación & Uso \\	
						\hline
						LUT & 3600 & 3596 & 6.77-6.76\%\\
						FF & 4555 & 4555 & 4.28\%\\
						IO & 16 & 16 & 12.80\%\\
						BUFG & 1 & 1 & 3.13\%\\
						\hline
					\end{tabular}
				}
			\end{center}
		}    
	\end{table}
	
	En este ejemplo, la cantidad de recursos utilizados es baja y el tiempo de sintetización e implementación es de 43 segundos y 41 segundos respectivamente.