\section{Validación del sistema de enclavamientos}

    La validación de las rutas de la tabla de enclavamientos es realizada por el RNA aplicando el Algoritmo \ref{alg:interlocking_tables}, explicado en la Sección \ref{sec:validar_tabla}. Las 16 rutas del señalamiento original (Tabla \ref{Tab:tabla_original_6}) tienen 16 rutas equivalentes en el señalamiento generado por el RNA (Tabla \ref{Tab:tabla_generated_6}), tal como se puede visualizar en la Tabla \ref{Tab:tabla_validation_6}, generada automáticamente por el RNA.

    \begin{table}[H]
        {
        \caption{Equivalencias entre las rutas originales y las generadas por el RNA.}
        \label{Tab:tabla_validation_6}
        %\centering
        \begin{center}
        \resizebox{0.5\textwidth}{!}{
	        \begin{tabular}{ c c c c }
	            \hline	
	                Original & Señales & RNA & Señales \\	
	            \hline
	                R$_{01}$ & S$_{01}$-S$_{06}$ & R$_{11}$ & S$_{22}$-S$_{27}$ \\
	                R$_{02}$ & S$_{01}$-S$_{13}$ & R$_{13}$ & S$_{22}$-T$_{03}$ \\
	                R$_{03}$ & S$_{01}$-S$_{18}$ & R$_{12}$ & S$_{22}$-J$_{19}$ \\
	                R$_{04}$ & S$_{06}$-S$_{07}$ & R$_{15}$ & S$_{27}$-S$_{33}$ \\
	                R$_{05}$ & S$_{06}$-S$_{12}$ & R$_{16}$ & S$_{27}$-T$_{01}$ \\
	                R$_{06}$ & S$_{21}$-S$_{19}$ & R$_{21}$ & S$_{37}$-J$_{20}$ \\
	                R$_{07}$ & S$_{21}$-S$_{14}$ & R$_{22}$ & S$_{37}$-B$_{36}$ \\
	                R$_{08}$ & S$_{05}$-S$_{02}$ & R$_{10}$ & C$_{21}$-J$_{18}$ \\
	                R$_{09}$ & S$_{09}$-S$_{08}$ & R$_{06}$ & J$_{14}$-J$_{16}$ \\
	                R$_{10}$ & S$_{08}$-S$_{05}$ & R$_{07}$ & J$_{16}$-C$_{21}$ \\
	                R$_{11}$ & S$_{10}$-S$_{08}$ & R$_{03}$ & T$_{06}$-J$_{16}$ \\
	                R$_{12}$ & S$_{15}$-S$_{16}$ & R$_{05}$ & T$_{10}$-T$_{07}$ \\
	                R$_{13}$ & S$_{18}$-S$_{16}$ & R$_{08}$ & J$_{19}$-T$_{07}$ \\
	                R$_{14}$ & S$_{19}$-S$_{20}$ & R$_{08}$ & J$_{20}$-C$_{29}$ \\
	                R$_{15}$ & S$_{20}$-S$_{02}$ & R$_{17}$ & C$_{29}$-J$_{18}$ \\
	                R$_{16}$ & S$_{07}$-S$_{11}$ & R$_{19}$ & S$_{33}$-T$_{05}$ \\
	            \hline
	        \end{tabular}
		}
		\end{center}
    }    
    \end{table}
    
    Las rutas R1, R2, R4, R14, R18 y R20 (Tabla \ref{Tab:tabla_generated_6}) generadas por el RNA que no tienen equivalencias en el señalamiento original (Tabla \ref{Tab:tabla_original_6}) se deben a que el RNA creó señales extras. La ruta R1 fue creada por el RNA al añadir la señal T02 para proteger el final de vía del \textit{netElement} ne04, creando una ruta con la señal B26. La ruta R2 fue creada por el RNA al añadir la señal T04 para proteger el final de vía del \textit{netElement} ne06, creando una ruta con la señal J18. La ruta R4 fue creada por el RNA al añadir la señal T08 para proteger el final de vía del \textit{netElement} ne41, creando una ruta con la señal S37.
    
    Respecto a las otras tres rutas: la ruta R14 fue creada por el RNA al añadir la señal B26 para proteger al cambio Sw02 previo a la curva del \textit{netElement} ne04, formando una ruta con la señal C21; la ruta R18 fue creada por el RNA al añadir la señal J13 para proteger la juntura  del \textit{netElement} ne11, formando una ruta con la señal S33; la ruta R20 fue creada por el RNA al añadir la señal T09 para proteger el final de vía del \textit{netElement} ne42, creando una ruta con la señal B36.
    
    Los elementos indicados (finales de vías, curvas y cambios de vías), no se encontraban complemente protegidos en el señalamiento original.
    
    Para finalizar, el RNA comprueba los principios de señalamiento ferroviario explicados en la Sección \ref{sec:validar_principios}, aplicando los algoritmos indicados, de los cuáles se obtuvieron los siguientes resultados:
    
    \begin{itemize}
    	\item Principio de autoridad (Algoritmo \ref{alg:ppio_autoridad}): cobertura del 100\% de los \textit{netElements}.
    	\item Principio de claridad (Algoritmo \ref{alg:ppio_claridad}): rutas 100\% independientes.
    	\item Principio de anticipación (Algoritmo \ref{alg:ppio_anticipacion}): cobertura del 100\% de los puntos críticos.
    	\item Principio de granularidad (Algoritmo \ref{alg:ppio_granularidad}): 100\% de rutas divididas a su mínima expresión.
    	\item Principio de terminalidad (Algoritmo \ref{alg:ppio_terminalidad}): 100\% de finales de vías protegidos.
    	\item Principio de infraestructura (Algoritmo \ref{alg:ppio_infraestructura}): 100\% de infraestructura protegida.
    	\item Principio de no bloqueo (Algoritmo \ref{alg:ppio_nobloqueo}): 100\% de cambios de vías protegidos.
    \end{itemize}	