\subsection{Doble recubrimiento}

	\label{sec:function_5}
	
	Para evitar que una formación colisione con otra formación que se encuentre detenida más adelante en la misma vía o circulando a menor velocidad, el sistema de enclavamiento deberá controlar las señales entre ambas para regular la velocidad y distancia entre ellas. Tal como se explicó en la Sección \ref{sec:signals}, las señales pueden presentar diferentes aspectos. Cada aspecto determinará un rango de velocidad permitido, siendo rojo el mas restrictivo. La Figura \ref{fig:ACG_recrubrimiento_1} ilustra el comportamiento del señalamiento cuando dos formaciones circulan en el mismo sentido, separadas por una distancia de seguridad.
	
	\begin{figure}[!h]
		\centering
		\includegraphics[width=1\textwidth]{Figuras/recubrimiento}
		\centering\caption{Protección por doble recubrimiento.}
		\label{fig:ACG_recrubrimiento_1}
	\end{figure}
	
	Debido al bloqueo por ocupación, todas las secciones ocupadas por una formación presentan una señal a peligro (roja). Inmediatamente detrás de cada formación se genera una secuencia de señales denominada doble recubrimiento. La cantidad de señales y la secuencia de aspectos variará según el operador de la red, las normas locales o nacionales. Algunos países utilizan la secuencia rojo-doble amarillo-amarillo-verde y esta es la secuencia de aspectos que implementa el ACG en este trabajo, aunque cabe aclarar que el ACG puede modificarse para implementar otras secuencias. Para simplificar la representación, en la Figura \ref{fig:ACG_recrubrimiento_1} se reemplazó la señal doble amarilla por una señal naranja. En este trabajo siempre se representará mediante una señal naranja a una señal doble amarilla.
	
	La formación que circula por detrás (formación A en la Figura \ref{fig:ACG_recrubrimiento_1}) se encuentra frente a una señal de aspecto verde, por lo
	que puede continuar su marcha sin restricciones, siempre y cuando su velocidad sea menor a la velocidad máxima permitida en la red ferroviaria. Si la formación A reduce la distancia a la formación B, pasará a estar regida por una señal de aspecto amarillo. Si esto sucediera, la formación A deberá disminuir su velocidad para volver a situarse dentro de una sección verde. Lo mismo ocurriría si la formación A alcanzara una señal de aspecto doble amarillo, indicada mediante la señal naranja en la Figura 3.8. En este caso dado que la distancia entre formaciones es aún menor, deberá reducirse aún más la velocidad.
	
	%La formación que circule por detrás se encuentra frente a una señala de aspecto verde, por lo que puede continuar su marcha a la velocidad actual. No obstante, si su velocidad fuese mayor a la de la próxima formación, podría reducir la distancia entre ambas y pasar la señal de aspecto amarillo. Si esto sucediera, la formación deberá disminuir su velocidad para volver a situarse dentro de una sección verde. 
	
	Si la formación continúa teniendo una mayor velocidad que su par, la distancia entre ambas se reducirá y las señales permitirán velocidades mas o mas reducidas, hasta que la distancia se incremente a un valor seguro.