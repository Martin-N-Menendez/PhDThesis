\subsection{Sistemas de enclavamiento}

A modo de ejemplo, se ilustra en la Figura \ref{fig:enclavamiento} un sistema de cambios en una vía simple con bypass. Esto permite que dos formaciones puedan cruzarse en sentidos opuestos sin colisionar.

    \begin{figure}[!h]
        \centering
        \includegraphics[width=1\textwidth]{Figuras/bypass}
        \centering\caption{XXXXX.}
        \label{fig:enclavamiento}
    \end{figure}

Para evitar la colisión, se requiere un control seguro que evite que las formaciones avancen hacia secciones ya ocupadas por otras. También debe evitar que las formaciones avancen sobre los cambios cuando estos aún no han terminado de posicionarse en su lugar, lo que provocaría descarrilamientos. A este control se lo denomina sistema de enclavamiento y, en definitiva, impide que se produzcan configuraciones no seguras y controla los semáforos que habilitan o no los itinerarios de las formaciones. Una falla en un enclavamiento puede poner en peligro cientos de vidas humanas y generar gastos considerables. Por lo tanto, en el diseño del sistema de enclavamiento se deben cumplir estrictos parámetros de fiabilidad, disponibilidad, mantenibilidad y seguridad (RAMS).
    
\subsection{Bloqueo de máquina de cambios por ocupación}

	Evitar el descarrilamiento de las formaciones es una de las funciones del sistema de enclavamientos. Esto puede ocurrir principalmente en dos situaciones: formaciones circulando a alta velocidad en las curvas o conmutaciones en la máquina de cambios mientras una formación circula sobre el cambio de vías. Para evitar este último escenario, el sistema de enclavamientos implementa un bloqueo de la máquina de cambios por ocupación, tal como se ilustra en la Figura \ref{fig:ACG_ocupacion}.

    \begin{figure}[!h]
        \centering
        \includegraphics[width=1\textwidth]{Figuras/ocupacion}
        \centering\caption{Bloqueo de máquina de cambios por ocupación de secciones adyacentes.}
        \label{fig:ACG_ocupacion}
    \end{figure}

	La funcionalidad implementada radica en inhibir la conmutación de la máquina de cambios si alguna de las secciones de vías próximas al cambio de vías se encuentra ocupada. De esta manera, se garantiza que la posición del cambio de vías se mantendrá al detectar una formación aproximándose y no se permitirá su conmutación hasta que la formación se encuentre completamente alejada una distancia de seguridad. 
\subsection{Requerimiento de rutas y bloqueo de cambios en ruta}

	\label{sec:function_2}
	
	Para evitar colisiones entre formaciones que circulan en sentido contrario, es necesario asegurarse que las rutas habilitadas no compartan secciones de vías o elementos ferroviarios entre sí. Para lograr esto, el sistema de enclavamientos bloquea la activación de rutas conflictivas, tal como se ilustra en la Figura \ref{fig:ACG_bloqueo}.

    \begin{figure}[!h]
        \centering
        \includegraphics[width=0.7\textwidth]{Figuras/bloqueo_rutas}
        \centering\caption{Bloqueo de rutas conflictivas.}
        \label{fig:ACG_bloqueo}
    \end{figure}

	En el ejemplo de la Figura \ref{fig:ACG_bloqueo} se puede ver una formación que circula de derecha a izquierda por la vía inferior, al tener una señal verde que la habilita. El bloqueo de rutas conflictivas se manifiesta al forzar el aspecto rojo de las señales que habilitan dichas rutas, y al inhibir cualquier cambio de aspecto posible. Las rutas conflictivas pueden compartir todo el trayecto con la ruta principal, como la señal roja de la vía inferior; pero también pueden ser señales de rutas convergentes como la señal roja de la vía superior. De esta manera, se disminuyen las chances de colisiones frontales o laterales, respectivamente.
	
	
\subsection{Protección por aproximación en cancelación de ruta}

	\label{sec:function_3}
	
	La distancia de frenado es un aspecto esencial a considerar cuando una ruta en curso es cancelada. La ruta en cuestión debe seguir protegida durante un tiempo de seguridad.
	
	La Figura \ref{fig:ACG_aproximacion_1} introduce el caso de una formación que tenía una ruta aprobada (aspecto verde) que comenzaba en la sección violeta y abarcaba toda la sección naranja. Por seguridad, ambos cambios de vías fueron bloqueados y sus respectivas señales contrarias fueron forzadas a aspecto rojo y bloqueadas.

    \begin{figure}[!h]
        \centering
        \includegraphics[width=1\textwidth]{Figuras/aproximacion_1}
        \centering\caption{Formación aproximándose al inicio de la ruta.}
        \label{fig:ACG_aproximacion_1}
    \end{figure}
    
    Mientras la formación se encuentra en movimiento, el operador solicitó la cancelación de la ruta, cambiando el aspecto de la señal a rojo, tal como se visualiza en la Figura \ref{fig:ACG_aproximacion_2}. La formación quizás no tenga el tiempo ni la distancia suficiente para detenerse antes de la señal, por lo que se presentan dos escenarios. 
    
    \begin{figure}[!h]
        \centering
        \includegraphics[width=1\textwidth]{Figuras/aproximacion_2}
        \centering\caption{La ruta es cancelada mietnras la formación se aproxima.}
        \label{fig:ACG_aproximacion_2}
    \end{figure}
    
    El primer escenario es el mas favorable: la formación se detiene previo a la señal de peligro y se inicia un contador. Al cumplirse el tiempo de seguridad, las secciones asociadas a la ruta cancelada son liberadas. Lo mismo sucede con los cambios de vías y las señales conflictivas. Este tiempo otorgado permite comprobar que la formación efectivamente se detuvo antes de proceder con la liberación de los elementos ferroviarios para que puedan ser utilizados por otra ruta.
    
    \begin{figure}[!h]
        \centering
        \includegraphics[width=1\textwidth]{Figuras/aproximacion_3}
        \centering\caption{La formación se detiene exitosamente previo a la señal de peligro.}
        \label{fig:ACG_aproximacion_3}
    \end{figure}
    
	En el segundo escenario, la formación no logra detenerse previo a la señal de peligro, tal como se ilustra en la Figura \ref{fig:ACG_aproximacion_4}. Entonces, el sistema de enclavamiento no solamente no libera las secciones pertenecientes a la ruta cancelada, sino que también bloquea las próximas secciones, al no poder estimar cual será la distancia final de frenado de la formación.

    \begin{figure}[!h]
        \centering
        \includegraphics[width=1\textwidth]{Figuras/aproximacion_4}
        \centering\caption{La formación no se detiene previo a la señal de peligro.}
        \label{fig:ACG_aproximacion_4}
    \end{figure}
    
	Las secciones y elementos ferroviarios próximos se mantienen protegidos y enclavados hasta que la ruta se concluya, aún habiendo sido cancelada. Al finalizar la ruta, el sistema de enclavamiento liberará las secciones y elementos ferroviarios próximos al comprobarse que la formación se detuvo previo a la señal de finalización de la ruta.
	
\subsection{Protección por solape}

	Si una formación no detiene su marcha antes de una señal de peligro, el sistema de enclavamiento debe bloquear las secciones pertenecientes a esa ruta y la próxima, junto con la infraestructura asociado. La Figura \ref{fig:ACG_solape_1} ilustra este suceso, donde una formación ingresa a la sección violeta pasando una señal a peligro, sin tener la autorización requerida. A diferencia de la protección por aproximación, donde una formación no logra detenerse antes de ingresar a una ruta cancelada con poca anticipación, la protección por solape se ocupa de proteger la infraestructura en el caso de que la formación ingrese a una ruta que jamás fue habilitada.

    \begin{figure}[!h]
        \centering
        \includegraphics[width=1\textwidth]{Figuras/solape}
        \centering\caption{Formación ignora señal a peligro y se activa la protección por solape.}
        \label{fig:ACG_solape_1}
    \end{figure}
    
    Automáticamente, las secciones de la próxima ruta (coloreadas en naranja) son bloqueadas, a la vez que los cambios de vías cercanos y todas las señales tanto consecutivas como contrarias o convergentes. El bloqueo se removerá una vez que la formación se detenga en la próxima señal a peligro, luego de un tiempo de seguridad.
\subsection{Doble recubrimiento}

	Para evitar que una formación colisiones con una hipotética próxima formación que se encuentre detenida o circulando a menor velocidad, el sistema de enclavamiento deberá controlar las señales entre ambas para regular la velocidad y distancia entre ellas. Tal como se explicó en la Sección \ref{sec:signals}, las señales pueden presentar diferentes aspectos. Cada aspecto determinará un rango de velocidad permitido, siendo rojo el mas restrictivo. La Figura \ref{fig:ACG_recrubrimiento_1} ilustra el comportamiento del señalamiento cuando dos formaciones circulan en el mismo sentido, separadas por una distancia de seguridad.
	
	\begin{figure}[!h]
		\centering
		\includegraphics[width=1\textwidth]{Figuras/recubrimiento}
		\centering\caption{Protección por doble recubrimiento.}
		\label{fig:ACG_recrubrimiento_1}
	\end{figure}
	
	Debido al bloqueo por ocupación, todas las secciones ocupadas por una formación presentan una señal a peligro (roja). Inmediatamente detrás de cada formación se genera una secuencia de señales denominada doble recubrimiento. La cantidad de señales y la secuencia de aspectos variará según el operador de la red, las normas locales o nacionales. Algunos países utilizan la secuencia rojo-doble amarillo-amarillo-verde y será la que el ACG implementará. Por cuestiones de representación, la Figura \ref{fig:ACG_recrubrimiento_1} reemplazo la señal doble amarilla por una señal naranja.
	
	La formación que circule por detrás se encuentra frente a una señala de aspecto verde, por lo que puede continuar su marcha a la velocidad actual. No obstante, si su velocidad fuese mayor a la de la próxima formación, podría reducir la distancia entre ambas y pasar la señal de aspecto amarillo. Si esto sucediera, la formación deberá disminuir su velocidad para volver a situarse dentro de una sección verde. 
	
	Si la formación continúa teniendo una mayor velocidad que su par, la distancia entre ambas se reducirá y las señales permitirán velocidades mas o mas reducidas, hasta que la distancia se incremente a un valor seguro.
\subsection{Liberación secuencial}

	\label{sec:ACG_liberacion}
	
	Las rutas conflictivas no pueden ser habilitadas a la vez, pero existen algunas rutas que son solo parcialmente conflictivas, porque comparten una parte de la infraestructura y no toda. La implementación de la liberación secuencial aumenta la flexibilidad en la asignación y habilitación de rutas, mejorando la logística permitida por el sistema de enclavamientos. En la Figura \ref{fig:ACG_secuencial_1} se ilustra una formación que iniciará una ruta ya habilitada, para lo cual ya han sido bloqueadas las secciones (coloreadas en naranja) y la infraestructura (coloreadas en rojo).
	
	 \begin{figure}[!h]
	     \centering
	     \includegraphics[width=1\textwidth]{Figuras/secuencial_1}
	     \centering\caption{Formación iniciando una ruta ferroviaria.}
	     \label{fig:ACG_secuencial_1}
	 \end{figure}
 
	Al ocupar las secciones de vías, debido al bloqueo por ocupación, la señal de inicio de la ruta pasa a peligro y se bloquea la sección consecutiva a la ruta (coloreado en naranja), debido a la protección por solape. Esto se ilustra en la Figura \ref{fig:ACG_secuencial_2}.
	
	\begin{figure}[!h]
    	 \centering
	     \includegraphics[width=1\textwidth]{Figuras/secuencial_2}
    	 \centering\caption{Formación activando el bloqueo por ocupación y el bloqueo por solape.}
    	 \label{fig:ACG_secuencial_2}
	\end{figure}
 
 	Una vez que la formación desocupa las secciones de vías asociadas al cambio de vías anterior, el sistema de enclavamientos libera inmediatamente toda la infraestructura asociada, como se ilustra en la Figura \ref{fig:ACG_secuencial_3}. A la vez, el sistema de enclavamientos debe esperar a que se cumpla el plazo de seguridad antes de liberar la infraestructura posterior al fin de la ruta. Solamente son liberadas las secciones y señales que ya no son conflictivas.
	   
	\begin{figure}[!h]
	  \centering
	  \includegraphics[width=1\textwidth]{Figuras/secuencial_3}
	  \centering\caption{Liberación secuencial de la infraestructura por detrás de la formación.}
	  \label{fig:ACG_secuencial_3}
	\end{figure}
 
 	Transcurrido el tiempo de seguridad, el sistema de enclavamientos libera las secciones, cambios de vías, señales y toda infraestructura posterior al fin de la ruta, tal como se ilustra en la Figura \ref{fig:ACG_secuencial_4}.

	 \begin{figure}[!h]
	     \centering
	     \includegraphics[width=1\textwidth]{Figuras/secuencial_4}
	     \centering\caption{Liberación secuencial de la infraestructura por delante de la formación.}
	     \label{fig:ACG_secuencial_4}
	 \end{figure}
	    
	Para cada sistema de enclavamientos generado, dependiendo de la infraestructura disponible, el ACG implementa las funciones de seguridad presentadas en la Sección \ref{sec:function_1}, Sección \ref{sec:function_2}, Sección \ref{sec:function_3}, Sección \ref{sec:function_4}, Sección \ref{sec:function_5} y Sección \ref{sec:ACG_liberacion}.
	
	En las siguientes secciones se profundizará en la implementación de cada uno de los módulos del sistema y su comportamiento dinámico.