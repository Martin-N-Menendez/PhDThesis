\subsection{Módulo Detector}

El módulo \textit{detector} es el encargado de detectar el inicio y final de cada trama, validando que el contenido de la misma tenga N caracteres, conforme al formato de trama expuesto en la Sección \ref{sec:UART}. A medida que la validación tiene lugar, los caracteres ASCII son convertidos en valores booleanos (\textit{std\_logic}) dentro de un vector de elementos booleanos llamado \textit{packet}[N] (\textit{std\_logic\_vector} de N elementos). El diagrama de bloques de las máquinas de estado finitas con camino de datos se muestra en la Figura \ref{fig:Detector_module}.

\begin{figure}[H]
	\centering
	\includegraphics[width=1\textwidth]{Figuras/Detector_module.png}
	\centering\caption{FSMD del módulo \textit{Detector}.}
	\label{fig:Detector_module}
\end{figure}

En cada pulso de reloj (\textit{clk\_i}), el módulo UART envía un caracter por medio de la señal \textit{r\_data} (8 bytes) y un pulso (\textit{r\_available}) para informar que un nuevo dato ha sido enviado. El pulso de reloj es utilizado principalmente en el módulo \textit{Counter\_0\_to\_N}, cuyo parámetro N ya ha sido calculado por el ACG previo a generar el código y es la cantidad de caracteres que serán enviados a continuación del caracter de inicio. El proceso de detección y validación de la trama recibida se describe el diagrama de estados de la Figura \ref{fig:Detector_FSMD}.

\begin{figure}[H]
	\centering
	\includegraphics[width=0.8\textwidth]{Figuras/Detector_FSMD.png}
	\centering\caption{Diagrama de estados del módulo \textit{Detector}.}
	\label{fig:Detector_FSMD}
\end{figure}

El módulo \textit{Detector} inicia por detecto en el estado \textit{start}, aguardando por el caracter de inicio de trama '$<$'. Al recibir el caracter de inicio de trama, el módulo transiciona al estado \textit{reading}. En el estado \textit{reading} se recibirán solamente los caracteres ASCII '0' y '1'. Si al terminar de recibir N caracteres, el próximo caracter no es el de fin de trama '$>$' entonces se transiciona al estado \textit{error}, se reinician las variables auxiliares, la trama se descarta y se vuelve al estado \textit{start}.

Si el próximo caracter luego de leer N valores ASCII '0' y '1' es el caracter de fin de trama '$>$', entonces el módulo transiciona al estado \textit{final}, donde se da por válida la trama y se habilita su envío al módulo \textit{decoder}, para volver al estado \textit{start} a la espera de un nuevl caracter de inicio de trama, reiniciando todas las variables auxiliares.

Internamente se tienen diversas variables auxiliares para controlar si se han recibido los delimitadores y si la cantidad recibida es correcta. Eso cobra gran importancia al realizar los ensayos, porque se puede diferenciar rápidamente la fuente de posibles errores.