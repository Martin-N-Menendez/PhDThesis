\subsection{Modulo de comunicación}

	Si consideramos la lista de elementos dinámicos y cada estado que pueden admitir, es claro que la cantidad de señales sobrepasaría por mucho la limitada cantidad de puertos que una FPGA pueda proveer. Es por eso que se decidió que la información deberá ser recibida por la FPGA y transmitida desde la FPGA en formato serie. El formato serie es ampliamente utilizado en las redes ferroviarias, por ejemplo RS-485 o MVB en las redes de comunicación de trenes  TCN [REF]. Esta comunicación deberá ser flexible para ser utilizada en diferentes implementaciones con menor o mayor demanda de recursos.
	
	En la Figura \ref{fig:GeneralCom} se presenta la propuesta de conexión de la FPGA con una computadora externa, junto con los módulos internos de comunicación.	La UART (del inglés \textit{Universal Asynchronous Receiver-Transmitter}) es la unidad encargada de recibir y transmitir las tramas de datos entre la FPGA y la computadora.
	
	\begin{figure}[H]
		\centering
		\includegraphics[width=1\textwidth]{example-image}
		\centering\caption{FPGA.}
		\label{fig:GeneralCom}
	\end{figure}
	
	El bloque de recepción UART es el encargado de procesar las tramas con un baudrate preestablecido y almacenar el valor de la trama en la FIFO de entrada. La trama será enviada al sistema de enclavamientos, junto con una serie de pulsos para indicar cuándo deben ser leídos. El sistema de enclavamientos la procesará y devolverá una nueva trama a la FIFO de salida. Finalmente la nueva trama será enviada a la UART de tramisión que enviará la información con el mismo baudrate que fue recibido.
		
	En la Figura \ref{fig:Stream} se ilustra el formato definido para las tramas de entrada y salida. La trama tendrá un tamaño de entrada N y de salida M, con N menor que M, debido a que el estado de ocupación de las vías es de sólo lectura. Además, la trama tendrá un caracter delimitador de entrada y de salida ($<$ y $>$ respectivamente). Todos los elementos de la trama serán en formato ASCII, para poder ser interpretados fácilmente en una terminal y ser menos susceptibles a errores por alteraciones en algún bit aleatorio.
	
	\begin{figure}[H]
		\centering
		\includegraphics[width=1\textwidth]{Figuras/Tramas.png}
		\centering\caption{Tramas de entrada y salida.}
		\label{fig:Stream}
	\end{figure}
	
	De la lista de estados de cada elemento dinámico dado anteriormente se deduce que cada elemento requiere un sólo caractér para definir su estado, salvo los semáforos y cambios de vías dobles, que necesitan dos caracteres. Por lo tanto, el largo de la trama de entrada y de salida queda definido por la Fórmula \ref{eq:StreamLength_in} y la Fórmula \ref{eq:StreamLength_out}, respectivamente.
	
	\begin{equation} 
		\label{eq:StreamLength_in}
		\text{N} = 1\text{bit} (\text{N}_{NET}+\text{N}_{\text{RTS}}+\text{N}_{\text{LCB}}+\text{N}_{\text{SSW}}+\text{N}_{\text{SCR}})+2\text{bits} (\text{N}_{\text{SIG}}+\text{N}_{\text{DSW}})
	\end{equation}
	
	\begin{equation} 
		\label{eq:StreamLength_out}
		\text{M} = 1\text{bit} (\text{N}_{\text{RTS}}+\text{N}_{\text{LCB}}+\text{N}_{\text{SSW}}+\text{N}_{\text{SCR}})+2\text{bits} (\text{N}_{\text{SIG}}+\text{N}_{\text{DSW}})
	\end{equation}
	
	La implementación de los módulos de transmisión y recepción de la UART es invariante para cada locación, es decir, los recursos asignados serán los mismos, cualquiera sea el tamaño del sistema a implementar. Los módulos de memorias FIFO, en cambio, dependen de las características y del tamaño del sistema. Locaciones mas complejas tendrán valores de N y M mayores y, por lo tanto, requerirán FIFOs mas grandes. 
	
	Con este criterio de diseño, en todos los demás casos, la FIFO de salida tendrá el mismo tamaño que la FIFO de entrada o a lo sumo será 50 \% menor, lo que representa un ahorro de 25 \% de los recursos estimados. Por ejemplo, si se necesita que la entrada tenga 15 bits y la salida 7 bits y se le asignara el mismo tamaño a ambas FIFOs; tanto la FIFO de entrada como la de salida necesitarán 16 bits cada una, dando un total de 32 bits. Pero si se aplica el criterio de tamaños desacoplados, entonces para la FIFO de salida podrían asignarse solamente 8 bits,
	dando un total de 24 bits, un 25 \% menos que los 32 bits que necesitaría si ambas FIFOs quedaran definidas según los datos de la entrada.
	
\subsubsection{Módulo Detector}

\lipsum[1]

El módulo detector tiene como función recibir una secuencia de caracteres y armar una salida con un vector de elementos booleanos. Un diagrama en bloques del funcionamiento del módulo se muestra en la figura 3.9

\begin{figure}[H]
	\centering
	\includegraphics[width=1\textwidth]{example-image}
	\centering\caption{FPGA.}
	\label{fig:XXX}
\end{figure}

La UART envía secuencialmente un caracter por medio de la señal r\_data (8 bytes) y un pulso (r\_disponible) para informar que un nuevo dato ha sido enviado, además de indicar por medio de la señal N la cantidad de caracteres que serán
enviados.

El proceso de detección se ilustra en la figura 3.10.

\begin{figure}[H]
	\centering
	\includegraphics[width=1\textwidth]{example-image}
	\centering\caption{FPGA.}
	\label{fig:XXX}
\end{figure}

En la figura 3.10 se tiene un estado inicial en el cual se espera el caracter de inicio de la trama ("<") que provoca una transición al estado de lectura. En dicho estado se recibirán hasta N caracteres mientras se actualiza un contador interno. Cuando el contador interno iguale la cantidad N, se verifica si el próximo caracter es el de
fin de trama (">"). 

Si el caracter leído es el de final de trama, se pasa al estado final, donde el paquete es considerado válido y enviado a la próxima etapa junto con su pulso de validación del dato. Si el caracter leído es distinto, entonces se descarta toda la trama y se vuelve al inicio a la espera de otro caracter de inicio de trama, reiniciando
todas las variables auxiliares.

Internamente se tienen diversas variables auxiliares para controlar si se han recibido los delimitadores y si la cantidad recibida es correcta. Eso cobra gran importancia al realizar los ensayos, porque se puede diferenciar rápidamente la fuente de posibles errores.
\subsection{Módulo Decoder}

El módulo \textit{Decoder} es el encargado de demultiplexar la trama \textit{packet}[N] ya validada por el módulo \textit{Detector}. El módulo \textit{Decoder} recibe el vector de elementos booleanos \textit{packet}[N] y la señal \textit{process} que indica cuando puede iniciar el proceso de demultiplexación. La salida serán todos los vectores de estado de los elementos ferroviarios, si existen. El diagrama de bloques de las máquinas de estado finitas con camino de datos se muestra en la Figura \ref{fig:Decoder_module}.

\begin{figure}[H]
	\centering
	\includegraphics[width=1\textwidth]{Figuras/Decoder_module.png}
	\centering\caption{FSMD del módulo \textit{Decoder}.}
	\label{fig:Decoder_module}
\end{figure}

Esta demultiplexación no es ni equitativa para todos los vectores de salida, ni tampoco existirán todos los vectores de salida. La porción de \textit{packet}[N] correspondiente a cada vector será en función de la cantidad de elementos de cada tipo presentes en la locación. Esto ya fue calculado previamente por el ACG y explicado en la Sección \ref{sec:UART} al definir el formato de la trama. Si la cantidad de un cierto elemento ferroviario es mayor que uno, el ACG implementará el estado de ese elemento con un \textit{std\_logic\_vector} del tamaño adecuado. Si solo existe un elemento ferroviario de ese tipo, el ACG implementará un \textit{std\_logic}. Si no existiese ningún elemento ferroviario en la locación, el ACG no implementará ninguna de las funcionalidades relativas a dicho elemento, optimizando el uso de recursos en la FPGA.
\subsubsection{Módulo Encoder}

\lipsum[1]
\subsection{Módulo Printer}
	\label{sec:printer}
	
	El módulo Printer (ver Figura \ref{fig:GeneralSystem}) realiza la conversión de cada elemento de un vector de M elementos hexadecimales (\textit{packet}[M], M elementos de 4 bits) en caracteres hexadecimales (1 byte). Cada elemento del vector es analizado en cada ciclo de reloj (clk\_i) y demultiplexado, de manera tal de convertir un elemento por vez, para luego enviar el byte correspondiente al módulo UART para su posterior transmisión al exterior. El diagrama de bloques de la máquina de estados finitos con camino de datos se muestra en la Figura \ref{fig:Printer_module}.
	
	\begin{figure}[H]
		\centering
		\includegraphics[width=1\textwidth]{Figuras/Printer_module.png}
		\centering\caption{FSMD del módulo \textit{Printer}.}
		\label{fig:Printer_module}
	\end{figure}
	
	En cada ciclo de reloj el módulo \textit{Printer} demultiplexa el vector \textit{packet}[M] para obtener un elemento lógico que procesar, según el valor del contador vigente, que se incrementa en cada ciclo, hasta un máximo de M-1. Si el elemento \textit{packet}[i] es un valor hexadecimal, se enviará un byte equivalente en ASCII. Por ejemplo, se enviará un byte equivalente al 'A' ASCII si el elemento \textit{packet}[i] es una 'A' hexadecimal.
	
	Cada dos ciclos de reloj el módulo \textit{Printer} genera un pulso para habilitar el envío del último byte generado. Junto con el caracter se envía la señal \textit{wr\_uart} para indicarle a la UART que ese dato debe ser guardado en la FIFO de salida y la señal \textit{processed} para indicarle al módulo \textit{Detector} que se pueden procesar nuevas tramas. El ciclo de procesamiento de la trama a transmitir se describe el diagrama de estados de la Figura \ref{fig:Printer_FSMD}.
	
	\begin{figure}[H]
		\centering
		\includegraphics[width=0.8\textwidth]{Figuras/Printer_FSMD.png}
		\centering\caption{Diagrama de estados del módulo \textit{Printer}.}
		\label{fig:Printer_FSMD}
	\end{figure}
	
	El módulo \textit{Printer} inicia por detecto en el estado \textit{restart}, a la espera de recibir la señal \textit{process} del módulo \textit{Encoder}. Se tienen dos estados (\textit{cycle\_1} y \textit{cycle\_2}) para generar el pulso de reloj necesario para mantener sincronizadas las tramas. Cuando el contador haya recorrido los M elementos de \textit{packet}[M], el módulo vuelve al estado \textit{restart}, para esperar una nueva señal \textit{process} para volver a procesar una nueva trama de datos.
	
	Si la trama recibida es incorrecta, o si ya fue impresa, entonces la señal \textit{process} será '0' y el modulo \textit{Printer} dejará de enviar datos a la UART. Si la señal \textit{process} mantiene un estado lógico positivo, el proceso de impresión continuará hasta que la UART indique que no pueda recibir mas datos o que alguna etapa previa informe de algún error en el proceso.
\subsection{Módulo Selector}
	\label{sec:selector}
	
	Se añadió el módulo \textit{Selector} (ver Figura \ref{fig:GeneralSystem}) para poder facilitar el testeo de la comunicación serial al permitir anular la totalidad del sistema de enclavamiento. De esta manera, es posible validar la lectura, detección y escritura de tramas en bucle en forma independiente al sistema de enclavamiento. Esta funcionalidad es habilitada cambiando la posición de un switch físico de la FPGA y se desactiva invirtiendo su posición. El diagrama de bloques de la máquina de estados finitos con camino de datos diseñado para lograr este objectivo se muestra en la Figura \ref{fig:Selector_module}.
	
	\begin{figure}[H]
		\centering
		\includegraphics[width=0.55\textwidth]{Figuras/Selector_module.png}
		\centering\caption{FSMD del módulo \textit{Selector}.}
		\label{fig:Selector_module}
	\end{figure}