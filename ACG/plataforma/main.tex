\section{Consideraciones respecto a la implementación en una FPGA del código generado por el ACG}
	\label{sec:plataforma}
		
	%La elección de una FPGA por sobre un microprocesador se debe a las ventajas expuestas en la Sección \ref{sec:FPGA} respecto a la concurrencia del sistema, mayor nivel de seguridad y facilidad para redundar el sistema. Además, los sistemas implementados en FPGA son considerados puramente hardware y no software, por lo que solamente deben cumplir los requerimientos de la norma EN 50129, especialmente el anexo F referido a FPGAs, y no la norma EN 50128 que se encarga del software. Los alcances de estas normas ya fueron explicados en la Sección \ref{sec:normas}.
	
	El código VHDL generado por el ACG es independiente de la plataforma que se utilice. Es decir, cualquier familia de FPGAs es compatible con el sistema de enclavamientos que el ACG puede generar, siempre que se consideren ciertos detalles. En primer lugar, debido a su diseño con comunicación serial, todas las implementaciones utilizarán la misma cantidad de pines. En segundo lugar, se deberá dimensionar la FPGA a utilizar en base a la cantidad de Look-Up-Tables (LUTs) y Flip-Flops (FFs) que la implementación requiera. A modo de referencia, la Tabla \ref{Tab:tabla_ACG_total} ilustra estos parámetros para nueve ejemplos que serán analizados en la Sección \ref{sec:ejemplo_1} y los apéndices. Finalmente, al ser un diseño basado exclusivamente en hardware, el mismo será determinista, concurrente e incluso es factible añadir una capa adicional de software de alto nivel, no cubierta en este trabajo, que podría usarse por ejemplo para implementar la comunicación por MVB o MQTT.