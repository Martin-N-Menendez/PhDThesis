\section{Principios de señalamiento ferroviario}
    \label{sec:principios}
    
    El proceso de diseño del señalamiento requiere reglas claras y bien definidas sobre cuántas señales colocar, dónde colocar cada señal, bajo que condiciones, de qué tipo deben ser las señales, cómo deben orientarse, etc. Lamentablemente, el criterio utilizado a la hora de definir el señalamiento dista de ser uniforme de nación a nación. La obligatoriedad de ciertas señales, la protección de ciertos elementos, el granularidad de la red o incluso el tener las reglas por escrito son factores que cambian al atravesar las fronteras de cada país. Esto implica, claramente, una barrera enorme al tratar de integrar las redes ferroviarias transnacionales, como en el caso Europeo durante formación de la Unión Europea[REF].
    
    Para la realización de este trabajo se optó por recurrir al Instituto de Ingenieros en Señalamiento Ferroviario (IRSE, por sus siglas en inglés) [IRSE] y a la Junta de Normas y Seguridad de la Industria Ferroviaria (RISSB, por sus siglas en inglés) [RISSB]. Los reglamentos de diseño y definiciones de estas organizaciones son aceptadas por una gran cantidad de empresas del sector ferroviario y autoridades de gran peso. Entre ellas, por la agencia de transporte de Nueva Gales del Sur, en Australia (TfNSW, por sus siglás en inglés) [TfNSW]. De ésta última se recopilaron los siguientes principios de diseño ferroviario:
    
    \begin{itemize}
        \item [($P_1$)] Principio de autoridad: el derecho del tren a circular esta limitada a una pequeña porción de la infraestructura.
        \item [($P_2$)] Principio de claridad: las autoridades otorgadas o negadas no deben ser ambiguas. 
        \item [($P_3$)] Principio de anticipación: los conductores de trenes deben ser advertidos de los peligros cercanos con el suficiente tiempo para reaccionar.
        \item [($P_4$)] Principio de granularidad: las rutas deben ser lo mas cortas posibles.
        \item [($P_5$)] Principio de terminalidad: los conductores de trenes deben ser advertidos cuando se encuentren circulando próximos al final de la red.
        \item [($P_6$)] Principio de infraestructura: los conductores de trenes deben ser advertidos de cualquier infraestructura dinámica o estática próxima.
        \item [($P_7$)] Principio de no bloqueo: se debe evitar en todo momento que los trenes bloqueen el acceso a la infraestructura o ramificaciones a otros tresnes, de ser posible.
    \end{itemize}

    La totalidad del análisis realizado en este proyecto se basa en los principios expuestos. Se puede deducir de los mismos que un sistema de señalamiento debe proteger cada elemento ferroviario ($P_1$,$P_5$,$P_6$), en cada dirección posible ($P_3$,$P_7$). Por lo tanto debemos considerar la posibilidad de que cada tramo de vía puede ser transitado en ambos sentidos ($P_2$,$P_4$) . Esto implica a su vez considerar todas las rutas posibles soportadas por la red ferroviaria, y no solamente las necesarias desde un punto de vista logístico.