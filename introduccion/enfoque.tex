\subsection{Enfoque aplicado}

    Durante la Maestría en Sistemas Embebidos se desarrollo un primer prototipo del ACG en base a un rudimentario modelo de redes de grafos [REF]. Fue este modelo de grafos el que permitió probar el ACG con una amplia variedad de topologías, debido a la flexibilidad, linealidad y escalabilidad del modelo. En esta etapa temprana del proyecto ya se había adoptado un enfoque geográfico.

    La única desventaja de la elección temprana de un enfoque geográfico fue la, aparente, inexistencia de herramientas o soportes formales, para lo cual era necesario desarrollar todo desde cero sin un estándar que lo sostenga. Aún así, una vez desarrollada la herramienta se podía reutilizar las veces necesarias, por lo que las ventajas a largo plazo eran mayores. Al descubrir la existencia de railML, rápidamente se comenzó la migración del ACG para ser compatible con el estándar.

    El desarrollo del RNA, siendo posterior al ACG y a la incorporación de railML, fue completamente en línea con el estándar desde un inicio. Al ser compatibles con railML, tanto el RNA como el ACG deben ser compatibles también con el modelo planteado por RailTopomodel. Debido a esto, el desarrollo de la herramienta deberá seguir indefectiblemente los lineamientos de un enfoque geográfico. 