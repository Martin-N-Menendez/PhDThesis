\subsection{Enfoque aplicado}

    Durante la Maestría en Sistemas Embebidos se desarrollo un primer prototipo del ACG en base a un rudimentario modelo de redes de grafos \cite{Paper_206}. Fue este modelo de grafos el que permitió probar el ACG con una amplia variedad de topologías, debido a la flexibilidad, linealidad y escalabilidad del modelo \cite{Paper_109,Paper_149,Paper_150}. En esta etapa temprana del proyecto ya se había adoptado un enfoque geográfico.

    La única desventaja de la elección temprana de un enfoque geográfico fue la, aparente, inexistencia de herramientas o soportes formales, para lo cual era necesario desarrollar todo desde cero sin un estándar que lo sostenga. Aún así, una vez desarrollada la herramienta se podía reutilizar innumerables veces, por lo que las ventajas a largo plazo eran mayores que realizar un diseño a mano, a medida de cada locación. Al descubrir la existencia de railML, rápidamente se comenzó la migración del ACG para ser compatible con el estándar.

    El desarrollo del RNA fue posterior al desarrollo del ACG, y fue completamente en línea con el estándar railML desde un inicio. Al ser compatibles con railML, tanto el RNA como el ACG deben ser compatibles también con el modelo planteado por RailTopomodel. Debido a esto, el desarrollo de la herramienta siguió indefectiblemente los lineamientos de un enfoque geográfico.
    
    %El AGG se desarrolló en las etapas finales del doctorado, aprovechando el enfoque geográfico y el gran nivel de detalle que obtiene el RNA de cada archivo railML. El objetivo fue obtener una interfaz que simplificase tanto la transmisión de los comandos, cómo la recepción y la representación de los estados del señalamiento y el enclavamiento.
    
    El AGG se desarrolló con el objetivo de obtener una interfaz que simplifique tanto la transmisión de los comandos, como la recepción y la representación de los estados del señalamiento y el enclavamiento.