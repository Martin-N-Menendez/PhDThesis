\subsection{Amenazas a la validez del sistema}
    \label{sec:validacion}
    Para asegurar la validez de nuestro enfoque, analizamos tanto la validez interna como la externa \cite{Paper_139,Paper_121,Paper_185}. En cuanto a la validez interna, es importante garantizar una relación causal entre el diseño proporcionado y la señalización generada sin que ningún otro factor externo afecte el resultado. Podemos aprovechar el hecho de que el modelo ferroviario que el RNA utiliza es un sistema lineal debido a que el RNA se basa en redes de grafos \cite{Paper_109,Paper_112,Paper_149,Paper_150,Paper_201} y las redes de grafos son sistemas lineales \cite{Paper_19,Paper_86,Paper_89,Paper_101,Paper_102,Paper_114,Paper_115,Paper_141,Paper_142,Paper_144,Paper_146,Paper_151,Paper_154,Paper_155,Paper_162,Paper_163,Paper_169,Paper_171,Paper_180}. De esta manera, las señales generadas para dos elementos ferroviarios A y B serían las mismas que las señales generadas para el elemento ferroviario A más las señales generadas para el elemento ferroviario B. Este análisis se puede extender a cualquier número de elementos ferroviarios diferentes. Esto puede provocar solapamientos de señales, por lo tanto, se añade la posibilidad de que los usuarios de la herramienta habiliten o deshabiliten la simplificación de las mismas, así como establecer la distancia mínima entre elementos para considerar si se superponen o no.

    Los usuarios pueden seleccionar qué elementos ferroviarios se analizarán. La elección de los elementos a analizar es extremadamente poderosa porque nos permite probar la generación de señales para cada elemento ferroviario individualmente para validar que se estén analizando correctamente. Una vez que esta validación se realiza para cada elemento y se confirma que no hay factores externos que afecten la señalización generada, podemos seleccionarlos simultáneamente, sin simplificación. De esta manera, la linealidad del sistema se manifestará por sí misma y, en algunos casos, también la necesidad de simplificación de la señalización (ver Sección \ref{sec:limpieza}). Debido a la linealidad, los elementos ferroviarios que están muy cerca uno del otro se pueden considerar como un solo elemento, lo que resulta en una nueva señalización que tendría menos señales que la suma de sus señales anteriores debido a la simplificación de señales redundantes.

    Respecto a la demostración formal de los resultados, los estudios académicos (ver Sección \ref{sec:estudios}) suelen depender en su mayoría de métodos formales para verificar y validar sus diseños \cite{Paper_20,Paper_55,Paper_184,Paper_144,Paper_87,Paper_186,Paper_187,Paper_188,Paper_89,Paper_141,Paper_161,Paper_164,Paper_165,Paper_171,Paper_189,Paper_172,Paper_174,Paper_177,Paper_178}. Sin embargo, el RNA se basa en redes de grafos y, según la norma IEC 61508 \cite{Paper_63,Paper_77,Paper_78,Paper_79,Paper_80,Paper_81,Paper_82,Paper_83}, las redes de grafos son un método semi-formal. Además, casi la mitad de los estudios respaldados por las empresas mas importantes de la industria ferroviaria utilizan técnicas semi-formales \cite{Paper_44,Paper_59,Paper_60,Paper_88,Paper_94,Paper_95,Paper_123,Paper_196,Paper_145,Paper_147,Paper_148,Paper_173} para reducir la brecha entre la formalidad académica y las necesidades de la industria. Con el fin de realizar pruebas en sistemas ferroviarios reales, damos mayor énfasis a la validación de un enfoque más práctico y menos formal, de acuerdo con los requisitos y tendencias de la industria ferroviaria \cite{Paper_51,Paper_52,Paper_57,Paper_58,Paper_61,Paper_67,Paper_70,Paper_71,Paper_72,Paper_73,Paper_74,Paper_75,Paper_76,Paper_77,Paper_84,Paper_129,Paper_134,Paper_152,Paper_166,Paper_167}.
    
    Adicionalmente a la validación del resultado, es importante validar que el archivo railML generado sea sintácticamente correcto según los estándares de railML. El RNA realiza validaciones de sintaxis tanto al importar el archivo en formato railML como al generar uno nuevo. Esto se puede confirmar fácilmente importando el archivo railML en Design4Rail \cite{DESIGN4RAIL}, una herramienta certificada por railML.org. Esta herramienta realiza una validación de sintaxis de nuestros archivos railML generados y también se utiliza para visualizar los diseños ferroviarios generados automáticamente.

    En cuanto a la validez externa, la inclusión de nueve casos de estudio ferroviarios cuidadosamente seleccionados presentados en el Capítulo \ref{sec:resultados} tiene como objetivo cubrir una amplia gama de topologías y un uso extensivo de los elementos ferroviarios más comunes. Cuatro de estos ejemplos son diseños reales de ferrocarriles y los otros cinco se crearon para introducir topologías que se utilizan ampliamente en muchos países. Por lo tanto, cualquier otro diseño ferroviario compartiría la mayoría de las características y elementos modelados en nuestros ejemplos. Si se detecta un nuevo elemento en el diseño, se consideraría como un elemento a proteger y RNA generará la señalización adecuada de acuerdo con los principios P3 y P6 explicados en la Sección \ref{sec:principios}.

    En definitiva, nuestro proceso de validación implica una comparación automática entre las tablas de enclavamiento aprobadas por las autoridades ferroviarias y las tablas generadas por RNA. La ruta definida por RNA (o conjunto de rutas) debe tener una ruta correspondiente en el diseño original de señalización. Cualquier ruta no definida originalmente debe mejorar la seguridad general al proteger la infraestructura ferroviaria que no se consideró originalmente o mejorar la logística al agregar nuevas rutas no conflictivas. RNA también realiza una validación de sintaxis del archivo railML, indicando si alguna parte de la estructura XML está ausente o dañada. 
    
    Finalmente, validamos que todos los principios de señalización introducidos en este artículo sean cumplidos por el señalamiento generado por el RNA mediante un proceso de análisis posterior del diseño y la nueva señalización, asegurando que se cumpla cada principio de diseño ferroviario.