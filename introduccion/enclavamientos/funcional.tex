\subsection{Enfoque funcional}

    A la hora de abordar el análisis de las redes ferroviarias, hemos encontrado que, a grandes rasgos, existen dos estrategias muy diferenciadas: el enfoque funcional y el enfoque geográfico [REF], cada un con sus fortalezas y debilidades. Ambos enfoques se asemejan a la discusión de arquitecturas CISC (del inglés, Complex Instruction Set Computing) vs RISC (del inglés, Reduced Instruction Set Computing), donde el primero centraliza las decisiones en un único módulo y el segundo distribuye las decisiones en pequeños módulos de funciones acotadas.    

    En el enfoque funcional las decisiones se basan en la 'tabla de enclavamientos', que define cada ruta que puede ser solicitada por el operario. Encontramos entonces, el primer gran problema del enfoque funcional: como se ejemplificó en la Sección \ref{sec:tablas}, las soluciones dadas por una tabla de enclavamientos no son únicas, dependen del itinerario que se quiera establecer en base a la infraestructura. Ese itinerario puede variar con el tiempo, haciendo necesario añadir o eliminar algunas rutas, teniendo que volver a implementar y certificar todo el sistema. Además, muchas de las soluciones no contemplan todas las rutas posibles, por lo que el diseño es incompleto y siempre existirá el riesgo de tener que repetir todo el proceso desde cero.

    El segundo problema radica en que, al ser el enfoque funcional una arquitectura CISC que prioriza la funcionalidad a nivel macro, abstrayéndose de la topología de la red, la concurrencia de las rutas no está garantizada. Es decir, si N rutas dependen del estado de un elemento en común, no puede garantizarse ni que el resultado de cada ruta se calcule en paralelo, ni tampoco que el resultado de cada ruta se obtenga en simultáneo. Para solucionar esto, es necesario repetir N veces el elemento en cuestión, asociando uno a cada ruta que condiciona, incrementando la complejidad del diseño [REF]. 
    
    Claramente, ya que el enfoque funcional no garantiza la concurrencia del sistema, es necesario tomar medidas que terminan aumentando la cantidad de recursos necesarios. A medida que la complejidad de la red ferroviaria aumenta, incrementando a su vez la interrelación de sus elementos, la necesidad de mantener la concurrencia del sistema termina generando que la memoria utilizada crezca exponencialmente [REF].

    Por lo tanto, el enfoque funcional es de muy fácil implementación para topologías pequeñas y tradicionalmente se lo considera como el único enfoque por defecto. Sin embargo, presenta grandes falencias a la hora de resolver sistemas de mediana y alta complejidad. El enfoque funcional no posee concurrencia de forma directa, desperdicia mucha memoria y su solución muchas veces resulta incompleta. Un sistema de enclavamientos diseñado en base a un enfoque funcional difícilmente será mantenible en el tiempo ni mucho menos escalable o fácil de actualizar.