\subsection{Uso del estándar railML en la industria ferroviaria}

    El estándar railML es promovido por empresas de gran peso en la industria ferroviaria como Siemens, Thales, Alstom, CAF, ADIF y Toshiba, que concentran la mayoría de la cuota de mercado global \cite{PARTNERS}. Adicionalmente, diversas instituciones y organismos ferroviarios a nivel estatal y nacional hacen uso del estándar railML en sus desarrollos ferroviarios \cite{PARTNERS}, tales como: Queensland Rail, Transdev Deutschland, Transperth, Saudi Railway Company y Transport for New South Wales, entre otras.

    En sus primeros años de vida railML experimentó varios cambios, pero no fue hasta su versión 3.0 con enfoque geográfico que el uso del estándar creció exponencialmente. Entre el 40 y el 60\% de sus usuarios adoptaron el estándar en los últimos siete años \cite{PARTNERS}.

    Podemos encontrar las herramientas mas diversas basadas en railML: analizadores de infraestructura \cite{MAPREX}, planificador logístico para material rodante \cite{IVU}, visualizadores de datos \cite{RAILVIVID}, planificadores de infraestructura \cite{VISALL3D} y visualizadores/simuladores de infraestructura enclavamiento \cite{DESIGN4RAIL}. Muchas de ellas certificadas e intercompatibles entre sí. Aunque la mayoría son herramientas de código cerrado, el estándar railML es abierto y sigue un principio bottom-top: todas las necesidades de la industria son tenidas en cuenta para ser incorporadas en nuevas versiones del estándar, siguiendo el exitoso modelo del estándar USB, Bluetooth y GSM [BUSCAR REF]. 