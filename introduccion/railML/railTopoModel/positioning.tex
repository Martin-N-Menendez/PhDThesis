\subsubsection{Posicionamiento}

    RailTopoModel utiliza diferentes sistemas de posiciomaniento: intrínseco, geográfico y esquemático. Las coordenadas intrínsecas se encuentran en el rango 0 a 1, relativas a la posición dentro del netElement. Estas coordenadas son obligatorias, junto con el largo asociado al netElement, independientemente de si se definieron o no las demás coordenadas. Las coordenadas geográficas son coordenadas absolutas, por ejemplo de un sistemas GPS. Finalmente las coordenadas esquemáticas son relativas a la posición del elemento dentro del archivo que modele ese sistema, muchas veces utilizadas en herramientas de software para posicionar los elementos en una interfaz gráfica.
