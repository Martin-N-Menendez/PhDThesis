\subsection{RailTopoModel: modelo topológico de infraestructura ferroviaria basado en grafos}

    En 2016 la UIC (del inglés, International Union of Railways) publicó el Standard 30100 [REF] en el cual definió un formato de intercambio de datos ferroviarios llamado RailTopoModel. RailTopoModel es un modelo topológico de infraestructura ferroviaria basado en grafos. El modelo abarca tres tipos de niveles, la topología de la red, objetos materiales, objetos inmateriales y objetos lógicos, tal como se ilustra en la Figura \ref{fig:RTM_3}. 

    \begin{figure}[!h]
        \centering
        \includegraphics[width=1\textwidth]{Figuras/objetos}
        \centering\caption{Alcance del modelo de UIC RailTopoModel.}
        \label{fig:RTM_3}
    \end{figure}

    Cada uno de estos elementos posee propiedades y características propias, que pueden ser físicas o abstractas. Por ejemplo, los objetos materiales pueden ser adimensionales (señales, balizas, etc), unidimensionales (vías, plataformas, etc) o bidimensionales (estaciones, túneles, etc). Debido a como fue diseñado, RailTopoModel es el modelo ideal para implementar un enfoque geográfico.
    
\subsubsection{Paquete base}

    El estándar RailTopoModel se divide en cuatro paquetes: la base, la topología, el posicionamiento y la red de entidades. La base incluye toda la información de alto nivel de la infraestructura que son comúnes a toda la red \cite{Paper_146}. Por ejemplo, el sistema de alimentación eléctrico utilizado puede ser por tercer riel o catenarias y este no cambia a lo largo de toda la red.

\subsubsection{Modelo de grafos}
    \label{sec:RTM}
    
    Otros estudios han tratado de modelar el trazado ferroviario utilizando teoría de grafos en el pasado [REF]. Pero todos ellos definían a las vías como las aristas del grafo y los cambios de vías como los nodos, lo cual dificultaba enormemente la inclusión de otros elementos ferroviarios como las plataformas, los pasos a nivel o los semáforos. RailTopoModel se apega mas a la teoría clásica de grafos y define como nodos a la unidad mínima de recursos físicos, llamándolos netElements y a la conexión física entre ellos como aristas, llamándolos netRelations.

    Un netElement debe contener un tramo de vía en su totalidad o varios tramos, pero un tramo de vía no puede tener varios netElements asociados. Opcionalmente, un netElement puede tener, además cualquier otro elemento ferroviario asociado como plataformas, semáforos o balizas. Para entender mejor como se materializa este concepto se tiene la Figura \ref{fig:grafos_1} que ejemplifica el modelado en grafos de una maquina de cambios.

    \begin{figure}[!h]
        \centering
        \includegraphics[width=1\textwidth]{Figuras/grafos}
        \centering\caption{Modedelado en grafos de una máquina de cambios simple.}
        \label{fig:grafos_1}
    \end{figure}

    Se tiene una máquina de cambios simple que conecta una vía de circulación (horizontal) con una vía de maniobras (oblicua). El netElement asociado a la vía principal antes de la bifurcación lo llamaremos A. Este netElement es también al que se asocia al objeto de la máquina de cambios. El netElement asociado a la vía de maniobras lo denominamos B y al asociado a la vía de continuación de la cirlación lo denominamos C.

    En la representación del modelo de grafos, el netRelation 1 relaciona el netElement A y B de forma biyectiva. Es decir, A esta relacionado con B y B está relacionado con A. De la misma forma se asignan los netRelations 2 y 3. Sin embargo, no hay que confundir que dos vías estén conectadas con que un tren puede circular por ellas. Una formación podría circular por el tramo A-C, A-B, B-A y C-A, pero no podrá circular desde B hacia C sin pasar por A o viceversa. Es físicamente imposible que un tren realice ese movimiento de una sola maniobra. Para circular desde B hacia C primero deberá finalizar el tramo B-A, modificar la posición de la máquina de cambios y recorrer el tramo A-C. A la propiedad asignada a los netRelation que son físicamente transitables se la denomina navegabilidad y es una característica esencial de las redes de grafos que modelan redes ferroviarias. En este ejemplo, solo los netRelation 1 y 2 poseen navegabilidad.
\subsubsection{Nivel micro, meso y macro}

\lipsum[1]
\subsubsection{Posicionamiento}

    RailTopoModel utiliza diferentes sistemas de posiciomaniento: intrínseco, geográfico y esquemático. Las coordenadas intrínsecas se encuentran en el rango 0 a 1, relativas a la posición dentro del netElement. Estas coordenadas son obligatorias, junto con el largo asociado al netElement, independientemente de si se definieron o no las demás coordenadas. Las coordenadas geográficas son coordenadas absolutas, por ejemplo de un sistemas GPS. Finalmente las coordenadas esquemáticas son relativas a la posición del elemento dentro del archivo que modele ese sistema, muchas veces utilizadas en herramientas de software para posicionar los elementos en una interfaz gráfica.

