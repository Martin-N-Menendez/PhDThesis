\subsection{Empresas del sector ferroviario}

    Al requerir ingenieros y técnicos más especializados, el conocimiento requerido tanto para diseñar como para mantener y operar los sistemas de enclavamientos electrónicos se concentra en menos actores. La tendencia mundial en las últimas décadas ha sido el depender de soluciones cerradas provistas por alrededor de una docena de empresas, muchas veces incompatibles entre sí. Mas del 80\% del material rodante que circula en el mundo proviene de alguna de estas doce empresas \cite{MARKET}:

    \begin{itemize}
        \item CRRC Railway (China) \cite{CRRC}
        \item Alstom (Francia) \cite{ALSTOM}
        \item Bombardier Transportation (Canada) \cite{BOMBARDIER}
        \item CAF (España) \cite{CAF}
        \item Hitachi (Japón) \cite{HITACHI}
        \item Transmashholding (Rusia) \cite{TRANSMASHHOLDING}
        \item Hyundai Rotem (Corea del sur) \cite{HYUNDAI}
        \item Siemens Mobility (Alemania) \cite{SIEMENS}
        \item Thales (Reino unido) \cite{THALES}
        \item General Electic Transportation (EEUU) \cite{GENERAL}
        \item Caterpillar (EEUU) \cite{CATERPILLAR}
        \item Stadler Rail (Suiza) \cite{STADLER}
    \end{itemize}
    
    Al lado de cada empresa se puede ver su país de origen. Cuatro de esos países (China, EE.UU., Canadá y Rusia) son las naciones con mayor extensión territorial, para lo cual un sistema ferroviario robusto es esencial. El diseño de un sistema de enclavamiento se rige por el mismo estándar de calidad y seguridad que los sistemas aeroespaciales y nucleares: IEC 61508 \cite{Paper_77,Paper_78,Paper_79,Paper_80,Paper_81,Paper_82,Paper_83}.

    