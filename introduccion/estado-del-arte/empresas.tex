\subsection{Empresas del sector ferroviario}

    La evolución tecnológica presenta las ventajas ya discutidas respecto a reducción de costos y tamaños, además de permitir abordar problemas mas complejos y cubrir mas aspectos de seguridad que décadas atrás. Esto, obviamente, tiene un contraparte negativa: al requerir ingenieros y técnicos más especializados, el conocimiento requerido tanto para diseñar como para mantener y operar estos sistemas se concentra mas o mas en menos actores. La tendencia mundial en las últimas décadas ha sido el depender de soluciones cerradas provistas por alrededor de una docena de empresas, muchas veces incompatibles entre sí. Mas del 80\% del material rodante que circula en el mundo proviene de alguna de estas doce empresas:

    \begin{itemize}
        \item CRRC Railway (China)
        \item Alstom (Francia)
        \item Bombardier Transportation (Canada)
        \item CAF (España)
        \item Hitachi (Japón)
        \item Transmashholding (Rusia)
        \item Hyundai Rotem (Corea del sur)
        \item Siemens Mobility (Alemania)
        \item Thales (Reino unido)
        \item General Electic Transportation (EEUU)
        \item Caterpillar (EEUU)
        \item Stadler Rail (Suiza)
    \end{itemize}
    
    Todas las naciones listadas se encuentran entre las mas ricas y prósperas del mundo. Cuatro de ellas son las naciones con mayor extensión territorial, para lo cual un sistema ferroviario robusto es esencial. El diseño de un sistema de enclavamiento se rige por el mismo estándar de calidad y seguridad que los sistemas aeroespaciales y nucleares: IEC XXXXX [REF].

    