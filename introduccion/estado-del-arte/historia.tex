\subsection{Tipos de sistemas de enclavamiento}

    Lamentablemente, los sistemas de enclavamiento en Argentina son en su mayoría mecánicos, de comienzos del siglo XX, y otra cantidad considerable son electromecánicos, de más de 40 años de antigüedad. Muchos de ellos ya han agotado su vida útil y deben ser reemplazados. Otros, en cambio, han estado en desuso por años y necesitan ser repuestos, pero solo una docena de empresas en el mundo realizan el diseño del sistema y los costos para un bypass simple rondan las decenas de millones de dólares. Por esto, es importante contar con sistemas electrónicos de diseño y fabricación nacional.
    
    Además, existen diferentes lugares donde aún resta instalar este tipo de sistemas, por lo que su implementación constituye una necesidad real para el desarrollo de la infraestructura ferroviaria de señalamiento en Argentina.
    
    A comienzos del siglo XX, se implementaron los sistemas de enclavamiento mediante soluciones mecánicas. Utilizaban palancas, como las que se visualizan en la Figura \ref{fig:enclavamiento_2}, para comandar los cambios de vías y semáforos.
    
        \begin{figure}[h]
            \centering
            \includegraphics[width=0.5\textwidth]{example-image}
            \centering\caption{XXXXX.}
            \label{fig:enclavamiento_2}
        \end{figure}
    
    Una vez que se constituye una configuración de posiciones de palancas que habilitan un trayecto, estas quedan "enclavadas" mecánicamente, es decir, su posición se bloquea y no es físicamente posible cambiarla. A medida que se van moviendo ciertas palancas, las demás que pudieran representar situaciones no seguras quedan enclavadas, y solo se pueden mover aquellas cuyo accionamiento representa una situación segura. De esa manera, se garantiza que no se generarán configuraciones tales que las formaciones colisionen entre sí.
    
    Las tecnologías más modernas heredaron el término "enclavamiento", aunque ya no se tengan palancas enclavadas en posiciones fijas. En lugar de palancas físicas, se utilizan sistemas electrónicos y lógica programable para lograr el mismo objetivo de garantizar la seguridad ferroviaria.
    
    %----
    
    A mediados del siglo XX se desarrolló el sistema de enclavamiento electromecánico. Su funcionamiento se basa en relés (Figura \ref{fig:enclavamiento_3}) y circuitos de vía, de forma tal de poder detectar la presencia de un tren y comandar tanto las señales como las barreras de los pasos a nivel.
        
        \begin{figure}[h]
            \centering
            \includegraphics[width=0.5\textwidth]{example-image}
            \centering\caption{XXXXX.}
            \label{fig:enclavamiento_3}
        \end{figure} 
    
    Los sistemas de enclavamiento electromecánicos son comandados por un operario mediante un panel de control (Figura \ref{fig:enclavamiento_4}). El operario solicita al sistema de enclavamiento las rutas que el conductor ferroviario necesita para circular. El sistema permitirá solo la operación de cambio de vías seguras. En caso contrario, se tendrán las salidas "enclavadas" y el sistema de enclavamiento impedirá mediante los semáforos el avance de la formación hasta que pueda realizarse el cambio de forma segura.
    
        \begin{figure}[h]
            \centering
            \includegraphics[width=0.5\textwidth]{example-image}
            \centering\caption{XXXXX.}
            \label{fig:enclavamiento_4}
        \end{figure}
    
    %--- 
    HABLAR DE ENCLAVAMIENTO ELECTRONICO