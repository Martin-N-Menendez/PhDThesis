\subsection{Enfoque geográfico}

    En el enfoque geográfico, el énfasis está puesto en la interacción entre los componentes a partir de su posición en la red y no en su funcionalidad a nivel de sistema. Por ende, el enfoque geográfico no necesita una tabla de enclavamientos que defina su comportamiento, sino definir genéricamente los componentes y establecer una representación matemática de las conexiones entre ellos. 
    
    En ese sentido, este enfoque requiere un nivel de análisis previo mucho mayor y, por lo tanto, un mayor esfuerzo de desarrollo. No obstante, es mucho mas escalable a topologías de cualquier tamaño al hacer un uso mas eficiente de los recursos al tener concurrencia directa. Además, es posible añadir nuevos componentes en el futuro, haciéndolo mucho mas flexible de ser aplicado en otros países con diferentes elementos ferroviarios. Este enfoque independiza la estrategia de diseño de la locación, haciendo posible replicar de forma sistemática el mismo conjunto de herramientas en cualquier topología.

    Aunque el concepto de ruta sigue existiendo, ya no es el foco central del proceso de diseño. La tabla de enclavamiento deja de ser una entrada del proceso y pasa a ser un subproducto. La herramienta que realice el análisis de la red ferroviaria deberá obtener todas las rutas posible que soporta esa topología y registrarlas en una tabla de enclavamientos. La tabla de enclavamientos del enfoque funcional debe estar contenida en la tabla de enclavamientos del enfoque geográfico, ambos enfoques deben ser consistentes.