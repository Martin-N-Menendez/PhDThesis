\subsection{Antecedentes de trabajos académicos en la temática}
\label{sec:estudios}
    Dada la variedad de topologías (algunas mencionadas en la Sección \ref{sec:topologias}), es deseable automatizar tanto el análisis de la red como el proceso de diseño del señalamiento. De esta manera, se puede reducir el error humano que puede afectar el proceso en cualquiera de sus etapas: análisis, diseño, implementación, verificación y validación.

    Varios artículos abordan la automatización de las tablas de enclavamiento \cite{Paper_2,Paper_45,Paper_162,Paper_170,Paper_182}, que son las tablas que indican cuáles elementos ferroviarios utilizan cada ruta que la formación puede solicitar al sistema. También podemos encontrar trabajos respecto de la automatización de otros elementos como el comportamiento de los enclavamientos \cite{Paper_21,Paper_90,Paper_93,Paper_99,Paper_198,Paper_158,Paper_197}, las rutas \cite{Paper_114}, el señalamiento de los pasos a nivel \cite{Paper_86}, topologías simples \cite{Paper_143,Paper_161}, así como sus simulaciones \cite{Paper_100,Paper_181}.

    Existen, además, diversos estudios acerca de la verificación y validación de los sistemas ferroviarios utilizando métodos formales \cite{Paper_7,Paper_10,Paper_14,Paper_85,Paper_87,Paper_89,Paper_91,Paper_113,Paper_144,Paper_186,Paper_187,Paper_188,Paper_141,Paper_161,Paper_164,Paper_165,Paper_171,Paper_184,Paper_189}. Algunos emplean herramientas que analizan y validan el modelo tales como NuSMV \cite{Paper_115,Paper_155,Paper_161,Paper_185}, SafeCap \cite{Paper_189,Paper_194,Paper_195}, o soluciones completas como el B-method \cite{Paper_190,Paper_191,Paper_192,Paper_193}. No obstante, luego de recopilar mas de 150 artículos \cite{BIBLIO}, no hemos encontrado una herramienta (o conjunto de herramientas) que realice el análisis de una red ferroviaria arbitraria, generando automáticamente su señalamiento completo y la implementación automática de su sistema de enclavamiento asociado. Los pocos casos que aventuran análisis automáticos lo realizan solo para topologías simples de un único elemento \cite{Paper_157,Paper_183}, mientras que en este trabajo se busca la generalización para cualquier red ferroviaria con cualquier combinación de diversos elementos ferroviarios.