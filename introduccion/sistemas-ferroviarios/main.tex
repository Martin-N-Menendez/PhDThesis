


\section{Sistemas ferroviarios}

    Las redes ferroviarias modernas presentan dos elementos fundamentales en su infraestructura: una topología y elementos ferroviarios. La topología es el entramado de vías férreas conectadas de forma arbitraria, cuyo diseño busca cumplir una función particular, interconectando diversos elementos ferroviarios. Estos elementos pueden ser para determinar la ubicación del tren, para delimitar la circulación de vehículos en cruces ferroviarios, permitir el ascenso y descenso de pasajeros, o para modificar dinámicamente los caminos por los que los trenes circulan, entre otras funciones.

    Cualquiera sea el elemento involucrado, altera las funciones de la red y su inminente proximidad debe informarse cuanto antes al conductor del tren. Este podrá decidir, de estar permitido, modificar o no su accionar antes de alcanzar dicho elemento. Es tarea del señalamiento ferroviario alertar a los conductores ferroviarios de cualquier elemento que pueda representar un peligro, evitando así colisiones con otras formaciones o descarrilamientos en zonas críticas. El señalamiento ferroviario incluye un elemento fundamental: los semáforos.

    Los semáforos (de ahora en mas denominados "señales"), constituyen el medio de comunicación primario entre los conductores y su entorno, informándoles de la habilitación o denegación de uso de las vías posteriores mediante su color, denominado aspecto. Cada señal puede presentar un único aspecto por vez de un conjunto posible que varía según el país o la región. Los aspectos utilizados en Argentina son el verde (permitido avanzar), amarillo (atención) y rojo (detenerse). Algunas señales pueden no tener el aspecto verde (señales de maniobras de dos aspectos) o incorporar un aspecto extra entre el rojo y el amarillo (señales de cuatro aspectos que incluyen el doble amarillo).
    
    Dos señales consecutivas con la misma dirección y sentido constituyen una ruta ferroviaria. Los operarios ferroviarios solicitan al sistema de enclavamientos las rutas que necesitan en base a la logística deseada. El sistema de enclavamientos habilitará o denegará las rutas solicitadas en función del estado de los elementos ferroviarios cercanos y de las demás rutas activas. Esta función es vital para el sistema ferroviario y su fin último es permitir la circulación de trenes de la forma mas segura o, de no ser posible, no permitir circulación alguna.

    El diseño de los sistemas de señalamiento es un proceso complejo que involucra, principalmente, tres etapas: el análisis de la red ferroviaria, la detección de zonas de mayor probabilidad de descarrilamientos y colisiones, y la óptima localización de las señales correctas para cada función [REF]. La generación automática del señalamiento es de gran importancia y utilidad para el desarrollo de redes ferroviarias nuevas o para revitalizar redes ferroviarias abandonadas o en desuso. Es relevante incluso en redes ferroviarias que son alteradas debido a la adición, modificación o eliminación de elementos ferroviarios tales como pasos a nivel o plataformas, lo cual implica que el señalamiento debo actualizarse en consecuencia.

\section{Principios de señalamiento ferroviario}
    \label{sec:principios}
    
    El proceso de diseño del señalamiento requiere reglas claras y bien definidas sobre cuántas señales colocar, dónde colocar cada señal, bajo que condiciones, de qué tipo deben ser las señales, cómo deben orientarse, etc. Lamentablemente, el criterio utilizado a la hora de definir el señalamiento dista de ser uniforme en los distintos países. La obligatoriedad de ciertas señales, la protección de ciertos elementos, la granularidad de la red o incluso el tener las reglas por escrito son factores que cambian al atravesar las fronteras de cada país. Esto implica, claramente, una barrera enorme al tratar de integrar las redes ferroviarias transnacionales, como en el caso Europeo durante formación de la Unión Europea \cite{Paper_4,Paper_185}.
    
    Para la realización de este trabajo se optó por recurrir al Instituto de Ingenieros en Señalamiento Ferroviario (IRSE, por sus siglas en inglés) \cite{IRSE} y a la Junta de Normas y Seguridad de la Industria Ferroviaria (RISSB, por sus siglas en inglés) \cite{Paper_175,Paper_176,Paper_203}. Los reglamentos de diseño y definiciones de estas organizaciones son aceptadas por una gran cantidad de empresas del sector ferroviario y autoridades de gran peso. Entre ellas, por la agencia de transporte de Nueva Gales del Sur, en Australia (TfNSW, por sus siglas en inglés) \cite{Paper_202}. De ésta última se recopilaron los siguientes principios de diseño ferroviario:
    
    \begin{itemize}
        \item [($P_1$)] Principio de autoridad: la combinación de todas las rutas alcanza todo el tendido ferroviario.
        \item [($P_2$)] Principio de claridad: las autoridades otorgadas o negadas no deben ser ambiguas. 
        \item [($P_3$)] Principio de anticipación: los conductores de trenes deben ser advertidos de los peligros cercanos con el suficiente tiempo para poder reaccionar.
        \item [($P_4$)] Principio de granularidad: las rutas habilitan el uso de una pequeña porción de la infraestructura.
        \item [($P_5$)] Principio de terminalidad: los conductores de trenes deben ser advertidos cuando se encuentren circulando próximos al final de la red.
        \item [($P_6$)] Principio de infraestructura: los conductores de trenes deben ser advertidos de cualquier infraestructura dinámica o estática próxima.
        \item [($P_7$)] Principio de no bloqueo: se debe evitar en todo momento que los trenes bloqueen el acceso a la infraestructura o ramificaciones a otros trenes, de ser posible.
    \end{itemize}

    La totalidad del análisis realizado en este trabajo se basa en los principios expuestos. Se puede deducir de los mismos que un sistema de señalamiento debe proteger cada elemento ferroviario ($P_1$,$P_5$,$P_6$), en cada dirección posible ($P_3$,$P_7$). Por lo tanto debemos considerar la posibilidad de que cada tramo de vía puede ser transitado en ambos sentidos ($P_2$,$P_4$). Esto implica a su vez considerar todas las rutas posibles soportadas por la red ferroviaria, y no solamente las necesarias desde un punto de vista logístico.
\subsection{Elementos ferroviarios}

El sistema ferroviario consta de diversos elementos que incluyen la infraestructura (tendido ferroviario, plataformas, cruces de vía), sensores (circuitos de vía, contadores de eje), actuadores (barreras ferroviarias, máquinas de cambios) e interfaces visuales con los conductores (semáforos ferroviarios). Todos estos elementos se interrelacionan y funcionan en conjunto dentro del sistema de señalamiento y el sistema de enclavamiento. Cada uno de estos elementos será descripto en la sucesivas secciones, en orden tal de integrar los conceptos anteriores de la forma mas clara posible.

\subsubsection{Vías}

Las vías férreas (Figura XX) son un elemento ferroviario esencial y son la columna vertebral de la infraestructura ferroviaria. Estas constituyen el sitio por el cual se desplazan los trenes, definiendo no solo la dirección del desplazamiento, sino también restringiendo el dominio del tren. Esto lo diferencia de otros medios de transporte como el automóvil que no necesitan un camino para circular y, aún teniendo una carretera, puede moverse libremente por fuera de esta.

Las vías se encuentran separadas por una distancia fija que se mide desde sus caras internas, denominada trocha. Solamente las formaciones compatibles con ese parámetro de trocha pueden circular por el tendido ferroviario. El valor de la trocha puede variar entre las denominadas trocha angosta (600 a 1372 mm, estandar imperial británico) y trocha ancha (1520 a 3000 mm, estándar ruso, indio, ibérico, irlandes). Se estableció el valor intermedio de 1425 mm como valor de trocha internacional, usado ampliamente en Europa, Norteamérica y Oceanía.

%%%

Las vías se dividen en secciones y por seguridad se establece que cada sección puede contener solo una formación por vez. Las mismas pueden tener largos variables en zonas urbanas de entre 500 a 1000 metros en zonas rurales. 

Cada vía puede ser clasificada en dos grupos: vías ascendentes o vías descendentes. Las ascendentes son aquellas por las cuales los trenes circulan únicamente en la dirección del kilometraje en sentido creciente. Las descendentes son aquellas por las cuales los trenes circulan únicamente en la dirección del kilometraje en sentido decreciente [REF]. El kilómetro 0 es la estación principal de la línea ferroviaria, como por ejemplo: Plaza Constitución (para la línea Roca), Once de septiembre (para la línea Sarmiento) o Retiro (para las líneas Mitre y San Martín). 

Existen vías de maniobra que pueden ser tanto ascendentes como descendentes. Estas vinculan, mediante un cambio de vías, una sección ascendente con otra descendente, en la cual los trenes deben circular a una velocidad reducida.

\lipsum[1]
\includegraphics{example-image}
\lipsum[1]
\subsubsection{Fin de vía y transiciones}

\lipsum[1]

\lipsum[1]
\includegraphics{example-image}
\lipsum[1]
\subsubsection{Sistemas de detección de formaciones ferroviarias}

Es de vital importancia que el sistema pueda determinar la posición de un tren dentro del tendido ferroviario. De esta manera, poder habilitar la circulación en secciones donde no exista peligro de colisión con otros formaciones o, por el contrario, detener la marcha de las formaciones anteriores para evitar accidentes. Existen diversas maneras de detectar la posición de un tren, entre ellas el uso de circuitos de vía y contadores de ejes. 

Los circuitos de vía (Figura \ref{fig:deteccion_1}) son dispositivos electrónicos que aplican una diferencia de potencial finita entre los rieles. Cuando una formación ingresa a la sección, sus ruedas metálicas cortocircuitan ambos rieles. El cortocircuito es detectado por el relé y este, a su vez, reporta el estado al resto del sistema. 

    \begin{figure}[!h]
        \centering
        \includegraphics[width=1\textwidth]{Figuras/circuito_via}
        \centering\caption{Circuito de vía libre y ocupado.}
        \label{fig:deteccion_1}
    \end{figure}

En caso de que la alimentación se interrumpa, el cableado sufra alguna falla, vandalismo, inundación, o que efectivamente una formación ocupe la sección, el circuito de vía reportará que la sección se encuentra ocupada. De esta manera, solo es posible recibir un reporte de sección desocupada cuando la sección efectivamente se encuentre desocupada. A este principio se le denomina fail-safe [REF]. Es decir, si por alguna razón algo falla, el sistema adopta la condición más restrictiva, mitigando la posibilidad de una situación peligrosa

Los sistemas contadores de ejes (Figura \ref{fig:deteccion_2}) consisten en sensores pasivos instalados en la cara interna de unos de los rieles y un sistema externo de procesamiento de datos. Estos sistemas no dependen de la aplicación de tensiones en la vía. Además, no solo permiten detectar la presencia de una formación, sino que también pueden usarse para medir la integridad de la formación, sabiendo el largo esperado de la misma. 

    \begin{figure}[!h]
        \centering
        \includegraphics[width=1\textwidth]{example-image}
        \centering\caption{Contadores de ejes.}
        \label{fig:deteccion_2}
    \end{figure}

Al igual que los circuitos de vía, los sistemas contadores de eje siguen el principio de fail-safe, adoptando la condición mas restrictiva en caso de falla. Ambos sistemas pueden utilizarse en simultáneo, de ser requerido.
\subsubsection{Estaciones ferroviarias}

Las estaciones ferroviarias son las zonas donde las formaciones pueden detenerse para que los pasajeros puedan descender y nuevos pasajeros puedan abordar. En función del tamaño de las formaciones y la geografía del lugar, las plataformas desde donde ascienden y descienden los pasajeros pueden estar elevadas con respecto al suelo o a ras del mismo. El largo de las plataformas también depende de la cantidad de coches de las formaciones.

Como puede verse en la Figura \ref{fig:estacion_1}, las estaciones ferroviarias incluyen no solo a las plataformas, sino que también pueden centralizar el control de varias operaciones logísticas como la asignación de rutas. No obstante, en este trabajo nos referiremos a las estaciones como plataformas indistintamente.

    \begin{figure}[h]
        \centering
        \includegraphics[width=0.5\textwidth]{example-image}
        \centering\caption{XXXXX.}
        \label{fig:estacion_1}
    \end{figure}

Las estaciones de mayor complejidad o de mayor convergencia de ramales suelen concentrar el control de la estación donde se encuentran y varias estaciones vecinas. Las estaciones terminales, a menudo, pueden incluso tener control total de varios ramales completos.
\subsubsection{Cruces ferroviarios}

\lipsum[1]

\lipsum[1]
\includegraphics{example-image}
\lipsum[1]
\subsubsection{Máquina de cambios}

\lipsum[1]

\lipsum[1]
\includegraphics{example-image}
\lipsum[1]
\subsubsection{Señales ferroviarias}

\lipsum[1]

\lipsum[1]
\includegraphics{example-image}
\lipsum[1]

\subsection{Sistemas de enclavamiento}

\lipsum[1]

\includegraphics{example-image}

\lipsum[1]

\subsubsection{Bloqueo de máquina de cambios por ocupación}

\lipsum[1]
\includegraphics{example-image}
\lipsum[1]
\subsubsection{Requerimiento de rutas y bloqueo de cambios en ruta}

\lipsum[1]
\includegraphics{example-image}
\lipsum[1]
\subsubsection{Proteccion por aproximacion}

\lipsum[1]
\includegraphics{example-image}
\lipsum[1]
\subsubsection{Proteccion por solape}

\lipsum[1]
\includegraphics{example-image}
\lipsum[1]
\subsubsection{Doble recubrimiento}

\lipsum[1]
\includegraphics{example-image}
\lipsum[1]
\subsubsection{Liberacion secuencial}

\lipsum[1]
\includegraphics{example-image}
\lipsum[1]
\subsection{Topologias ferroviarias}

\subsubsection{Simple}

\lipsum[1]

    \begin{figure}[h]
        \centering
        \includegraphics[width=1\textwidth]{Figuras/simple}
        \centering\caption{Topología simple.}
        \label{fig:simple_1}
    \end{figure}
    
\lipsum[1]
\subsubsection{Bypass}

\lipsum[1]
\includegraphics{example-image}
\lipsum[1]
\subsection{Estación de alta densidad}

A medida que mas ramales ferroviarios coexisten en la misma línea se vuelve inevitable que varias líneas compartan la misma estación utilizando diferentes plataformas en paralelo. Con una logística mas flexible, las diferentes líneas incluso pueden utilizar de forma alternada las mismas plataformas y, por lo tanto, las mismas vías principales. Además, es necesario contar con mecanismos para retirar trenes de la red para su mantenimiento y volver a inyectarlos a la red cuando la demanda aumente. Esto se logra por medio de talleres ferroviarios en las inmediaciones de las estaciones que actúan como un hub ferroviario. La topología de estación de alta densidad se ilustra en la Figura \ref{fig:hub_1}.

    \begin{figure}[h]
        \centering
        \includegraphics[width=1\textwidth]{Figuras/hub}
        \centering\caption{Topología de Estación de alta densidad.}
        \label{fig:hub_1}
    \end{figure}
    
%Las tareas del sistema de enclavamientos van aumentando en complejidad a medida que se suman nuevos elementos ferroviarios. Debe coordinar diversas formaciones de distintas líneas, accediendo a diferentes plataformas, cumpliendo diferentes horarios de arribo y partida. A su vez, debe asegurarse de que las formaciones circulen con seguridad pero sin descuidar la puntualidad. Adicionalmente, debido a que la demanda varía a lo largo del día, deberá tener flexibilidad para inyectar nuevas formaciones a la red o remover las que presenten desperfectos técnicos. Todas estas acciones deben realizarse en simultáneo y en un entorno de alto dinamismo.

Las tareas del sistema de enclavamientos van aumentando en complejidad a medida que se suman nuevos elementos ferroviarios. Por ejemplo, debe segurar que las formaciones circulen con seguridad pero sin afectar la disponibilidad del sistema. %Adicionalmente, debido a que la demanda varía a lo largo del día, deberá tener flexibilidad para inyectar nuevas formaciones a la red o remover las que presenten desperfectos técnicos. Todas estas acciones deben realizarse en simultáneo y en un entorno de alto dinamismo.
\subsubsection{Terminal}

\lipsum[1]

    \begin{figure}[h]
        \centering
        \includegraphics[width=1\textwidth]{Figuras/terminal}
        \centering\caption{Topología terminal.}
        \label{fig:terminal_1}
    \end{figure}
    
\lipsum[1]
\subsubsection{Complejas}

\lipsum[1]
\includegraphics{example-image}
\lipsum[1]