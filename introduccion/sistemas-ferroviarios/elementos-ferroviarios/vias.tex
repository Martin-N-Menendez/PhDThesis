\subsubsection{Vías}

Las vías férreas son el elemento ferroviario mas esencial, son la columna vertebral de la infraestructura ferroviaria. Estas constituyen el sitio por el cual se desplazan los trenes, definiendo no solo la dirección del desplazamiento, sino también restringiendo el dominio del tren. Esto lo diferencia de otros medios de transporte como el automóvil que no necesitan un camino para circular y, aún teniendo una carretera, puede moverse libremente por fuera de esta.

Las vías se encuentran separadas por una distancia fija que se mide desde sus caras internas, denominada trocha (Figura \ref{fig:vias_1}). Solamente las formaciones compatibles con ese parámetro de trocha pueden circular por el tendido ferroviario. El valor de la trocha puede variar entre las denominadas trocha angosta (600 a 1372 mm, estándar imperial británico) y trocha ancha (1520 a 3000 mm, estándar ruso, indio, ibérico, irlandés). Se estableció el valor intermedio de 1425 mm como valor de trocha internacional, usado ampliamente en Europa, Norteamérica y Oceanía.

    \begin{figure}[h]
        \centering
        \includegraphics[width=0.5\textwidth]{example-image}
        \centering\caption{Vías ferroviarias y trocha.}
        \label{fig:vias_1}
    \end{figure}
    
Existen limitaciones logísticas y físicas por las cuales el tendido ferroviario no puede ser un continuo rígido. En primer lugar, las vías deben ser de un tamaño acotado, tal que puedan transportarse a la locación donde serán instaladas en tramos rectos o curvos. En segundo lugar, la dilatación y contracción de las vías debido a los cambios de temperatura añaden una restricción respecto a la distancia mínima que deben tener entre las mismas. De lo contrario, la dilatación del material puede provocar daños irreparables a la infraestructura y estos, a su vez, ser motivo de descarrilamientos, como ya ha ocurrido en los comienzos de la industria ferroviaria [REF]. 
    
Cada vía puede ser clasificada en dos grupos: vías ascendentes o vías descendentes (\ref{fig:vias_2}). Las ascendentes son aquellas por las cuales los trenes circulan únicamente en la dirección del kilometraje en sentido creciente. Las descendentes son aquellas por las cuales los trenes circulan únicamente en la dirección del kilometraje en sentido decreciente [REF]. 

    \begin{figure}[h]
        \centering
        \includegraphics[width=0.5\textwidth]{example-image}
        \centering\caption{Vías ascendentes y descendentes.}
        \label{fig:vias_2}
    \end{figure}

El kilómetro 0 es la estación principal de la línea ferroviaria, como pueden ser las terminales de Plaza Constitución (para la línea Roca), Once de septiembre (para la línea Sarmiento) y Retiro (para las líneas Mitre y San Martín).  Existen vías de maniobra que pueden ser tanto ascendentes como descendentes. Estas vinculan, mediante un cambio de vías, una sección ascendente con otra descendente, en la cual los trenes deben circular a una velocidad reducida. 

Las vías se agrupan en secciones que, por cuestiones de seguridad y logística, se establece que solo pueden ser utilizadas por un tren a la vez. Estas secciones pueden ser de varios kilómetros en zonas rurales o unos pocos cientos de metros en zonas urbanas, donde la red necesita una mayor granularidad debido a la densidad del tráfico ferroviario en las grandes urbes.