\subsubsection{Vías}

Las vías férreas (Figura XX) son un elemento ferroviario esencial y son la columna vertebral de la infraestructura ferroviaria. Estas constituyen el sitio por el cual se desplazan los trenes, definiendo no solo la dirección del desplazamiento, sino también restringiendo el dominio del tren. Esto lo diferencia de otros medios de transporte como el automóvil que no necesitan un camino para circular y, aún teniendo una carretera, puede moverse libremente por fuera de esta.

Las vías se encuentran separadas por una distancia fija que se mide desde sus caras internas, denominada trocha. Solamente las formaciones compatibles con ese parámetro de trocha pueden circular por el tendido ferroviario. El valor de la trocha puede variar entre las denominadas trocha angosta (600 a 1372 mm, estandar imperial británico) y trocha ancha (1520 a 3000 mm, estándar ruso, indio, ibérico, irlandes). Se estableció el valor intermedio de 1425 mm como valor de trocha internacional, usado ampliamente en Europa, Norteamérica y Oceanía.

%%%

Las vías se dividen en secciones y por seguridad se establece que cada sección puede contener solo una formación por vez. Las mismas pueden tener largos variables en zonas urbanas de entre 500 a 1000 metros en zonas rurales. 

Cada vía puede ser clasificada en dos grupos: vías ascendentes o vías descendentes. Las ascendentes son aquellas por las cuales los trenes circulan únicamente en la dirección del kilometraje en sentido creciente. Las descendentes son aquellas por las cuales los trenes circulan únicamente en la dirección del kilometraje en sentido decreciente [REF]. El kilómetro 0 es la estación principal de la línea ferroviaria, como por ejemplo: Plaza Constitución (para la línea Roca), Once de septiembre (para la línea Sarmiento) o Retiro (para las líneas Mitre y San Martín). 

Existen vías de maniobra que pueden ser tanto ascendentes como descendentes. Estas vinculan, mediante un cambio de vías, una sección ascendente con otra descendente, en la cual los trenes deben circular a una velocidad reducida.

\lipsum[1]
\includegraphics{example-image}
\lipsum[1]