\subsubsection{Estaciones ferroviarias}

Las estaciones ferroviarias son las zonas donde las formaciones pueden detenerse para que los pasajeros puedan descender y nuevos pasajeros puedan abordar. En función del tamaño de las formaciones y la geografía del lugar, las plataformas desde donde ascienden y descienden los pasajeros pueden estar elevadas con respecto al suelo o a ras del mismo. El largo de las plataformas también depende de la cantidad de coches de las formaciones.

Como puede verse en la Figura \ref{fig:estacion_1}, las estaciones ferroviarias incluyen no solo a las plataformas, sino que también pueden centralizar el control de varias operaciones logísticas como la asignación de rutas. No obstante, en este trabajo nos referiremos a las estaciones como plataformas indistintamente.

    \begin{figure}[h]
        \centering
        \includegraphics[width=0.5\textwidth]{example-image}
        \centering\caption{XXXXX.}
        \label{fig:estacion_1}
    \end{figure}

Las estaciones de mayor complejidad o de mayor convergencia de ramales suelen concentrar el control de la estación donde se encuentran y varias estaciones vecinas. Las estaciones terminales, a menudo, pueden incluso tener control total de varios ramales completos.