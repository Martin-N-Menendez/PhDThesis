\subsubsection{Señales ferroviarias}

El sistema de señalamiento utiliza los semáforos ferroviarios (en adelante denominados señales) para indicarle al conductor del tren si tiene autoridad de tránsito en al próximo tramo de vías y a qué velocidad se le permite circular; esto, por medio del color del semáforo, denominado aspecto. A diferencia de los semáforos vehiculares, en los que cada color es alternado por otro de la secuencia rojo-amarillo-verde en función del tiempo, los semáforos ferroviarios cambian su aspecto en función de los eventos de los tramos siguientes. En la Figura \ref{fig:signal_1} se presenta un esquema de señales de tres aspectos, que es el tipo de semáforo que se utiliza en la gran mayoría de las líneas ferroviarias.

    \begin{figure}[h]
        \centering
        \includegraphics[width=0.5\textwidth]{example-image}
        \centering\caption{XXXXX.}
        \label{fig:signal_1}
    \end{figure}

Otra diferencia fundamental es que no todos los semáforos ferroviarios poseen tres aspectos. Los semáforos de maniobras constan de solo dos, amarillo (precaución) y rojo (prohibido avanzar), y algunas líneas, como la Línea Roca, utilizan semáforos de cuatro aspectos. En la Figura \ref{fig:signal_2} se visualizan los semáforos de dos aspectos. Se utilizan en cambios de vías donde, por su peligrosidad, solo se podrían permitir aspectos rojos y amarillos.

    \begin{figure}[h]
        \centering
        \includegraphics[width=0.5\textwidth]{example-image}
        \centering\caption{XXXXX.}
        \label{fig:signal_2}
    \end{figure}

Los semáforos de cuatro aspectos son utilizados en la Línea Roca y poseen un doble amarillo antes del amarillo simple, para permitir así tramos de vías más cortos de forma segura. Como no son objeto de estudio del presente trabajo, no serán explicados aquí. (EDITAR)

    \begin{figure}[h]
        \centering
        \includegraphics[width=0.5\textwidth]{example-image}
        \centering\caption{XXXXX.}
        \label{fig:signal_3}
    \end{figure}

EXPLICAR MAS LOS SEMAFOROS.