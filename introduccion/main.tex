\chapter{Introducción}

    El diseño de los sistemas de señalamiento es un proceso complejo que involucra, principalmente, tres etapas: el análisis de la red ferroviaria, la detección de zonas de mayor probabilidad de descarrilamientos y colisiones, y la óptima localización de las señales correctas para cada función [REF]. La generación automática del señalamiento es de gran importancia y utilidad para el desarrollo de redes ferroviarias nuevas o para revitalizar redes ferroviarias abandonadas o en desuso. Es relevante incluso en redes ferroviarias que son alteradas debido a la adición, modificación o eliminación de elementos ferroviarios tales como pasos a nivel o plataformas, lo cual implica que el señalamiento debo actualizarse en consecuencia.
    
    En este trabajo presentamos una herramienta capaz de automatizar el diseño del señalamiento y la implementación del sistema de enclavamiento ferroviario para cualquier locación, dado únicamente el diseño de su traza ferroviaria. Se discutirán los detalles del diseño de la herramienta, su arquitectura, casos de uso y aplicaciones en locaciones teóricas y reales.

\section{Topologías ferroviarias}

    Las redes ferroviarias presentan dos piezas fundamentales en su infraestructura: su topología (la traza ferroviaria) y los elementos ferroviarios que la componen. La topología es el entramado de vías férreas conectadas de forma arbitraria, cuyo diseño busca cumplir una función particular, interconectando diversos elementos ferroviarios. Estos elementos pueden ser para determinar la ubicación del tren, para delimitar la circulación de vehículos en cruces ferroviarios, permitir el ascenso y descenso de pasajeros, o para modificar dinámicamente los caminos por los que los trenes circulan, entre otras funciones.

    Cualquiera sea el elemento involucrado, altera las funciones de la red y su inminente proximidad debe informarse cuanto antes al conductor del tren. Este podrá decidir, de estar permitido, modificar o no su accionar antes de alcanzar dicho elemento. Es tarea del señalamiento ferroviario alertar a los conductores ferroviarios de cualquier elemento que pueda representar un peligro, evitando así colisiones con otras formaciones o descarrilamientos en zonas críticas. El señalamiento ferroviario incluye un elemento fundamental: los semáforos.

    Los semáforos (de ahora en mas denominados "señales"), constituyen el medio de comunicación primario entre los conductores y su entorno, informándoles de la habilitación o denegación de uso de las vías posteriores mediante su color, denominado aspecto. Cada señal puede presentar un único aspecto por vez de un conjunto posible que varía según el país o la región. Los aspectos utilizados en Argentina son el verde (permitido avanzar), amarillo (atención) y rojo (detenerse). Algunas señales pueden no tener el aspecto verde (señales de maniobras de dos aspectos) o incorporar un aspecto extra entre el rojo y el amarillo (señales de cuatro aspectos que incluyen el doble amarillo).
    
    Dos señales consecutivas con la misma dirección y sentido constituyen una ruta ferroviaria. Los operarios ferroviarios solicitan al sistema de enclavamientos las rutas que necesitan en base a la logística deseada. El sistema de enclavamientos habilitará o denegará las rutas solicitadas en función del estado de los elementos ferroviarios cercanos y de las demás rutas activas. Esta función es vital para el sistema ferroviario y su fin último es permitir la circulación de trenes de la forma mas segura o, de no ser posible, no permitir circulación alguna.

    \subsection{Bypass}

Cuando es necesario interconectar dos puntos separados por una distancia de cientos de kilómetros, resulta económicamente poco conveniente construir vías en ambos sentidos. No obstante, construir una sola vía bidireccional presenta inconvenientes logísticos notorios: Una formación que circule entre los puntos A y B excluye a cualquier formación que quiera circular de B a A sin colisionar. No sería posible utilizar la infraestructura en el sentido opuesto mientras se encuentre ocupada.

La solución mas utilizada emplea una vía bidireccional única intercalada cada cierta cantidad de kilómetros por islas de enclavamiento a modo de bypass, como se ilustra en la Figura \ref{fig:bypass_1}. Estas islas permiten que las formaciones puedan cruzarse sin riesgo de colisión. La primer formación en llegar a la isla de enclavamientos accede al bypass por la vía superior y espera a que la formación en sentido contrario circule por la vía inferior. Una vez despejado el camino que resta por recorrer, la formación reingresa a la vía principal y retoma su marcha.

    \begin{figure}[h]
        \centering
        \includegraphics[width=1\textwidth]{Figuras/bypass}
        \centering\caption{Topología bypass.}
        \label{fig:bypass_1}
    \end{figure}
    
Las topologías de bypass se utilizan principalmente para transportar materias primas entre locaciones rurales a grandes distancias de los puestos. Es deseable tanto una logística óptima, para transportar mas bienes y mas rápido, cómo un sistema seguro que garantice que los bienes lleguen a destino.
    \subsection{Simple}

En entornos urbanos donde las estaciones ferroviarias se encuentran separadas entre sí por unos pocos kilómetros es necesaria una interconectividad mayor. El sistema ferroviario debe satisfacer la demanda de una población mayor y a la vez coexistir con un trazado vehicular mucho mas denso que cruza al trazado ferroviario en varios puntos. En este contexto, una topología simple como la presentada en la Figura \ref{fig:simple_1} es una solución óptima al problema planteado.

    \begin{figure}[h]
        \centering
        \includegraphics[width=1\textwidth]{Figuras/simple}
        \centering\caption{Topología simple.}
        \label{fig:simple_1}
    \end{figure}

El cruce entre el trazado ferroviario y el trazado vehicular se denomina paso a nivel. El sistema de enclavamientos deberá garantizar que el paso a nivel se encuentre despejado de vehículos y peatones antes de permitir la circulación de trenes sobre el mismo. Esto se logra mediante el uso de una barrera ferroviaria, un mecanismo que mantiene la barrera en alto mientras no se detecten formaciones en las proximidades del paso a nivel.

Las topologías simples suelen contar con dos vías unidireccionales en sentido ascendente y descendente. 
Las vías ascendentes son aquellas por las que las formaciones circulan en la dirección del kilometraje creciente. Mientras que las vías descendentes son aquellas por las que circulan en la dirección del kilometraje decreciente [REF]. El kilómetro cero es la estación principal de la línea ferroviaria, como por ejemplo: Plaza Constitución (Línea Roca), Once de Septiembre (Línea Sarmiento) o Retiro (Línea Mitre y Linea San Martín). Las formaciones pueden cambiar de vía ascendente a descendente, o viceversa, utilizando un cambio ferroviario.
    \subsection{Estación de alta densidad}

A medida que mas ramales ferroviarios coexisten en la misma línea se vuelve inevitable que varias líneas compartan la misma estación utilizando diferentes plataformas en paralelo. Con una logística mas flexible, las diferentes líneas incluso pueden utilizar de forma alternada las mismas plataformas y, por lo tanto, las mismas vías principales. Además, es necesario contar con mecanismos para retirar trenes de la red para su mantenimiento y volver a inyectarlos a la red cuando la demanda aumente. Esto se logra por medio de talleres ferroviarios en las inmediaciones de las estaciones que actúan como un hub ferroviario. La topología de estación de alta densidad se ilustra en la Figura \ref{fig:hub_1}.

    \begin{figure}[h]
        \centering
        \includegraphics[width=1\textwidth]{Figuras/hub}
        \centering\caption{Topología de Estación de alta densidad.}
        \label{fig:hub_1}
    \end{figure}
    
%Las tareas del sistema de enclavamientos van aumentando en complejidad a medida que se suman nuevos elementos ferroviarios. Debe coordinar diversas formaciones de distintas líneas, accediendo a diferentes plataformas, cumpliendo diferentes horarios de arribo y partida. A su vez, debe asegurarse de que las formaciones circulen con seguridad pero sin descuidar la puntualidad. Adicionalmente, debido a que la demanda varía a lo largo del día, deberá tener flexibilidad para inyectar nuevas formaciones a la red o remover las que presenten desperfectos técnicos. Todas estas acciones deben realizarse en simultáneo y en un entorno de alto dinamismo.

Las tareas del sistema de enclavamientos van aumentando en complejidad a medida que se suman nuevos elementos ferroviarios. Por ejemplo, debe segurar que las formaciones circulen con seguridad pero sin afectar la disponibilidad del sistema. %Adicionalmente, debido a que la demanda varía a lo largo del día, deberá tener flexibilidad para inyectar nuevas formaciones a la red o remover las que presenten desperfectos técnicos. Todas estas acciones deben realizarse en simultáneo y en un entorno de alto dinamismo.
    \subsection{Estación Terminal}

Las estaciones terminales presentan una gran cantidad de vías principales y plataformas en paralelo, en las cuales confluyen una o varias líneas ferroviarias. A diferencia de estaciones de alta densidad que pueden presentar finales de vía relativos, las estaciones terminales poseen finales de vía absolutos. Es decir, las formaciones que circulan por la vía descendente deberán detener su marcha completamente antes de llegar al fía de vía, para luego retomar su marcha en sentido contrario, por la vía ascendente. Esta operación puede realizarse de manera inmediata en formaciones con locomotras eléctricas en ambos extremos del tren o con locomotoras diesel luego de varias maniobras que requieren el uso de diversos cambios de vías. En la Figura \ref{fig:terminal_1} se ilustra un ejemplo de una estación terminal.

    \begin{figure}[H]
        \centering
        \includegraphics[width=1\textwidth]{Figuras/terminal}
        \centering\caption{Topología terminal.}
        \label{fig:terminal_1}
    \end{figure}

En las estaciones terminales suelen confluir la información en tiempo real de la terminal y las estaciones más próximas de la línea, o incluso la información en tiempo real de la totalidad de la línea. Esta característica, además de ser la estación de mayor tamaño de la línea, les otorga una jerarquía tal que suelen concentrar parcial o totalmente el control del señalamiento de la red. Las decisiones tomadas en una estación terminal tienen un gran impacto en el sistema de transporte de toda la línea, directa o indirectamente. Estas operaciones deben considerar cientos o miles de estados en simultáneo, por lo que ejecutarlas de forma manual es muy complejo o incluso imposible. Un sistema de enclavamientos moderno, robusto, que pueda garantizar una altísima disponibilidad, mantenibilidad y seguridad es indispensable para llevar a cabo estas tareas.
    \subsubsection{Complejas}

\lipsum[1]
\includegraphics{example-image}
\lipsum[1]
\section{Estado del arte}

\section{Tipos de sistemas de enclavamiento}

    Lamentablemente, los sistemas de enclavamiento en Argentina son en su mayoría mecánicos, de comienzos del siglo XX, y otra cantidad considerable son electromecánicos, de más de 40 años de antigüedad. Muchos de ellos ya han agotado su vida útil y deben ser reemplazados. Otros, en cambio, han estado en desuso por años y necesitan ser repuestos, pero solo una docena de empresas en el mundo realizan el diseño del sistema y los costos para un bypass simple rondan las decenas de millones de dólares. Por esto, es importante contar con sistemas electrónicos de diseño y fabricación nacional.
    
    Además, existen diferentes lugares donde aún resta instalar este tipo de sistemas, por lo que su implementación constituye una necesidad real para el desarrollo de la infraestructura ferroviaria de señalamiento en Argentina.
    
    A comienzos del siglo XX, se implementaron los sistemas de enclavamiento mediante soluciones mecánicas. Utilizaban palancas, como las que se visualizan en la Figura \ref{fig:enclavamiento_2}, para comandar los cambios de vías y semáforos.
    
        \begin{figure}[h]
            \centering
            \includegraphics[width=0.5\textwidth]{example-image}
            \centering\caption{XXXXX.}
            \label{fig:enclavamiento_2}
        \end{figure}
    
    Una vez que se constituye una configuración de posiciones de palancas que habilitan un trayecto, estas quedan "enclavadas" mecánicamente, es decir, su posición se bloquea y no es físicamente posible cambiarla. A medida que se van moviendo ciertas palancas, las demás que pudieran representar situaciones no seguras quedan enclavadas, y solo se pueden mover aquellas cuyo accionamiento representa una situación segura. De esa manera, se garantiza que no se generarán configuraciones tales que las formaciones colisionen entre sí.
    
    Las tecnologías más modernas heredaron el término "enclavamiento", aunque ya no se tengan palancas enclavadas en posiciones fijas. En lugar de palancas físicas, se utilizan sistemas electrónicos y lógica programable para lograr el mismo objetivo de garantizar la seguridad ferroviaria.
    
    %----
    
    A mediados del siglo XX se desarrolló el sistema de enclavamiento electromecánico. Su funcionamiento se basa en relés (Figura \ref{fig:enclavamiento_3}) y circuitos de vía, de forma tal de poder detectar la presencia de un tren y comandar tanto las señales como las barreras de los pasos a nivel.
        
        \begin{figure}[h]
            \centering
            \includegraphics[width=0.5\textwidth]{example-image}
            \centering\caption{XXXXX.}
            \label{fig:enclavamiento_3}
        \end{figure} 
    
    Los sistemas de enclavamiento electromecánicos son comandados por un operario mediante un panel de control (Figura \ref{fig:enclavamiento_4}). El operario solicita al sistema de enclavamiento las rutas que el conductor ferroviario necesita para circular. El sistema permitirá solo la operación de cambio de vías seguras. En caso contrario, se tendrán las salidas "enclavadas" y el sistema de enclavamiento impedirá mediante los semáforos el avance de la formación hasta que pueda realizarse el cambio de forma segura.
    
        \begin{figure}[h]
            \centering
            \includegraphics[width=0.5\textwidth]{example-image}
            \centering\caption{XXXXX.}
            \label{fig:enclavamiento_4}
        \end{figure}
    
    %--- 
    HABLAR DE ENCLAVAMIENTO ELECTRONICO
\subsection{Empresas del sector ferroviario}

    Al requerir ingenieros y técnicos más especializados, el conocimiento requerido tanto para diseñar como para mantener y operar los sistemas de enclavamientos electrónicos se concentra en menos actores. La tendencia mundial en las últimas décadas ha sido el depender de soluciones cerradas provistas por alrededor de una docena de empresas, muchas veces incompatibles entre sí. Mas del 80\% del material rodante que circula en el mundo proviene de alguna de estas doce empresas \cite{MARKET}:

    \begin{itemize}
        \item CRRC Railway (China) \cite{CRRC}
        \item Alstom (Francia) \cite{ALSTOM}
        \item Bombardier Transportation (Canada) \cite{BOMBARDIER}
        \item CAF (España) \cite{CAF}
        \item Hitachi (Japón) \cite{HITACHI}
        \item Transmashholding (Rusia) \cite{TRANSMASHHOLDING}
        \item Hyundai Rotem (Corea del sur) \cite{HYUNDAI}
        \item Siemens Mobility (Alemania) \cite{SIEMENS}
        \item Thales (Reino unido) \cite{THALES}
        \item General Electic Transportation (EEUU) \cite{GENERAL}
        \item Caterpillar (EEUU) \cite{CATERPILLAR}
        \item Stadler Rail (Suiza) \cite{STADLER}
    \end{itemize}
    
    Al lado de cada empresa se puede ver su país de origen. Cuatro de esos países (China, EE.UU., Canadá y Rusia) son las naciones con mayor extensión territorial, para lo cual un sistema ferroviario robusto es esencial. El diseño de un sistema de enclavamiento se rige por el mismo estándar de calidad y seguridad que los sistemas aeroespaciales y nucleares: IEC 61508 \cite{Paper_77,Paper_78,Paper_79,Paper_80,Paper_81,Paper_82,Paper_83}.

    
\subsection{Herramientas existentes}

\lipsum[1]
\subsection{Antecedentes de trabajos académicos en la temática}
\label{sec:estudios}
    Dada la variedad de topologías (algunas mencionadas en la Sección \ref{sec:topologias}), es deseable automatizar tanto el análisis de la red como el proceso de diseño del señalamiento. De esta manera, se puede reducir el error humano que puede afectar el proceso en cualquiera de sus etapas: análisis, diseño, implementación, verificación y validación.

    Varios artículos abordan la automatización de las tablas de enclavamiento \cite{Paper_2,Paper_162,Paper_182}, que son las tablas que indican que elementos ferroviarios utilizan cada ruta que la formación puede solicitar al sistema. También podemos encontrar trabajos respecto de la automatización de otros elementos como el comportamiento de los enclavamientos \cite{Paper_99,Papaer_158,Paper_182,Paper_197}, las rutas \cite{Papaer_114}, el señalamiento de los pasos a nivel \cite{Paper_85,Paper_86}, topologías simples \cite{Paper_93.Paper_162}, así como sus simulaciones.

    Existen, además, diversos estudios acerca de la verificación y validación de los sistemas ferroviarios utilizando métodos formales [REF]. Algunos emplean herramientas que analizan y validan el modelo tales como NuSMV [REF] or soluciones completas como el B-method [REF]. No obstante, luego de recopilar mas de 150 artículos [REF], no hemos encontrado una herramienta (o conjunto de herramientas) que realice el análisis de una red ferroviaria arbitraria, generando automáticamente su señalamiento y la implementación automática de su sistema de enclavamiento asociado. Los pocos casos que aventuran análisis automáticos lo realizan solo para topologías simples de un único elemento [REF], mientras que en este trabajo se busca la generalización para cualquier red ferroviaria con cualquier combinación de diversos elementos ferroviarios.
\subsection{Enfoque funcional vs enfoque geografico}

\lipsum[1]

\subsection{RailTopoModel}

\lipsum[1]

\subsubsection{Modelo de grafos}

\lipsum[1]
\subsubsection{Nivel micro, meso y macro}

\lipsum[1]
\subsection{RailML}

\lipsum[1]

\subsubsection{RailML 3.0}

    railML (del inglés, Railway Markup Language) es un estándar abierto de intercambio de datos basado en XML, para intercomunicar aplicaciones ferroviarias. El estándar railML 3.0 adopta gran parte de los conceptos de RailTopoModel y los expande, incorporando nuevos elementos, en base a las necesidades de las empresas ferroviarias que lo adoptan. Posee cinco módulos principales:

    \begin{itemize}
        \item Common (CO): similar al módulo base de RailTopoModel. Contiene datos transversales a todo el sistema, como el autor del archivo, el sistema eléctrico de alimentación, la versión de railML utilizada, etc.
        \item Infrastructure (IS): incorpora al módulo de topología de RailTopoModel, con sus netElement y netRelations. Ademas del sistema de coordenadas, la geometría y todos los elementos ferroviarios. Es estos últimos solo admite datos estáticos como la posición, tamaño y demás atributos invariantes en el tiempo.
        \item Interlocking (IL): incorpora los atributos dinámicos de los elementos ferroviarios mencionados en la infraestructura y agrega el listado de rutas.
        \item Rollingstock (RS): incorpora toda la información relativa al material rodante: coches, vagones, locomotoras, formaciones combinadas, etc.
        \item Timetable and Rostering (TT): incorpora detalles logísticos de la red ferroviaria.
    \end{itemize}

    A lo largo de esta tesis doctoral nos enfocaremos exclusivamente en los tres primeros módulos. Los cuales serán analizados detalladamente a medida que sea necesario introducirlos.
\subsubsection{Uso en la industria}

\lipsum[1]
\section{Impacto economico}

\lipsum[1]
\includegraphics{example-image}
\lipsum[1]