\subsection{Clase common}
    \label{sec:common}

    La clase Common define todos los parámetros que son invariantes a toda la red. Esta clase puede ser definida solo una vez o no definirse. Sus subclases son:

    \begin{itemize}
        \item ElectrificationSystems: define la tensión y frecuencia de la red eléctrica.
        \item OrganizationalUnits: define quien administra la infraestructura, quien fabrica y opera el material rodante, quien es el cliente final, quien es el contratista y quien posee la concesión del servicio. 
        \item SpeedProfiles: define el perfil de aceleración, velocidades y frenado.
        \item Positioning: define los sistemas de posicionamiento geométrico, linear y referido a la pantalla del editor.
    \end{itemize}

    Cada uno de sus campos internos es único, pero también son opcionales. A modo de ejemplo se muestra el Código \ref{lst:common}, donde se ilustra como no todas las subclases han sido definidos.
    
    \begin{lstlisting}[language = XML, caption = Clase Common , label = {lst:common}]
<common id="co_01">
    <organizationalUnits>
        <infrastructureManager id="im_01"/>
    </organizationalUnits>
    <positioning>
        <geometricPositioningSystems>
            <geometricPositioningSystem id="gps01">
                <isValid from="2023-07-26" to="2024-07-26"/>
                <name name="Example_9.railml" language="en"/>
            </geometricPositioningSystem>
        </geometricPositioningSystems>
        <linearPositioningSystems>
            <linearPositioningSystem linearReferencingMethod="absolute" 
            startMeasure="0" endMeasure="0" units="Km" id="loc-1">
                <isValid from="2023-07-26" to="2024-07-26"/>
                <name name="Loc-1" language="en"/>
            </linearPositioningSystem>
        </linearPositioningSystems>
    </positioning>
</common>
    \end{lstlisting}

    Solamente los subclases OrganizationalUnits y Positioning fueron definidas, pero tanto el RNA como el estándar railML en el que el RNA se basa consideran válido a este fragmento de código. Los vectores son definidos en plural, como en el caso de geometricPositioningSystems cuyo primer, y único elemento en este caso, es geometricPositioningSystem con id="gps01". Es habitual ver estos vectores a lo largo de todo el archivo y será esencial poder determinar su dimensión.