\subsection{Infraestructura y Redes de grafos}
    \label{sec:grafos}

    La clase Infraestructure define todos los elementos ferroviarios con características físicas, materiales y estáticas. Las subclases que contiene son:

    \begin{itemize}
        \item Topology: define la topología de la red mediante las subclases netElements y netRelationships. Incluye también la subclase Network para cumplir con el estándar de RailTopoModel.
        \item Geometry: define el sistema geométrico utilizado entre las subclases HorizontalCurve (en base al largo y el ángulo formado), GradientCurves (en base al largo y al gradiente ded la curva) y GeometryPoints (en base a una combinación de las dos subclases anteriores).
        \item FunctionalInfrastructure: define todos los elementos ferroviarios que el RNA analizará.
        \item PhysicalFacilities: actualmente el estándar lo define vacío, sin ninguna subclase interna salvo la subclase Any que puede ser utilizada como comodín.
        \item InfrastructureVisualizations: define las coordenadas y estructura del elemento ferroviario en base a sus subclases tRef (para indicar a que elemento afecta), SpotProjection (coordenada del elemento), LinearProjection (en caso de ser un elemento lineal) y AreaProjection (en caso de ser un elemento bidimensional).
        \item InfrastructureStates: define la validez de los datos referidos a cada elemento ferroviario.
    \end{itemize}

    Para realizar el análisis de la red, el RNA debe centrarse en tres de estas subclases: Topology (para conocer COMO es la red), FunctionalInfrastructure (para conocer QUE elementos tiene la red) y InfrastructureVisualizations (para conocer DONDE está cada elemento ferroviario).

    Empezando por la clase Topology, existen dos subclases esenciales para este análisis: la subclase netElements y la subclase netRelationships. Ambas son subclases vectores, constituidas por subclases mas pequeñas: netElement y netRelationship. Un ejemplo de la clase netElement se puede ver en el Código \ref{lst:netElement}.
    
    \begin{lstlisting}[language = XML, caption = Clase netElement , label = {lst:netElement}]
<netElement id="ne3">
    <associatedPositioningSystem id="ne3_aps01">
        <intrinsicCoordinate intrinsicCoord="0" id="ne3_aps01_ic01">
            <geometricCoordinate x="9384.050" y="0.000" positioningSystemRef="gps01"/>
        </intrinsicCoordinate>
        <intrinsicCoordinate intrinsicCoord="1" id="ne3_aps01_ic02">
            <geometricCoordinate x="7584.770" y="0.000" positioningSystemRef="gps01"/>
        </intrinsicCoordinate>
    </associatedPositioningSystem>
    <relation ref="nr_ne3ne46_swi77"/>
    <relation ref="nr_ne3ne53_swi77"/>
</netElement>
    \end{lstlisting}
    
    Los elementos mas importantes a destacar en el Código \ref{lst:netElement} son el id, el geometricCoordinate y el relation. De estos parametros podemos saber que el netElement es referido como "ne3", comienza en la coordenada (9384.050 ; 0.000) y termina en la coordenada (7584.770 ; 0.000). Además, ne3 se encuentra relacionado a los netElement ne46 y ne53 mediante un elemento referido como swi77, que mas adelante veremos que se trata de una máquina de cambios.

    Los netElement son los nodos de la red de grafos, pero estos no son puntuales, sino que son bidimensionales. De las coordenadas de ne3 podemos deducir que ne3 se encuentra definido de derecha a izquierda. Conocer la orientación del netElement será importante a la hora de definir la circulación, información necesaria para generar las rutas.

    La subclase netRelation son las aristas de la red de grafos que relacionan dos netElements entre si. En el Código \ref{lst:netRelation} se muestra un ejemplo de los netRelation que vinculan a los netElement ne3, ne46 y ne53.    
    
    \begin{lstlisting}[language = XML, caption = Clase netRelation , label = {lst:netRelation}]
<netRelation navigability="Both" positionOnA="1" positionOnB="1" id="nr_ne3ne46_swi77">
    <elementA ref="ne3"/>
    <elementB ref="ne46"/>
</netRelation>
<netRelation navigability="Both" positionOnA="1" positionOnB="1" id="nr_ne3ne53_swi77">
    <elementA ref="ne3"/>
    <elementB ref="ne53"/>
</netRelation>
<netRelation navigability="None" positionOnA="1" positionOnB="1" id="nr_ne46ne53_swi77">
    <elementA ref="ne46"/>
    <elementB ref="ne53"/>
</netRelation>
    \end{lstlisting}
    
    La primer netRelation es entre el netElement ne3 y el ne46, la segunda netRelation es entre el netElement ne3 y el ne53. Esto es consistente con los parámetros de relation que tenía el netElement ne3 en el Código \ref{lst:netElement}. Sin embargo, en el tercer netRelation vemos que existe una relación entre los netElement ne46 y ne53 pero, en este caso, el parámetro navigability es "None", a diferencia de los primeros dos que era "both". La navegabilidad es el parámetro que hace que una red de grafos tenga sentido como red ferroviaria: que exista una conexión no implica que sea físicamente utilizable por un tren, tal como se explico en la Sección \ref{sec:RTM}.
    
    % Explicar analisis de red de grafos


        \begin{algorithm}\captionsetup{labelfont={sc,bf}, labelsep=newline}
            \caption{Graph network analysis algorithm}
            \label{alg:graph_network}
            \begin{algorithmic}
                \STATE \{nodes\} $\gets$ get\_nodes(netElements)
                \STATE  order\_nodes(\{nodes\})
                \STATE \{netPaths\} $\gets$ get\_relations(\{nodes\},netRelations)
                \STATE \{neighbours\} $\gets$ get\_neighbours(\{netPaths\})
                \STATE \{switches\} $\gets$ get\_switches(\{nodes\},\{neighbours\})
                \STATE \{limits\} $\gets$ get\_limits(\{nodes\})
                \STATE analyze\_connectedness(\{netPaths\})
            \end{algorithmic}
        \end{algorithm}


\begin{algorithm}\captionsetup{labelfont={sc,bf}, labelsep=newline}
            \label{alg:connectedness}
            \caption{Connectivity algorithm}
            \begin{algorithmic}
                \STATE \{zones\} $\gets \{ \}$
                \STATE ADD first node in \{zones\}
                \FOR {node in \{nodes\}}
                    \FOR {zone in \{zones\}}
                        \IF {node not in zones[zone]}
                            \IF {neighbours(node) in zones[zone]}
                                \STATE zones[zone] ADD node
                            \ELSE
                                \STATE Define new\_zone
                                \STATE zones[new\_zone] ADD node
                            \ENDIF
                        \ENDIF
                    \ENDFOR
                \ENDFOR 
            \RETURN \{zones\}
            \end{algorithmic}
        \end{algorithm}

    \begin{algorithm}\captionsetup{labelfont={sc,bf}, labelsep=newline}
            \caption{Switches detector algorithm}
            \label{alg:switches}
            \begin{algorithmic}
                \STATE \{switches\} $\gets$ \{\}
                \IF {infrastructure.SwitchesIS != None} 
                    \FOR{i in infrastructure.SwitchesIS[0].SwitchIS}
                        \IF{i.Id not in switchesIS.keys()}
                            \STATE sw\_id $\gets$ i.Name[0].Name
                            \STATE j $\gets$ i.SpotLocation[0]
                            \STATE left\_id $\gets$ i.LeftBranch[0].NetRelationRef
                            \STATE right\_id $\gets$ i.RightBranch[0].NetRelationRef
                            \STATE switches[sw\_id] $\gets$ \{"Node":j.NetElementRef\}
                            \STATE switches[sw\_id] $\gets$ \{"Continue":i.ContinueCourse\}
                            \STATE switches[sw\_id] $\gets$ \{"Branch":i.BranchCourse\}
                            \STATE switches[sw\_id] $\gets$ \{"Dir":j.ApplicationDirection\}
                            \STATE switches[sw\_id] $\gets$ \{"LeftBranch":j.left\_id\}
                            \STATE switches[sw\_id] $\gets$ \{"RightBranch":j.right\_id\}
                        \ENDIF
                    \ENDFOR
                \ENDIF
                \STATE visual\_data $\gets$ visualization.Visualization
                \IF {visual\_data != None}
                    \FOR {i in  visual\_data[0].SpotElementProjection}
                        \STATE sw\_id $\gets$ i.Name[0].Name
                        \IF {"Sw" in sw\_id}
                            \STATE pos\_x $\gets$ int(i.Coordinate[0].X)
                            \STATE pos\_y $\gets$ int(i.Coordinate[0].Y)
                            \STATE switches[sw\_id] $\gets$ \{"Position":[pos\_x,-pos\_y]\}
                        \ENDIF 
                    \ENDFOR
                \ENDIF
            \RETURN switchesIS
            \end{algorithmic}
        \end{algorithm}

    %% Identificar nodos y aristas
    %% Identificar orientaciones
    %% Detectar conexidad
    %% Detectar si es una red ferroviaria o no


    \begin{algorithm}[hbt!]
        \caption{Level crossing algorithm}\label{alg:LC}
        \DontPrintSemicolon
        %\SetAlgoLined
        \SetNoFillComment
        \LinesNotNumbered 
        \For { netElement WITH LevelCrossing }
        {
            \tcc{Before reaching level crossing}
            [Signals] $\gets$ ADD circulation signal $>>>$\;
            \tcc{After leaving level crossing}
            [Signals] $\gets$ ADD circulation signal $<<<$\;
        }
        \KwResult{[Signals]} 
    \end{algorithm}



    
    
    En las siguientes subsecciones se analizaran en conjunto los elementos ferroviario descriptos en las subclases FunctionalInfrastructure y InfrastructureVisualizations.