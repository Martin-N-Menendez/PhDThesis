\section{Algoritmos de generación de señalamiento}
    \label{sec:generacion}
    
    En esta Sección se abordará la generación automática de señalamiento que realiza el RNA para cada uno de los elementos ferroviarios básicos y se mostrará como los principios de señalamiento (ver Sección \ref{sec:principios}) se cumplen para cada uno de ellos de forma individual. Se asumirá para el análisis de la red ferroviaria que todas las vías son bidireccionales y que los elementos ferroviarios deben ser protegidos en todas las direcciones. De esta forma, siempre se tendrá un análisis completo de la red ferroviaria, contemplando todos los posibles usos de la red que el operador pueda darle.
    
    Tal como se explico en la Sección \ref{sec:validacion}, al ser el modelo ferroviario de grafos un sistema lineal, podemos definir un principio de superposición de señales. Las señales generadas para un sistema con elementos ferroviarios A y B serán la superposición de las señales generadas para proteger al elemento A y las señales generadas para proteger al elemento B. No obstante, puede darse el caso de que una señal que originalmente protegía al elemento A también cumpla la función de proteger a B, por lo cual existe un solapamiento de señales. Por lo tanto, el señalamiento final tendrá menos señales que la suma de los señalamientos originales. Podemos generalizar esta conclusión en la siguiente fórmula:

    \[ \text{señales(a,b,\dots,z)} = \sum_{\text{elemento}=a}^{z} \text{señales(elemento)} - \displaystyle\bigcap_{\text{elemento}=a}^{z} \text{señales(elemento,elemento+1)} \]
    \[ \text{señales(a,b,\dots,z)} \leq \sum_{\text{elemento}=a}^{z} \text{señales(elemento)} \]

    Es decir, el señalamiento total siempre será menor a la suma del señalamiento generado para cada uno de los elementos ferroviarios, salvo que no existan señales solapadas. Para obtener el señalamiento final, el RNA debe no solo generar el señalamiento para cada elemento y combinarlo, sino también detectar solapamientos, inconsistencias o señales contradictorias que puedan comprometer la seguridad de la red y eliminarlas. La eliminación de señales deberá realizarse con ciertos criterios, para no disminuir el nivel de seguridad de la red ferroviaria. El proceso de simplificación del señalamiento se abordará en detalle en la Sección \ref{sec:simplificacion}.

    \subsection{Fin de via}

\lipsum[1]

\begin{algorithm}[hbt!]
            \caption{Line border and buffer stops algorithm}\label{alg:LBBS}
            \DontPrintSemicolon
            %\SetAlgoLined
            \SetNoFillComment
            \LinesNotNumbered 
            \For { netElement WITH BufferStops }
            {
                \If { NOT EXIST next netElement }
                {
                    [Signals] $\gets$ ADD stop signal $>>$\;
                    [Signals] $\gets$ ADD departure signal $<<$\;
                }
                \If { NOT EXIST prev netElement }
                {
                    [Signals] $\gets$ ADD stop signal $<<$\;
                    [Signals] $\gets$ ADD departure signal $>>$\;
                }
            }
            \For { netElement WITH LineBorder }
            {
                \If { netElement.Length $>$ FIXED\_LENGTH }
                {
                    \If { NOT EXIST next netElement }
                    {
                        [Signals] $\gets$ ADD departure signal $>>$\;
                    }
                    \If { NOT EXIST prev netElement }
                    {
                        [Signals] $\gets$ ADD departure signal $<<$\;
                    }
                }
            }
            \KwResult{[Signals]} 
        \end{algorithm}
    \subsection{Detectores}

\lipsum[1-2]

\begin{algorithm}[hbt!]
            \caption{Train detection elements algorithm}\label{alg:RJ}
            \DontPrintSemicolon
            %\SetAlgoLined
            \SetNoFillComment
            \LinesNotNumbered 
            \For { netElement WITH AxleCounters or RailJoints }
            {
                Track.Length = Length ( between RailJoints )\;
                \If { Track.Length $>$ FIXED\_LENGTH }
                {
                    [Signals] $\gets$ ADD circulation signal $>>>$\;
                    [Signals] $\gets$ ADD circulation signal $<<<$\;
                }
            }
            \KwResult{[Signals]} 
        \end{algorithm} 

\lipsum[1]
\includegraphics{example-image}\\
\lipsum[1]
    \subsection{Plataformas ferroviarias}

    % Autoridad > derecho limitado a una porcion
    % Claridad > autoridad no ambigua
    % Anticipacion > avisar con antelacion
    % Granularidad > rutas cortas y funcionales
    % Terminalidad > avisar fin de via
    % Infraestructura > avisar de infraestructura
    % No bloqueo > circulacion fluida

    En la Sección \ref{sec:platform} definimos la clase platform que modela a las plataformas ferroviarias. Las plataformas ferroviarias son un punto del recorrido donde las formaciones pueden detenerse, algunas veces de forma opcional dependiendo el itinerario, para que los pasajeros desciendan y nuevos pasajeros puedan ascender. Claramente existe una limitación de autoridad, las formaciones necesitan un nuevo permiso para continuar circulando una vez alcanzada la plataforma. Permiso que será otorgado o negado según el estado del sistema a continuación del recorrido. Las plataformas ferroviarias suelen encontrarse en zonas pobladas, cerca de otras infraestructuras, zonas comerciales o residenciales, por lo que avisar con antelación al maquinista que debe disminuir la marcha y/o detenerse es esencial.

    Además, es posible que las formaciones convivan con otras formaciones que también hacen uso de la estación, por lo que las señales deben ser unívocas y claras. Algunas estaciones pueden contener fines de vías o ramificaciones hacia talleres u otros ramales, por lo que también se aplica el principio de terminalidad y granuralidad. Finalmente, es importante el mantener una circulación fluida de las formaciones, de modo de no retrasar el itinerario de las formaciones que vienen detrás, por lo que se deberá aplicar el principio de no bloqueo.

    En el Algoritmo \ref{alg:PTF} definimos a la señal de partida como la única señal necesaria para operar una plataforma, asumiendo que la señal de ingreso de la misma será dada por otra instancia previa. De no existir otro elemento cercano por el cual se genere una señal, se puede asumir que la distancia a la plataforma es muy larga y, por lo tanto, se aplicaría el Algoritmo \ref{alg:RJ}, protegiendo a la plataforma. En el caso de que la vía se bidireccional se añadiran señales de partida para ambos sentidos.

    \begin{algorithm}[hbt!]
        \caption{Algoritmo de generación de señalamiento para platforms.}\label{alg:PTF}
        \DontPrintSemicolon
        %\SetAlgoLined
        \SetNoFillComment
        \LinesNotNumbered 
        \For { netElement WITH Platform }
        {
            \tcc{Before leaving platform from left}
            [Signals] $\gets$ ADD departure signal $\gg\gg$\;
            \tcc{After leaving platform from right}
            [Signals] $\gets$ ADD departure signal $\ll\ll$\;
        }
        \KwResult{[Signals]} 
    \end{algorithm}

    Aplicando el Algoritmo \ref{alg:PTF} a un sistema de dos vías paralelas con plataformas pertenecientes a la misma estación obtenemos el resultado ilustrado en la Figura \ref{fig:signal_platform}.Se asumieron que ambas vías son bidireccionales, en caso contrario solo se generarían las señales S01 y S02.
    
    \begin{figure}[h!]
        \centering
        \includegraphics[width=1\textwidth]{Figuras/platforms.PNG}
        \centering\caption{Señalamiento generado para estaciones ferroviarias.}
        \label{fig:signal_platform}
    \end{figure}
    
    Una formación que circule de izquierda a derecha deberá detenerse antes de la señal S01, en caso de utilizar la vía superior, o antes de la señal S04, en caso de utilizar la vía inferior. Análogamente, las formaciones deberán detenerse antes de las señales S03 y S02 en el caso de transitar de derecha a izquierda por la vía superior o inferior respectivamente. Sólo cuando estas señales otorguen a la formación autoridad para circular podrán reanudar su marcha hasta la próxima señal disponible, fuera del alcance de lo ilustrado en la Figura \ref{fig:signal_platform}.
    \subsection{Cruces de via}

\lipsum[1]

\begin{algorithm}[hbt!]
            \caption{Level crossing algorithm}\label{alg:LC}
            \DontPrintSemicolon
            %\SetAlgoLined
            \SetNoFillComment
            \LinesNotNumbered 
            \For { netElement WITH LevelCrossing }
            {
                \tcc{Before reaching level crossing}
                [Signals] $\gets$ ADD circulation signal $>>>$\;
                \tcc{After leaving level crossing}
                [Signals] $\gets$ ADD circulation signal $<<<$\;
            }
            \KwResult{[Signals]} 
        \end{algorithm}
    \subsection{Maquinas de cambios}

    \lipsum[1-2]

    \begin{algorithm}[hbt!]
        \caption{Switch algorithm}\label{alg:SW}
        \DontPrintSemicolon
        %\SetAlgoLined
        \SetNoFillComment
        \LinesNotNumbered 
        \For { Switch in Switches }
        {
            \tcc{All signals must point to switch}
            \Switch{ Switch.Type }
            {
                \Case{Start}
                {
                    [Signals] $\gets$ ADD circulation signal\;
                    [Signals] $\gets$ ADD maneuver signal\;
                }
                \Case{Continue branch}
                {
                    [Signals] $\gets$ ADD circulation signal\;
                }
                \Case{Detour branch}
                {
                    [Signals] $\gets$ ADD maneuver signal\;
                }
            }   
        }
        \KwResult{[Signals]} 
    \end{algorithm}

    \lipsum[1]
    \includegraphics{example-image}\\
    \lipsum[1-2]