\subsection{Limpieza de señales}
	\label{sec:limpieza}

	% Autoridad > derecho limitado a una porcion
	% Claridad > autoridad no ambigua
	% Anticipacion > avisar con antelacion
	% Granularidad > rutas cortas y funcionales
	% Terminalidad > avisar fin de via
	% Infraestructura > avisar de infraestructura
	% No bloqueo > circulacion fluida
	
	El Algoritmo \ref{alg:vertical} de simplificación por herencia vertical y el Algoritmo \ref{alg:horizontal} de simplificación por herencia horizontal tienen diferentes efectos en la cantidad de señales totales. El Algoritmo de herencia horizontal mueve las señales entre elementos cuando no existe suficiente espacio entre ellos. Sin embargo, el Algoritmo de herencia vertical duplica la cantidad de señales si se aplica el criterio \#2 o el \#4. Todos los algoritmos que utiliza el RNA pueden aplicarse de forma iterativa si las condiciones lo permiten, pero particularmente el Algoritmo \ref{alg:vertical} duplicará las señales por cada nivel de profundidad de la red ferroviaria. Esto llevará a tener mas señales que las necesarias en algunos netElements. Este exceso de señales deben eliminarse para mantener una proporción uno a uno entre una señal y una funcionalidad específica. El Algoritmo \ref{alg:reduction} se ocupa de la limpieza de las señales que se encuentran muy próximas entre si, eliminando siempre las que tengan menor prioridad. 

	\begin{algorithm}[hbt!]
        \caption{Signal reduction algorithm}\label{alg:reduction}
        \DontPrintSemicolon
        %\SetAlgoLined
        \SetNoFillComment
        \LinesNotNumbered 
        Sources = $\{$ T,L,S,B,P,C,J,X $\}$\;
        \For{ Sig\_A \& Sig\_B in [Signals]}
        {
            \If{ A != B  \& Sig\_A.From == Sig\_B.From }
            {
             Signalling\_zones.Pos = detect\_next\_dangers ( Sig\_A )\;
             A.Pos,B.Pos = Sig\_A.Pos,Sig\_B\;
             Safe = zone\_between(Signalling\_zones.Pos,A.Pos,B.Pos)\;
                 \If{ NOT safe }
                 {
                    \If{ $|A.Pos,B.Pos|$ $<$ MIN\_DISTANCE }
                    {
                        \If{ Sig\_A \& Sig\_B same orientation }
                        {
                            \If{ Sig\_A \& Sig\_B same direction }
                            {
                                \If { Sig\_A \& Sig\_B type "Circ" }
                                {
                                    \If{ Sig\_A.Idx $<$ Sig\_B.Idx }
                                    {
                                        delete Sig\_B  else Sig\_A
                                    }
                                    %{
                                    %    delete( Sig\_A )
                                    %}
                                }
                            }
                        }
                        \If{ Sig\_A \& Sig\_B different source}
                        {
                            delete\_lowest\_priority( Sig\_A,Sig\_B )\;
                        }
                    }
                }
            }
        }
        \KwResult{[Signals]} 
    \end{algorithm}

	La prioridad es definida según el origen de cada señal. Como se explicó en la Sección \ref{sec:generacion}, las señales pueden tener diferentes etiquetas en función de que elemento ferroviario están protegiendo. Las señales S C B y H protegen los cambios de vías, las señales P a las plataformas, las señales J a las junturas, las señales X a los cruces de vías. Todas ellas son señales de circulación, salvo las señales B y H que son de maniobras. La prioridad mas alta fue asignada a las señales que protegen los finales de vía absolutos (T) y relativos (L) porque son las únicas señales de parada que no pueden ser reemplazadas por señales menos restrictivas. 
	
	Esto no significa que todas las señales que protegen los finales de vía (T) eliminarán todas las señales de los pasos a nivel (X), sino que de tener otra señal cercana del mismo tipo que apunte en la misma dirección y sentido, pero de mayor prioridad, será ésta quien cumpla la función de proteger al paso a nivel. Las señales pueden tener diferentes roles, como ser señales de parada, de circulación, de partida o maniobra. Las señales de parada son absolutas y se utilizan en finales de vía, mientras que las señales de partida se utilizan para reiniciar la marcha de la formación. Las primeras poseen un aspecto rojo, mientras que las segundas pueden tener hasta tres aspectos. Las señales de circulación poseen tres aspectos y permiten velocidades mas altas en las ramas principales de la red. Las señales de maniobra poseen solamente dos aspectos y son utilizadas en los desvíos, tanto en ramas divergentes como en convergentes.
	
	En el caso de que dos señales que tengan igual dirección, sentido y origen, pero no tengan una región de señalamiento entre ellas, deban ser simplificadas. En este caso, se decidió eliminar la señal con índice mas alto. Esto se debe a que, una vez generado el señalamiento para cada elemento, las señales de mayor índice serán las producidas por los Algoritmos de herencia horizontal y vertical, que tiendan a añadir señales que con mucha probabilidad deben ser eliminadas a posteriori.
	
    Es importante resaltar que la asignación de prioridades es de vital importancia para la seguridad. Por ejemplo, que las señales B tengan mayor prioridad que las señales P. Si la plataforma se encuentra lo suficientemente alejada de un cambio de vías, una señal de maniobra que protege el cambio de vías no puede reemplazar una señal de circulación que protege a una plataforma porque siempre habrá una plataforma entre ambos y, por lo tanto, una región de señalamiento. Sin embargo, si la plataforma se encuentra lo suficientemente cerca del cambio de vías, la señal de maniobras B se movería hacia la región de señalamiento donde se encuentra la señal de circulación P, debido al Algoritmo \ref{alg:horizontal} de herencia horizontal por el cual dos elementos ferroviarios muy próximos combinan sus regiones de señalamiento. Por lo tanto, la señal B de dos aspectos reemplazara a la señal P de tres aspectos, reduciendo la velocidad máxima permitida en esa zona de la red ferroviaria, incrementando la seguridad de la red.	