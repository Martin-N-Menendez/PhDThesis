\chapter{Conclusiones}

\section{Resultados obtenidos}

% RailML
La adopción de un enfoque geográfico basado en railML 3.2 como eje central de este trabajo permitió desarrollar algoritmos flexibles para procesar múltiples topologías ferroviarias prácticamente sin restricciones de tipo o cantidad de elementos ferroviarios incluidos. Para implementar esto se desarrolló una librería que cubre las 524 clases de railML, lo cual amplía el estado del arte de topologías ferroviarias analizadas automáticamente de unos pocos elementos (ver Sección \ref{sec:estudios}) a la totalidad de elementos ferroviarios contemplados en el estándar railML. 

El diseño modular de las herramientas creadas, basado en railTopoModel, permite que sea fácil de ampliar y/o mantener en futuras versiones de railML. Adicionalmente, utilizar un estándar ferroviario ampliamente utilizado permite una gran compatibilidad con las herramientas que la industria ferroviaria utiliza actualmente, lo que abre la puerta a que la herramienta sea adoptada sin dificultad tanto por la industria como por la comunidad de railML.org.

% RNA
El analizador de redes ferroviarias (RNA), explicado en el Capítulo \ref{sec:RNA}, genera el señalamiento apropiado, en línea con los principios de señalamiento ferroviarios adoptados a nivel internacional, incluso para topologías ferroviarias de considerable complejidad. La tabla de enclavamientos generada es acorde a la tabla de enclavamientos de referencia, si se utilizan los parámetros de configuración del RNA apropiados. En algunos casos, el RNA añade señales de protección extra para los elementos ferroviarios que fueron ignorados en el diseño de referencia, lo que incrementa la seguridad y mejora la logística de la red.

La efectividad del RNA fue confirmada junto con autoridades de Trenes Argentinos que compararon el señalamiento generado por el RNA para la estación Belgrano C con el señalamiento de dicha estación. Este logro se destaca por sobre otros sistemas de generación de señalamiento que solo permiten analizar redes ferroviarias de baja complejidad, con una cantidad limitada de elementos, como se mencionó en la Sección \ref{sec:estudios}.

% ACG
El generador automático de código (ACG), explicado en el Capítulo \ref{sec:ACG}, permite una generación modular y escalable de cualquier locación analizada por el RNA. Incluso los ejemplos mas complejos y extensos de railML.org (ejemplo 3 y ejemplo 4) no llegan a utilizar ni un tercio de los recursos de una FPGA de bajo coste y tamaño medio. Todos los sistemas generados cuentan con: UART adaptativa, detección de tramas, triple redundancia a nivel del módulo central del sistema de enclavamiento (ver Sección \ref{sec:network}), control de estado de enclavamientos y descentralización de las decisiones en cada elemento ferroviario.

La implementación del sistema se realiza de forma tal que solo se instancian los puertos y módulos que el RNA detectó previamente como existentes, lo que ahorra una gran cantidad de recursos, como se explicó en la Sección \ref{sec:interlockingArch}. Adicionalmente, la implementación del comportamiento dinámico del enclavamiento basado en redes de Petri permite que se implementen la totalidad de las funcionalidades esperables en el control de rutas: pedido de rutas, cancelación de rutas, protección por aproximación, protección por solape, cancelación por timeout, protección de rutas opuestas y concatenación de rutas.

% AGG
El generador automático de interfaz gráfica de usuario (AGG), explicado en el Capítulo \ref{sec:AGG}, permite contar con una interfaz 100\% basada en railML. El entorno gráfico es intuitivo, a semejanza de una consola real que utilizaría un ingeniero de señalamiento en su día a día, como se mostró en la Figura \ref{fig:EJ1_AGG}. Es posible para el operador desplazarse por la interfaz de manera muy sencilla y también implementa el uso de zoom. El usuario puede interactuar con todos los elementos visibles en la interfaz: Las vías pueden ser ocupadas/desocupadas, los cambios de vías pueden modificar su posición, las barreras ferroviarias pueden cerrarse/abrirse y las rutas pueden solicitarse/cancelarse presionando las señales correspondientes. Cada interacción modifica en tiempo real la trama a ser enviada a la FPGA, cuya respuesta es leída y actualiza la interfaz gráfica con mínimas demoras, logrando una experiencia fluida.

% Impacto real
Finalmente, el uso combinado de todas las herramientas creadas (RNA, ACG y AGG) permite la generación, validación e implementación en pocos segundos del señalamiento ferroviario en la totalidad de los ejemplos analizados, tanto oficiales del estándar railML cómo los creados en base a estaciones reales de Argentina. El análisis de la red ferroviaria cuenta con diversos parámetros, ajustables por el usuario, que permiten adaptar la herramienta a las normativas de diseño de cada país. Adicionalmente, la validación de cada regla de señalamiento ferroviario permite un diseño robusto y seguro. El código VHDL generado es compatible con cualquier familia de FPGAs y puede ser sintetizado en menos de cinco minutos. Esto reduce el tiempo de diseño e implementación del orden de meses o semanas a pocos minutos. La interfaz gráfica generada a medida de cada locación es ideal tanto para realizar pruebas de gran complejidad sobre la plataforma real, como también para controlar un sistema ferroviario real.

\section{Trabajo futuro}

A futuro se buscará reforzar la verificación y validación de las herramientas por medio de métodos formales. Precisamente el Mg. Lic. Santiago Germino, que es parte de este grupo de trabajo, se encuentra actualmente trabajando en este tema. De hecho, su trabajo sobre conversión del RNA en una herramienta que explote las características del formato XML de railML ya ha sido publicado \cite{Paper_207}. Lo cual permitirá un mayor uso de la herramienta en entornos ferroviarios y su certificación para uso industrial.

Adicionalmente, se buscará adaptar el ACG para poder incorporar otros protocolos de comunicación mas robustos, como RS-485 o MVB (explicados en la Sección \ref{sec:UART}), e implementar el sistema en la plataforma de hardware desarrollada por la Comisión Nacional de Energía Atómica (CNEA), actualmente en posesión del Laboratorio de Sistemas Embebidos, que fue presentado en la Seccion \ref{sec:GICSAFE}. De esta manera, se podrán realizar pruebas reales en campo del sistema de enclavamientos generado, en locaciones reducidas y de poco tráfico, para su validación por parte de un operador ferroviario real.

%A\\
%\cite{Paper_1}\cite{Paper_3}\cite{Paper_6}\cite{Paper_7}\cite{Paper_10}\cite{Paper_11}\cite{Paper_14}\cite{Paper_16}\cite{Paper_18}\cite{Paper_20}\\
%\cite{Paper_27}\cite{Paper_33}\cite{Paper_39}\cite{Paper_40}\cite{Paper_44}\cite{Paper_45}\cite{Paper_48}\cite{Paper_50}\cite{Paper_59}\cite{Paper_51}\\
%\cite{Paper_55}\cite{Paper_57}\cite{Paper_58}\cite{Paper_60}\cite{Paper_61}\cite{Paper_63}\cite{Paper_67}\cite{Paper_52}\cite{Paper_104}\cite{Paper_106}\\
%\cite{Paper_56}\cite{Paper_62}\cite{Paper_69}\cite{Paper_92}\cite{Paper_111}\cite{Paper_110}\cite{Paper_108}\cite{Paper_105}\cite{Paper_93}\\
%\cite{Paper_113}\cite{Paper_85}\cite{Paper_91}\cite{Paper_90}\cite{Paper_139}\cite{Paper_129}\cite{Paper_121}\cite{Paper_119}\cite{Paper_137}\\
%\cite{Paper_134}\cite{Paper_145}\cite{Paper_147}\cite{Paper_148}\cite{Paper_152}\cite{Paper_172}\cite{Paper_173}\cite{Paper_174}\cite{Paper_177}\\
%\cite{Paper_178}\cite{Paper_153}\cite{Paper_166}\cite{Paper_167}\\


%\cite{Paper_2}\cite{Paper_4}\cite{Paper_5}\cite{Paper_8}\cite{Paper_9}\cite{Paper_15}\\
%\cite{Paper_12}\cite{Paper_13}\cite{Paper_17}\cite{Paper_19}\\
%\cite{Paper_21}\cite{Paper_22}\cite{Paper_23}\cite{Paper_24}\cite{Paper_25}\cite{Paper_26}\cite{Paper_28}\cite{Paper_29}\\
%\cite{Paper_30}\cite{Paper_31}\cite{Paper_32}\cite{Paper_34}\cite{Paper_35}\cite{Paper_36}\cite{Paper_37}\cite{Paper_38}\\
%\cite{Paper_41}\cite{Paper_42}\cite{Paper_43}\cite{Paper_46}\cite{Paper_47}\cite{Paper_49}\\
%\cite{Paper_53}\cite{Paper_54}\cite{Paper_64}\cite{Paper_65}\cite{Paper_66}\cite{Paper_68}\\
%\cite{Paper_70}\cite{Paper_71}\cite{Paper_72}\cite{Paper_73}\cite{Paper_74}\cite{Paper_75}\cite{Paper_76}\cite{Paper_77}\cite{Paper_78}\cite{Paper_79}\\
%\cite{Paper_80}\cite{Paper_81}\cite{Paper_82}\cite{Paper_83}\cite{Paper_84}\cite{Paper_86}\cite{Paper_87}\cite{Paper_88}\cite{Paper_89}\\
%\cite{Paper_94}\cite{Paper_95}\cite{Paper_96}\cite{Paper_97}\cite{Paper_98}\cite{Paper_99}\\
%\cite{Paper_100}\cite{Paper_101}\cite{Paper_102}\cite{Paper_103}\cite{Paper_107}\cite{Paper_109}\\
%\cite{Paper_112}\cite{Paper_114}\cite{Paper_115}\cite{Paper_116}\cite{Paper_117}\cite{Paper_118}\\
%\cite{Paper_120}\cite{Paper_122}\cite{Paper_123}\cite{Paper_124}\cite{Paper_125}\cite{Paper_126}\cite{Paper_127}\cite{Paper_128}\\
%\cite{Paper_130}\cite{Paper_131}\cite{Paper_132}\cite{Paper_133}\cite{Paper_135}\cite{Paper_136}\cite{Paper_138}\\
%\cite{Paper_140}\cite{Paper_141}\cite{Paper_142}\cite{Paper_143}\cite{Paper_144}\cite{Paper_146}\cite{Paper_149}\\
%\cite{Paper_150}\cite{Paper_151}\cite{Paper_154}\cite{Paper_155}\cite{Paper_156}\cite{Paper_157}\cite{Paper_158}\cite{Paper_159}\\
%\cite{Paper_160}\cite{Paper_161}\cite{Paper_162}\cite{Paper_163}\cite{Paper_164}\cite{Paper_165}\cite{Paper_168}\cite{Paper_169}\\
%\cite{Paper_170}\cite{Paper_171}\cite{Paper_175}\cite{Paper_176}\cite{Paper_179}\\
%\cite{Paper_180}\cite{Paper_181}\cite{Paper_182}\cite{Paper_183}\cite{Paper_184}\cite{Paper_185}\cite{Paper_186}\cite{Paper_187}\cite{Paper_188}\cite{Paper_189}\\
%\cite{Paper_190}\cite{Paper_191}\cite{Paper_192}\cite{Paper_193}\cite{Paper_194}\cite{Paper_195}\cite{Paper_196}\cite{Paper_197}\cite{Paper_198}\cite{Paper_199}\\
%\cite{Paper_200}\cite{Paper_201}\cite{Paper_202}\cite{Paper_203}\cite{Paper_204}\cite{Paper_205}\cite{Paper_206}\cite{Paper_207}\cite{Paper_208}\\