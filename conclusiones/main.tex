\chapter{Conclusiones}

\section{Resultados obtenidos}

% RailML
\lipsum[1]

% RNA
\lipsum[1]

% ACG
\lipsum[1]

% AGG
\lipsum[1]

% Impacto real
\lipsum[1]

%The RNA algorithm generates appropriate signalling, in accordance with the signalling principles defined, in layouts with considerable complexity. The interlocking tables generated are indistinguishable from the original ones under the right RNA configuration parameters and, in some cases, add extra protection for railway elements that were ignored in the original signallings, increasing safety and mobility. This performance was confirmed by an actual railway practitioner generating signalling for Belgrano C station using RNA and then comparing it with the actual signalling recently updated. This is a remarkable achievement considering that previous works on automatic signalling generation work only in low-complexity layouts having a limited number of element types. 

%The graph network from RailTopoModel, introduced in railML 3.2, allowed us to develop flexible algorithms to process multiple railway networks with no restriction on element types. What is more, the railML files generated pass our own and 3rd party syntax checker as well, confirming its full compatibility with any railML-based software. RNA expands the state of the art of railML to areas that were not discussed before in other articles, being able to be incorporated into a more complex design process, and it is feasible to be formally validated in future development steps.

%We already published a first version of an RNA implementation using XSLT \cite{XSLT}, a language specifically designed to process XML files like railML. We are working now to include in the XSLT implementation an automatic verification and validation for each step of the RNA.

\section{Trabajo futuro}

\lipsum[1]

A\\
\cite{Paper_1}\cite{Paper_3}\cite{Paper_6}\cite{Paper_7}\cite{Paper_10}\cite{Paper_11}\cite{Paper_14}\cite{Paper_16}\cite{Paper_18}\cite{Paper_20}\\
\cite{Paper_27}\cite{Paper_33}\cite{Paper_39}\cite{Paper_40}\cite{Paper_44}\cite{Paper_45}\cite{Paper_48}\cite{Paper_50}\cite{Paper_51}\cite{Paper_52}\\
\cite{Paper_55}\cite{Paper_56}\cite{Paper_57}\cite{Paper_58}\cite{Paper_59}\cite{Paper_60}\cite{Paper_61}\cite{Paper_62}\cite{Paper_63}\cite{Paper_67}\\
\cite{Paper_69}\cite{Paper_85}\cite{Paper_90}\cite{Paper_91}\cite{Paper_92}\cite{Paper_93}\cite{Paper_104}\cite{Paper_105}\cite{Paper_106}\cite{Paper_108}\\
\cite{Paper_110}\cite{Paper_111}\cite{Paper_113}\cite{Paper_119}\cite{Paper_121}\cite{Paper_129}\cite{Paper_134}\cite{Paper_137}\cite{Paper_139}\cite{Paper_145}\\
\cite{Paper_147}\cite{Paper_148}\cite{Paper_152}\cite{Paper_153}\cite{Paper_166}\cite{Paper_167}\cite{Paper_172}\cite{Paper_173}\cite{Paper_174}\cite{Paper_177}\cite{Paper_178}\\

%\cite{Paper_2}\cite{Paper_4}\cite{Paper_5}\cite{Paper_8}\cite{Paper_9}\cite{Paper_15}\\
%\cite{Paper_12}\cite{Paper_13}\cite{Paper_17}\cite{Paper_19}\\
%\cite{Paper_21}\cite{Paper_22}\cite{Paper_23}\cite{Paper_24}\cite{Paper_25}\cite{Paper_26}\cite{Paper_28}\cite{Paper_29}\\
%\cite{Paper_30}\cite{Paper_31}\cite{Paper_32}\cite{Paper_34}\cite{Paper_35}\cite{Paper_36}\cite{Paper_37}\cite{Paper_38}\\
%\cite{Paper_41}\cite{Paper_42}\cite{Paper_43}\cite{Paper_46}\cite{Paper_47}\cite{Paper_49}\\
%\cite{Paper_53}\cite{Paper_54}\cite{Paper_64}\cite{Paper_65}\cite{Paper_66}\cite{Paper_68}\\
%\cite{Paper_70}\cite{Paper_71}\cite{Paper_72}\cite{Paper_73}\cite{Paper_74}\cite{Paper_75}\cite{Paper_76}\cite{Paper_77}\cite{Paper_78}\cite{Paper_79}\\
%\cite{Paper_80}\cite{Paper_81}\cite{Paper_82}\cite{Paper_83}\cite{Paper_84}\cite{Paper_86}\cite{Paper_87}\cite{Paper_88}\cite{Paper_89}\\
%\cite{Paper_94}\cite{Paper_95}\cite{Paper_96}\cite{Paper_97}\cite{Paper_98}\cite{Paper_99}\\
%\cite{Paper_100}\cite{Paper_101}\cite{Paper_102}\cite{Paper_103}\cite{Paper_107}\cite{Paper_109}\\
%\cite{Paper_112}\cite{Paper_114}\cite{Paper_115}\cite{Paper_116}\cite{Paper_117}\cite{Paper_118}\\
%\cite{Paper_120}\cite{Paper_122}\cite{Paper_123}\cite{Paper_124}\cite{Paper_125}\cite{Paper_126}\cite{Paper_127}\cite{Paper_128}\\
%\cite{Paper_130}\cite{Paper_131}\cite{Paper_132}\cite{Paper_133}\cite{Paper_135}\cite{Paper_136}\cite{Paper_138}\\
%\cite{Paper_140}\cite{Paper_141}\cite{Paper_142}\cite{Paper_143}\cite{Paper_144}\cite{Paper_146}\cite{Paper_149}\\
%\cite{Paper_150}\cite{Paper_151}\cite{Paper_154}\cite{Paper_155}\cite{Paper_156}\cite{Paper_157}\cite{Paper_158}\cite{Paper_159}\\
%\cite{Paper_160}\cite{Paper_161}\cite{Paper_162}\cite{Paper_163}\cite{Paper_164}\cite{Paper_165}\cite{Paper_168}\cite{Paper_169}\\
%\cite{Paper_170}\cite{Paper_171}\cite{Paper_175}\cite{Paper_176}\cite{Paper_179}\\
%\cite{Paper_180}\cite{Paper_181}\cite{Paper_182}\cite{Paper_183}\cite{Paper_184}\cite{Paper_185}\cite{Paper_186}\cite{Paper_187}\cite{Paper_188}\cite{Paper_189}\\
%\cite{Paper_190}\cite{Paper_191}\cite{Paper_192}\cite{Paper_193}\cite{Paper_194}\cite{Paper_195}\cite{Paper_196}\cite{Paper_197}\cite{Paper_198}\cite{Paper_199}\\
%\cite{Paper_200}\cite{Paper_201}\cite{Paper_202}\cite{Paper_203}\cite{Paper_204}\cite{Paper_205}\cite{Paper_206}\cite{Paper_207}\cite{Paper_208}\\